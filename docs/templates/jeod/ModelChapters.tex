\setcounter{chapter}{0}

%----------------------------------
\chapter{Introduction}\hyperdef{part}{intro}{}\label{ch:intro}
%----------------------------------


\section{Purpose and Objectives of \ModelNameDesc}
%%% Incorporate the intro paragraph that used to begin this Chapter here.
%%% This is location of the true introduction where you explain why this 
%%% document exists and what it hopes to accomplish.

\section{Context within JEOD}
%%% Identify the Document/Model context within JEOD.
%%% Include a diagram of where the document is located in the 
%%% directory structure.
%%% Mention any model specific related documents and
%%% that JEOD.pdf is the parent document.

\section{Documentation History}
%%% Status of this and only this document.  Any date should be relevant to when 
%%% this document was last updated and mention the reason (release, bug fix, etc.)
%%% Mention previous history aka JEOD 1.4-5 heretage in this section.

\begin{tabular}{||l|l|l|l|} \hline
\DocumentChangeHistory
\end{tabular}

\section{Documentation Organization}
This document is formatted in accordance with the 
NASA Software Engineering Requirements Standard~\cite{NASA:SWE} 
and is organized into the following chapters:

\begin{description}
%% longer chapter descriptions, more information.

\item[Chapter 1: Introduction] - 
This introduction contains four sections; objective and purpose, contex within JEOD, status, and organization.  
The section titled objective and purpose provides the true introduction to the \ModelNameDesc\ and its reason 
for existance.  The next section identifies the context within JEOD of this document and the \ModelNameDesc.  
This is achieved with a directory diagram, a brief description of the interconnections with other models, and 
references to any supporting documents.  This is followed by the status of the document which includes
author, date, and reason for each revision.  Finally, there is a description of the how the document is organized.

\item[Chapter 2: Product Requirements] - 
Describes requirements for the \ModelNameDesc.

\item[Chapter 3: Product Specification] - 
Describes the underlying theory, architecture, and design of the \ModelNameDesc\ in detail.  It will be organized in 
three sections; Conceptual Design, Mathematical Formulations, and Detailed Design.

\item[Chapter 4: User Guide] - 
Describes how to use the \ModelNameDesc\ in a Trick simulation.  It is broken into three sections to represent the JEOD 
defined user types; Analysts or users of simulations (Analysis), Integrators or developers of simulations (Integration), 
and Model Extenders (Extension).

\item[Chapter 5: Verification and Validation] -  
Contains \ModelNameDesc\ verification and validation procedures and results.

\end{description}

%----------------------------------
\chapter{Product Requirements}\hyperdef{part}{reqt}{}\label{ch:reqt}
%----------------------------------
This model shall meet the JEOD project requirements specified in the 
\hyperref{file:\JEODHOME/docs/JEOD.pdf}{part1}{reqt}{JEOD} top-level document.

%%% Format for the model Requirements is open.  It should include requirements for this model 
%%% only and use requirment tags like the one below.
%\requirement{...}
%\label{reqt:...}
%\begin{description}
%  \item[...]\ \newline
%    The documentation for the model shall include
%
%    \subrequirement{}
%    \label{reqt:...}
%      Software requirements specification.
%      
%    ...
%   
%  \item[title]\ \newline
%    text
%
%  ...
%
%\end{description}

%----------------------------------
\chapter{Product Specification}\hyperdef{part}{spec}{}\label{ch:spec}
%----------------------------------

\section{Conceptual Design}

\section{Mathematical Formulations}

\section{Detailed Design}

%----------------------------------
\chapter{User Guide}\hyperdef{part}{user}{}\label{ch:user}
%----------------------------------
The Analysis section of the user guide is intended primarily for users of pre-existing simulations.  
It contains: 
\begin{itemize}
\item a description of how to modify \ModelNameDesc\ variables after the simulation 
has compiled, including an in-depth discussion of the input file,
\item an overview of how to interpret (but not edit) the S\_define file,
\item a sample of some of the typical variables that may be logged.
\end{itemize}

The Integration section of the user guide is intended for simulation developers.  
It describes the necessary configuration of the \ModelNameDesc\ within an 
S\_define file, and the creation of standard run directories.  The latter 
component assumes a thorough understanding of the preceding Analysis section of the user guide.
Where applicable, the user may be directed to selected portions of Product Specification (Chapter \ref{ch:spec}).

The Extension section of the user guide is intended primarily for developers 
needing to extend the capability of the \ModelNameDesc\.  Such users should have a 
thorough understanding of how the model is used in the preceding 
Integration section, and of the model 
specification (described in Chapter \ref{ch:spec}).

\section{Analysis}


\section{Integration}


\section{Extension}


%----------------------------------
\chapter{Verification and Validation}\hyperdef{part}{ivv}{}\label{ch:ivv}
%----------------------------------

\section{Verification}
%%% code imported from old template structure
%\inspection{<Name of Inspection>}\label{inspect:<label>}
% <description> to satisfy  
% requirement \ref{reqt:<label>}.

\section{Validation}
%%% code imported from old template structure
%\test{<Title>}\label{test:<label>}
%\begin{description}
%\item[Purpose:] \ \newline
%<description>
%\item[Requirements:] \ \newline
%By passing this test, the universal time module 
%partially satisfies requirement~\ref{reqt:<label1>} and 
%completely satisfies requirement~\ref{reqt:<label2>}.
%\item[Procedure:]\ \newline
%<procedure>
%\item[Results:]\ \newline
%<results>
%\end{description}
