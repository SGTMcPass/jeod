%&latex

\documentclass[10pt]{ltxdoc}
\DisableCrossrefs
\OnlyDescription

\newcommand\ModelPrefix{bogus}

\usepackage[section,nosigpage,fancyvrb]{dynenv}


% Bring in the hyper ref environment
\usepackage[colorlinks,bookmarks]{hyperref}
\hypersetup{
   pdftitle={The dynenv Class and dynenv Package},
   pdfauthor={David Hammen, Odyssey Space Research, LLC}}

\MakeShortVerb{|}


\newcommand\package[1]{%
  \texorpdfstring{\textsf{#1}\xspace}{#1}}
\newcommand\dynenv{\package{dynenv}}
\newcommand\command[1]{%
  \texorpdfstring{\texttt{\textbackslash{}#1}\xspace}{\textbackslash{}#1}}
\newcommand{\inanglebrackets}[1] {\textless{}#1\textgreater{}}
\newcommand\option[1]{\texttt{#1}}
\newcommand{\cmdarg}[1] {\texttt{\{\inanglebrackets{#1}\}}}
\newcommand{\cmdopt}[1] {\texttt{[\inanglebrackets{#1}]}}

\makeatletter

\def\@maketitle{%
  \newpage
%  \null
    \vspace{-1.0em}%
  \begin{center}%
  \let \footnote \thanks
    {\LARGE \@title \par}%
    \vskip 1.5em%
    {\large
      \lineskip .5em%
      \begin{tabular}[t]{c}%
        \@author
      \end{tabular}\par}%
    \vskip 1em%
    {\large \@date}%
  \end{center}%
  \par
%  \vskip 0.5em%
}

\renewenvironment{abstract}{%
   \null\vfil
%   \@beginparpenalty\@lowpenalty
   \begin{center}%
      \bfseries \abstractname
%      \@endparpenalty\@M
   \end{center}}%
{%
   \par
   \null\vfil
%   \@beginparpenalty\@lowpenalty
   \begin{center}%
      \bfseries \contentsname
%      \@endparpenalty\@M
   \end{center}}


\renewcommand\tableofcontents{%
   \vspace{-1ex}
   \@starttoc{toc}%
}

\renewenvironment{frontmatter}{\@twocolumntrue}{\onecolumn}

%\newcounter{exhibit}
%\newcommand\exhibitname{Exhibit}
\renewenvironment{exhibit}
                 {\@float{exhibit}}
                 {\end@float}
 
\makeatother

\setcounter{tocdepth}{2}

\begin{document}


\pdfbookmark{Title Page}{titlepage}

\newbox\abstractbox
\setbox\abstractbox=\vbox{
  \begin{abstract}
    All JEOD model documents use the \dynenv suite to simplify document
    specification and to give the document package a uniform appearance.
    The suite provides tools that aid in document production, in naming and
    referencing other JEOD models, and in specifying and referencing
    requirements and IV\&V artifacts.
  \end{abstract}}

\title{The \dynenv Class and \dynenv Package}
\author{David Hammen\\Odyssey Space Research, LLC}
\date{06/06/11\\\box\abstractbox}

\begin{frontmatter}
\maketitle
\tableofcontents
\end{frontmatter}

\section{Introduction}
This note describes the key capabilities of the JEOD \LaTeX\ \dynenv class and
\dynenv package.
All JEOD model documents must use either the \dynenv class or package.
The suite has four primary purposes:
\begin{itemize}
\item Simplify the process of specifying a model document.
\item Optionally load a common set of packages.
This reduces the work and knowledge burden on model document authors
and helps provide a common look to the JEOD documentation suite.
\item Provide a standardized way to name and reference JEOD models and
JEOD documents.
\item Provide a standardized way to represent and reference
requirements, inspections, and tests.
\end{itemize}

\section{Using the Suite}
\label{sec:usage}

The \dynenv suite can be used as a class or as a package. The syntax of
the two approaches is quite similar:

\begin{quote}
|\documentclass[<options>]{dynenv}|
\end{quote}
versus
\begin{quote}
|\usepackage[<options>]{dynenv}|
\end{quote}

\subsection{The \dynenv Class}
The \dynenv class is a very simple \LaTeX\ class.
The body of |dynenv.cls|, sans comments is:
\begin{quote}
|\LoadClass[twoside,11pt,titlepage]{report}|\\
\ \\
|\PassOptionsToPackage{complete}{dynenv}|\\
|\DeclareOption*{\PassOptionsToPackage{\CurrentOption}{dynenv}}|\\
|\ProcessOptions\relax|\\
\ \\
|\RequirePackage{dynenv}|
\end{quote}

The class re-implements the \package{report} class with a fixed set of options,
currently |twoside,11pt,titlepage|,
and then imports the \dynenv package.
All options to the \dynenv class, plus the |complete| option,
are passed on to the \dynenv package.

\subsection{The \dynenv Package}
The \dynenv package is the workhorse of the \dynenv suite.
The remainder of this document describes features provided by
the \dynenv package.

\subsection{Recommended Practice}
The recommended practice is to use the \dynenv class in conjunction with the
document production macros described in section~\ref{sec:doc_production}.
Doing so results in a very short and very comprehensible main document.

\subsection{Class and Package Options}
\label{sec:options}
All options provided to the \dynenv class are passed
on without processing to the \dynenv package.
The options to the \dynenv package fall into two categories:
Options that have an explicit option handler, and arbitrary options that do not have a handler.
The options with explicit handlers are:
\begin{description}
\item[|sigpage|] Passed to the \package{dyncover} package.
When specified, the document will have a signature page.
\item[|nosigpage|] Passed to the \package{dyncover} package.
When specified, the document will not have a signature page.
Model documents should not specify either the |sigpage| or |nosigpage| option.
They should instead rely on the default. 
\item[|section|] Indicates that the highest-level sectioning command
in the document is |\section|. This document, for example.
Model documents should never use the |section| option.
\item[|sections|] An alias for |section|.
\item[|part|] Indicates that the highest-level sectioning command
in the document is |\part|. The documents for large models are sometimes
written in parts to make the document more manageable.
\item[|parts|] An alias for |part|.
\item[|appendix|] Specifying this option causes the \dynenv suite to load
the \package{appendix} package and modifies the behavior of the |backmatter|
macro to suit the use of that package.
\item[|complete|] Causes the \dynenv package to load a set of commonly used
packages, the auto-generated |paths.def| file, the model-specific
package, and finally, the \package{hyperref} package. (The \package{hyperref}
should always be the very last package that is loaded in any \LaTeX
document.)
\item[|full|] An alias for |complete|.
\item[|hyperref|] This is explicitly marked as an illegal option.
Never specify |hyperref| as an option.
\end{description}
All options other than those listed above indicate additional packages
to be loaded by the \dynenv package. These additional packages are loaded
without any options. If a model document needs to use some non-standard package
with options specified, the model-specific package file should use the
|\RequirePackage| macro to load the package.

\section{Packages and Files Loaded by the \dynenv Package}
\label{sec:packages}
The \dynenv package loads a number of standard \LaTeX\ packages plus additional
packages and files as indicated by the options to the \dynenv class or package.
These packages and files are described below, in the order in which they
are loaded by the \dynenv package. The descriptions below are brief.
Users who want to obtain more information on one of the standard packages
are encouraged to read the documentation provided by the package developer
by typing |texdoc <package_name>| on the command line.

\subsection{Package \package{jeodspec}}
This package identifies the current JEOD release,
defines some simple shortcut macros,
and specifies the models that comprise the current release.
See section~\ref{sec:jeodspec} for details.

\subsection{Package \package{geometry}}
Specifying page layout in \LaTeX\ is a bit tricky and error-prone.
The \package{geometry} package solves these issues. The \dynenv package
uses the \package{geometry} package to specify a standard page layout
for JEOD documents. 

\subsection{Package \package{xspace}}
Consider the shortcut macro |\newcommand{fbz}{foo bar baz}|. Users of this
macro have to take care to add a hard space after invoking the macro in the
middle of a sentence but never to do so before punctuation. The \package{xspace}
package solves this problem. Define the macro as |\newcommand{fbz}{foo bar baz\xspace}| and voila, there is no need for that hard space.

\subsection{Package \package{xkeyval}}
The \package{xkeyval} package provides a convenient mechanism for specifying
and processing options to a class, package, or macro. The \dynenv package
uses the capabilities provided by \package{xkeyvals} to specify and process
the options to the document production macros
(see section~\ref{sec:doc_production}), plus a few other macros.

\subsection{Package \package{longtable}}
The \package{longtable} package provides the ability to create tables that
span multiple pages. Several of the automatically generated tables
in JEOD model documents are long tables.

\subsection{Package \package{appendix}}
The \package{appendix} package alters some of the base \LaTeX\ mechanisms
for presenting appendices and also provides the ability to have end of chapter
appendices.

The \package{appendix} package is not a part of the standard or extended sets of
packages loaded by the \dynenv package. The package is loaded only if the
|appendix| option is provided to the \dynenv class or package.
The \package{appendix} package should not be used in documents with parts.

\subsection{AMS Math Packages \package{amsmath}, \package{amssymb}, and \package{amsfonts}}
The \package{amsmath} package improves upon and augments the base
\LaTeX\ mechanisms for representing and displaying mathematical formulas
and is the basis for the American Mathematical Society \LaTeX\ suite.
The \package{amssymb} package defines miscellaneous mathematical symbols
beyond those provided by the \package{amsmath} package, and the
\package{amsfonts} package provides things such as
blackboard bold letters, Fraktur letters, and other miscellaneous symbols.

The AMS math packages are not a part of the standard set of packages
loaded by the \dynenv package but they are a part of the extended set
of packages (those enabled by specifying the \dynenv |complete| option).

\subsection{Package \package{graphicx}}
The \package{graphicx} package provides a range of capabilities. In JEOD,
the graphics bundle is used primarily to include graphics into a document
via the |includegraphics| macro.
The \package{graphicx} package is not a part of the standard set of packages
loaded by the \dynenv package but it is part of the extended set
of packages, enabled by specifying the \dynenv |complete| option.

\subsection{User-Specified Packages}
All options provided to the \dynenv package other than those explicitly
handled by a option handler specify some non-standard package to be
loaded. See section~\ref{sec:options} for a list of options that have
explicit handlers.

One side effect of this handling of \dynenv package options is that
options that do have explicit handlers cannot be used to load a package.
That said, only three packages in the \LaTeX\ distribution are shadowed by
\dynenv package options:
\begin{description}
\item[|appendix|] The \dynenv package loads the \package{appendix} package
  when the |appendix| option is specified, so this shadowing doesn't count.
\item[|hyperref|] The \dynenv package explicitly makes |hyperref| an illegal
  option because the \package{hyperref} package by design needs to be the
  very last package loaded in a document. That the \package{hyperref} package
  is shadowed by the |hyperref| option is intentional.
\item[|section|] The \package{section} package changes the appearance of
  sectioning commands. JEOD documents should not use this package.
\end{description}

The use of the extended options capability is optional.
The mechanism is a bit opaque. Some authors would rather explicitly
invoke the |\RequirePackage| macro in the model-specific package file.
Another issue is that the packages loaded via this mechanism are loaded without
options. If options need to be specified those packages must be loaded
in the model-specific package file.

\subsection{Package \package{dynmath}}
The \package{dynmath} package provides macros that implement the
recommended practice for displaying vectors, matrices, and quaternions.
This package is not a part of the \LaTeX\ distribution but is a part
of the standard set of packages loaded by \dynenv.
Documentation for this package is available in the JEOD file
|docs/templates/jeod/dynmath.pdf|.

\subsection{Package \package{dyncover}}
The \package{dyncover} package creates cover pages for JEOD documents.
This package is not a part of the \LaTeX\ distribution but is a part
of the standard set of packages loaded by \dynenv.
Documentation for this package is, ummm, RTFC.

\subsection{File \package{paths.def}}
The standard model document makefile automatically generates the file
|paths.def|. This file defines a standard set of macros whose values depend
on the model and on the make target. (The relative paths lose one level of indirection when the file is built with |make install| versus |make|.)
The names defined in this file are used in |dynenv.sty| in various places.

As a fictitious example, consider the model located in
|$JEOD_HOME/models/interactions/psychoceramics|.
The contents of the |paths.def| file
for this model given the command |make| will be:
\begin{codeblock}
\ifx\model@pathsdef\endinput\endinput\else\let\model@pathsdef\endinput\fi|\\
\newcommand\JEODHOME{../../../../..}
\newcommand\MODELHOME{../..}
\newcommand\MODELDOCS{..}
\newcommand\MODELPATH{models/interactions/psychoceramics}
\newcommand\MODELTYPE{interactions}
\newcommand\MODELGROUP{interactions}
\newcommand\MODELDIR{psychoceramics}
\newcommand\MODELNAME{psychoceramics}
\newcommand\MODELTITLE{\PSYCHOCERAMICS}
\endinput
\end{codeblock}

The \dynenv package loads this file via |\input{paths.def}| when the
package is operating in |complete| mode. The burden of loading this file
otherwise falls on the model document author. Note that the file can safely
be included multiple times because the first line of |paths.def| is the
\TeX\ equivalent of the one-time include protection used in
\Cplusplus header files.

\subsection{Model-Specific Package}
One such place where the \dynenv package uses one of names defined in
|paths.def| is the very next operation the \dynenv package performs
when the package is operating in |complete| mode: It loads the model-specific
package file via |\RequirePackage{\MODELNAME}|. The model-specific
package file defines a standard set of macro names whose definitions are
model-specific, plus additional macros that are specific to the model.
See section~\ref{sec:model.sty} for details.
\subsection{Package \package{hyperref}}
The \package{hyperref} package makes all cross-references in a PDF document
into hyperlinks and adds the ability to insert hyperlinks to other documents.
The \package{hyperref} package is not a part of the standard set of packages
loaded by the \dynenv package but it is part of the extended set
of packages, enabled by specifying the \dynenv |complete| option.

To make the content linkable, the package needs to modify many macros created
by other packages. The \package{hyperref} is typically the very last package
loaded in a \LaTeX\ document to ensure that it can make those alterations.
The \dynenv package loads \package{hyperref} when the \dynenv package is
used per the recommended practice. This means that a |.tex| file that
follows the recommended practice must by necessity have a very simple
\LaTeX\ preface. There should be no |\usepackage| statements in the main
|.tex| file for a model document that follows the recommended practice.


\section{Package \package{jeodspec}}
\label{sec:jeodspec}
The \package{jeodspec} package defines commands that need to be updated with
every release and with additions to or deletions from the JEOD model suite.
The maintainers of this package will be JEOD administrators and model developers
rather than \TeX\ programmers. The contents of |jeodspec.sty| are designed
with this in mind. Extensive knowledge of \TeX\ is not required.

The first set of definitions in |jeodspec.sty| specify items that need
to be changed with each release:
\begin{description}
\item[\command{JEODid}] The name of the current (or upcoming) JEOD release.
\item[\command{RELEASEMONTH}] The name of the month when the release will occur.
\item[\command{RELEASEYEAR}] The four-digit year when the release will occur.
\end{description}

The next set of definitions specify items that need less frequent changes:
\begin{description}
\item[\command{JEODMANAGERS}] The names and titles of the people who
need to approve the JEOD documentation suite.
\item[\command{JEODORG}] The JSC organization under which JEOD is developed
and maintained.\item[\command{JEODJSCNUM}] The official JSC document number
for the JEOD documentation suite.
\end{description}

The third set of definitions create shortcut commands for text phrases used in
multiple documents. Some comments:
\begin{itemize}
\item Model authors:
Please do not add model-specific shortcuts to |jeodspec.sty|.
These shortcuts should be for widely-used terminology.
The model-specific .sty file is the place to define such shortcuts.
\item The expansion of the older items in this set do not end with
\command{xspace}. Please consider using \command{xspace} in future definitions.
\end{itemize}

The final set of definitions in |jeodspec.sty| identify the models that
comprise the current JEOD release. Each JEOD model is specified via the
\command{addmodel} macro. \command{addmodel} takes four arguments, as follows:
\begin{itemize}
\item The model name,
by convention an all-caps, letters-only identifier.
For example, this argument is |ATMOSPHERE| for the atmosphere model.
\item The model parent directory name, one of dynamics, environment,
interactions, or utils. The atmosphere model is located in
|$JEOD_HOME/models/environment/atmosphere|, so in this case the parent
directory name is |environment|.
\item The model directory name. For the atmosphere model this is |atmosphere|.
Do not escape underscores in the model directory name. The model directory
name for the Earth Lighting Model is |earth_lighting|, not
|earth\_lighting|.
\item The model title. This is often an initial caps version of the
model name, plus "Model". For example, the title of the atmosphere model
is |Atmosphere Model|.
\end{itemize}

\section{Document Production}
\label{sec:doc_production}
The \dynenv package provides three macros that simplify the specification of a
document, |frontmatter|, |mainmatter|, and |backmatter|. These macros write
the front matter, main matter, and back matter of a JEOD document.

\subsection{The Structure of a JEOD Document}
All JEOD document must have:
\begin{itemize}
\item Front matter than comprises:
  \begin{itemize}
  \item The cover pages,
  \item An abstract or executive summary,
  \item A table of contents, and
  \item Optional lists of figures and tables.
  \end{itemize}
\item Body matter that comprises five chapters labeled:
  \begin{itemize}
  \item Introduction,
  \item Product Requirements,
  \item Product Specification,
  \item User Guide, and
  \item Inspections, Tests, and Metrics.
  \end{itemize}
  Details regarding the contents of these chapters is specified elsewhere.
  Documents that have multiple parts should have these five chapters
  in each part.
\item Back matter that comprises:
  \begin{itemize}
  \item Optional appendices, and
  \item A bibliography.
  \end{itemize}
  The bibliography is constructed automatically from the references in the
  document, the JEOD \BibTeX\ file, and the model-specific \BibTeX\ file.
\end{itemize}

\subsection{Front Matter}
The \command{frontmatter} macro constructs the front matter. The macro takes
one optional and one mandatory argument.
\begin{codeblock}
\frontmatter[options]{abstractOrSummary}
\end{codeblock}
The options are a comma-separated list of elements of the form
|key=value| or just |key|. The keys are:
\begin{description}
\item[\option{style=<style>}] Specifies whether the document summary is written
  using the abstract environment or as an unnumbered chapter.
  Use |style=abstract| for the former, |style=summary| for the latter.
\item[\option{label=<name>}]Specifies the PDF label of the summary.
This option is only needed when the summary is written as an unnumbered chapter
and the name of that chapter is something other than ``Executive Summary.''
\item[\option{figures}] Causes the list of figures to be generated.
\item[\option{tables}] Causes the list of tables to be generated.
\end{description}
Note: The optional argument is not completely optional.
You must specify the |style|.

The mandatory argument specifies the name of the file (sans the |.tex| suffix)
the contains the abstract or summary.

\subsubsection{Sample Usage}
The following generates front matter with a list of tables (but no list of
figures) with the abstract contained in the file |ModelAbstract.tex|.
\begin{codeblock}
\frontmatter[style=abstract,tables]{ModelAbstract}
\end{codeblock}

The next example generates front matter with a list of figures and a list of
tables with an executive summary contained in the file |summary.tex|.
\begin{codeblock}
\frontmatter[style=summary,figures,tables]{summary}
\end{codeblock}

The final example generates front matter with no list of figures or list of
tables with an abstract in |chapter*| form contained in the file |abstract.tex|.
\begin{codeblock}
\frontmatter[style=summary,label=Abstract]{abstract}
\end{codeblock}

\subsubsection{Abstract or Executive Summary?}
The purpose of the abstract or executive summary is to inform the reader of
the relevance of the subject of the document to some problem the reader needs
to solve. This preface to the document should clearly but concisely summarize
the subject of the document.

Use an abstract if this summary description is very brief, four or five
paragraphs at most. An abstract cannot contain any sectioning commands.
Longer descriptions that need some internal structure should be written in an
executive summary form.

An executive summary can be considerably longer than four or five paragraphs
but should never be  more than 10\% of the document as a whole.
An executive summary must at a minimum start with a |\chapter*| command and may
contain |\section*| and lower level sectioning commands. Always use the starred
form of the sectioning commands in an executive summary.
While the outline of the main matter is highly structured, this is not the
case for an executive summary. You don't need no stinkin' requirements,
for example. Write in a manner that best summarizes the material.

\subsubsection{Sample Abstract}

The following sample abstract summarizes the fictitious psychoceramics model.

\begin{codeblock}
\begin{abstract}
The cracked pot interaction, first investigated by Josiah S. Carberry, is
typically a small perturbative force in spacecraft dynamics. Refinements of
the interaction have been proposed over the years by a number of authors.
Future developments in this field may well revolutionize space flight.
This is the golden age of psychoceramics! The Psychoceramics Model
implements models of several of the more promising of these techniques.
\end{abstract}
\end{codeblock}

Assuming the above is the contents of the |abstract.tex| file in the |docs/tex|
directory of the psychoceramics model, the toplevel |psychoceramics.tex| file could include this abstract via
\begin{codeblock}
\frontmatter[style=abstract,tables]{abstract}
\end{codeblock}


\subsubsection{Sample Executive Summary}

The executive summary form of the preface provides a more detailed
(but still summary) description of the subject matter. An executive summary
for the above fictitious model is presented below.

\begin{codeblock}
\chapter*{Executive Summary}
\section*{Overview}
The cracked pot interaction, first investigated by Josiah S. Carberry, is
typically a small perturbative force in spacecraft dynamics. Refinements of
the interaction have been proposed over the years by a number of authors.
Future developments in this field may well revolutionize space flight.
This is the golden age of psychoceramics! The Psychoceramics Model
implements models of several of the more promising of these techniques.

\section*{Implemented Interactions}
A wide variety of interactions are implemented in the model, far too many to
enumerate in this introductory note. One of the more interesting is the
application of Bell's inequality to a vehicle at 37.235163 north latitude,
115.810645 west longitude on the 91st day of the year (92nd in leap years).
For other interactions provided by the model, users are advised to RTFC.

\section*{Implementation Techniques}
A variety of innovative programming techniques were employed to model the
more esoteric aspects of psychoceramics. The model makes extensive use of
obfuscatory techniques, several of which have been featured in the IOCCC.
Multiple virtual protected inheritance, quadruple dispatch, nondeterministic
behavior, and operator re-re-overloading are heavily employed. The model
explores the deepest and darkest corners of metaprogramming, and implements
features that hitherto have been lost since the days of INTERCAL and APL.
\end{codeblock}

Assuming the above is the contents of the |summary.tex| file in the |docs/tex|
directory of the psychoceramics model, the toplevel |psychoceramics.tex| file could include this executive summary via
\begin{codeblock}
\frontmatter[style= summary,tables]{summary}
\end{codeblock}


\subsection{Main Matter}
The \command{mainmatter} macro constructs the body matter. The macro
takes a single argument, a comma-separated list of files to be |\input|.
One approach to organizing the document content is to have one master file
that encapsulates all of the body content. For example, with the body
matter encapsulated in the file |ModelBody.tex|, one would use the following
to build the body of the document:
\begin{codeblock}
\mainmatter{ModelBody}
\end{codeblock}
Another approach is to place each of the five chapters in its own |.tex|
file. For example, if these five files are |intro.tex|, |reqt.tex|, |spec.tex|,
|guide.tex|, and |ivv.tex|, one would use the following
to build the body of the document:
\begin{codeblock}
\mainmatter{intro,reqt,spec,guide,ivv}
\end{codeblock}

\subsubsection{Introduction}
\label{sec:chapterone}
The first chapter of the document introduces the subject of the document
in a structured way. This chapter comprises four sections:
\begin{description}
\item[1.1 (Purpose and Objectives):] Brief description of the subject matter.
\item[1.2 (Context within JEOD):] Context of the subject matter within JEOD.
\item[1.3 (Document History):] History of the document.
\item[1.4 (Document Organization):] Table of contents in text form.
\end{description}
The bulk of this chapter is standard across almost all model documents.
Only sections 1.1 (purpose and objectives) and 1.3 (document history) are
document-specific.

The macro |\boilerplatechapterone| constructs this first chapter.
The macro takes two arguments, the contents of section 1.1 (purpose and objectives) and the table entries for section 1.3 (history). Example:

\begin{codeblock}
\boilerplatechapterone{
  The cracked pot interaction, first investigated by Josiah S. Carberry, is
  typically a small perturbative force in spacecraft dynamics. Refinements of
  the interaction have been proposed over the years by a number of authors.
  Future developments in this field may well revolutionize space flight.
  This is the golden age of psychoceramics! The Psychoceramics Model
  implements models of several of the more promising of these techniques.}
{
  \ModelHistory
}
\end{codeblock}

In the above example, the first argument is the first paragraph from the
executive summary. The second paragraph is the |\ModelHistory| command,
presumably defined in the model-specific .sty file.
See section~\ref{sec:model.sty}.


\subsubsection{Product Requirements}
\label{sec:chaptertwo}
Unlike the other four chapters, the requirements chapter has no sections.
It comprises one requirement after another.
Example:
\begin{codeblock}
\requirement{Flux Capacitor}
\label{reqt:1.21gigawatts}
\begin{description:}
\item[Requirement]
 The model shall simulate the behavior of a vehicle outfitted with a
 flux capacitor connected to a 1.21 gigawatt power source
 as the vehicle crosses the 88 MPH threshold.
\item[Rationale]
This capability is needed by the BTTF simulation group.
\item[Verification]
Inspection, test
\end{description:}
\end{codeblock}

\subsubsection{Product Specification}
\label{sec:chapterthree}
The product specification chapter of the document introduces the subject of the document
in a structured way. This chapter comprises four sections:
\begin{description}
\item[3.1 (Conceptual Design):] This section provides
a conceptual description of the model.
This section typically provides answers to the following:
\begin{itemize}
\item What are the key concepts of the model?
\item What is the model architecture?
\item What drove the design of the model?
\end{itemize}

\item[3.2 (Mathematical Formulations):] This section summarizes
the mathematics employed by the model.

In the case of a mathematics-free computer science model,
this section should be named "Key Algorithms."
\item[3.3 (Interactions):] Describes how the model interacts with
other JEOD models and other external agents.
\item[3.4 (Detailed Design):] Details on the design.
\item[3.5 (Inventory):] Details on the design.
\end{description}

FIXME.

\subsubsection{User Guide}
\label{sec:chapterfour}
\begin{description}
\item[4.1 (Instructions for Simulation Users):] Broad description of classes, key design concepts, etc.
\item[4.2 (Instructions for Simulation Developers):] Broad description of classes, key design concepts, etc.
\item[4.3 (Instructions for Model Developers):] Broad description of classes, key design concepts, etc.
\end{description}


\subsubsection{Inspections, Tests, and Metrics}
\label{sec:chapterfive}
\begin{description}
\item[5.1 (Inspections):] Broad description of classes, key design concepts, etc.
\item[5.2 (Tests):] Broad description of classes, key design concepts, etc.
\item[5.3 (Requirements Traceability):] Broad description of classes, key design concepts, etc.
\item[5.4 (Metrics):] Broad description of classes, key design concepts, etc.
\end{description}

\subsubsection{Multi-part Documents}
Word.

\subsection{Back Matter}
The \command{backmatter} macro constructs the back matter.
The macro takes one optional and one mandatory argument.
\begin{quote}
|\backmatter[options]{appendices}|
\end{quote}
The options are a comma-separated list of elements of the form
|key=value| or just |key|. The keys are:
\begin{description}
\item[\option{bibpos=<pos>}] Specifies whether the bibliography
is placed before (|pos=start|) or after (|pos=end|) the appendices.
\item[\option{modelbibfile=<name>}] Specifies the name of the model-specific
|.bib| file. This option is needed only in the case that the model author
has not followed the recommended practice of naming the model-specific
|.bib| file after the model.
\item[\option{nomodelbibfile}] Specifies that there is no model-specific
|.bib| file.
\end{description}

The mandatory argument specifies the names of the files (sans the |.tex| suffix)
the build the appendices.
Use an empty argument '{}' if the model has no appendices.

\subsection{Document Directory Files}
\subsubsection{Makefile}
\subsubsection{Model-Specific Package}\label{sec:model.sty}
when the \dynenv package is operating in |complete| mode,
it loads the model-specific
package file via |\RequirePackage{\MODELNAME}|

One such place where the \dynenv package uses one of names defined in
|paths.def| is the very next operation the \dynenv package performs
when the package is operating in |complete| mode: It loads the model-specific
package file via |\RequirePackage{\MODELNAME}|. For a model document that
is built following the recommended practice, the model-specific
package file must define the following:
\begin{description}
\item[\command{ModelRevision}] The revision number of the model document.\\
  This should be incremented in some manner when the contents of the model
  document are updated. It is recommended that this definition be placed very
  close to the start of the model-specific package to ease the update process.
\item[\command{ModelAuthor}] The name or names of the people who wrote the
  current revision of the model.\\
  The |\ModelAuthor| is used to generate the inside cover and
  is stored as the PDF author of the document.
  Multiple authors, if present, must be separated by newlines (|\\|).
\item[\command{ModelPrefix}] A one word name of the model.\\
  The |\ModelPrefix| is used to label the requirements, inspections, and tests
  that are specified in the model's |.tex| files.
\item[\command{ModelDesc}] A short description of the model.\\
  The |\ModelDesc| is stored as the PDF description of the document.
\item[\command{ModelKeywords}] A comma-separated list of keywords.\\
  The |\ModelKeywords| are stored as the PDF keywords of the document.
\end{description}

It is recommended that the document history also be stored in the model-specific
package file as the command \command{ModelHistory}. This practice places the
document revision number and history in close proximity, thereby easing update.
The command should define a list of table entries, newest to oldest.
Each entry must be of the form |Author & Date & Revision & Description \\|.
The fields are:
\begin{description}
\item[Author]       The authors responsible for the current document revision.
\item[Date]         The month and year of the document revision.
\item[Revision]     The document revision number in major.minor format.
\item[Description]  A brief description of why the revision was made.
\end{description}

A sample model-specific style file follows for the fictitious
psychoceramics model.


\begin{codeblock}
%%%%%%%%%%%%%%%%%%%%%%%%%%%%%%%%%%%%%%%%%%%%%%%%%%%%%%%%%%%%%%%%%%%%%%%%%%%%%%%%
% Model-specific style sheet for the Psychoceramics Model.
%%%%%%%%%%%%%%%%%%%%%%%%%%%%%%%%%%%%%%%%%%%%%%%%%%%%%%%%%%%%%%%%%%%%%%%%%%%%%%%%

\NeedsTeXFormat{LaTeX2e}[2009/01/01]% LaTeX date must be January 2009 or later

% Package definition.
\ProvidesPackage{psychoceramics}[2011/03/01 v1.0 psychoceramics model macros]

% Author and revision number of current revision of the document.
\newcommand\ModelAuthor{Josiah S. Carberry, III}
\newcommand\ModelRevision{1.0}

% Document history, in the form of a list of table entries.
% Entries should be in reverse time order (newest first).
% Each entry must be of the form
%   Author & Date & Revision & Description \\
\newcommand\ModelHistory {
  Josiah S. Carberry, III & April 2011 & 1.0 & Initial version \\
}

% Model description macros, used for the cover page and to tag requirements.
\newcommand\ModelPrefix{Psychoceramics}
\newcommand\ModelDesc{Psychoceramics Model\xspace}
\newcommand\ModelKeywords{BEM, FTL, BTTF}
\end{codeblock}

\subsubsection{Top-level .tex File}
Exhibit~\ref{exhibit:psychoceramics.tex} displayes the complete contents of
the top-level .tex file file the fictitious psychoceramics model.

\begin{exhibit}
\centering
\caption{Contents of \texttt{psychoceramics.tex}}
\label{exhibit:psychoceramics.tex}
\begin{codebox}
%%%%%%%%%%%%%%%%%%%%%%%%%%%%%%%%%%%%%%%%%%%%%%%%%%%%%%%%%%%%%%%%%%%%%%%%%%%%%%%%
% psychoceramics.tex
% Top level document for the Psychoceramics Model
%
%%%%%%%%%%%%%%%%%%%%%%%%%%%%%%%%%%%%%%%%%%%%%%%%%%%%%%%%%%%%%%%%%%%%%%%%%%%%%%%%

% Use the standard JEOD dynenv document class
\documentclass{dynenv}

% Add the requirements et al. to the table of contents
\dynenvaddtocitems{reqt=1,insp=2,test=2}

\begin{document}

% Front matter: Executive summary, table of contents, lists of figures & tables
\frontmatter[style=summary,figures,tables]{summary}

% Main document body: Each chapter is in its own file
\mainmatter{intro,reqt,spec,guide,ivv}

% Back matter: Bibliography
\backmatter{}

\end{document}
\end{codebox}
\end{exhibit}

\subsubsection{Lower Level .tex File}

\section{Requirements, Inspections, and Tests}
\label{sec:reqt_insp_test}
One of the key strengths of the JEOD project is the rigor of its
verification and validation process.
This section describes the macros that authors must use in their
model document |.tex| files to specify the requirements of the model,
the inspections of the model against the requirements, and the
software tests that demonstrate that the model satisfies the requirements
levied against it.

\subsection{Requirements}\label{sec:reqt}
The |\requirement| macro is used in the Requirements chapter of a model document
to introduce a new requirement. The macro takes one argument, the name of the
requirement. After introducing a requirement, give it a label and then describe
the requirement in a description block with a very specific form.
For example,
\begin{codeblock}
\requirement{Handling of Incoming Events}
\label{reqt:unique_label}
\begin{description}
\item[Requirement:]\ \\
  The model shall do this, that, and the other upon receipt of an event.
\item[Rationale:]\ \\
  This is cool, that is needed, and the other is just awesome.
\item[Verification:]\ \\
  Inspection, test
\end{description}
\end{codeblock}
The above generates the following:
\def\ModelPrefix{Example}
\requirement{Handling of Incoming Events}
\label{reqt:unique_label}
\begin{description}
\item[Requirement:]\ \\
  The model shall do this, that, and the other upon receipt of an event.
\item[Rationale:]\ \\
  This is cool, that is needed, and the other is just awesome.
\item[Verification:]\ \\
  Inspection, test
\end{description}

The requirement can be cross-referenced using the |\label| given to the
requirement. For example, |requirement~\ref{reqt:unique_label}| yields
requirement~\ref{reqt:unique_label}. The requirements can be inserted in
the table of contents if desired (see section~\ref{sec:add_to_toc}).

One problem with the way in which requirement~\ref{reqt:unique_label} was
written is that the use of the description environment forces the document 
author to write description item labels in a peculiar way.
The author needs to ensure item labels end with a colon and
needs to add that |\ \\| construct after the label to force a newline.
The \dynenv package provides a colon form of the description environment,
|description:|, to automatically perform both of these tasks:
\begin{codeblock}
\requirement{Handling of Incoming Events}
\label{reqt:another_label}
\begin{description:}
\item[Requirement]
  The model shall do this, that, and the other upon receipt of an event.
\item[Rationale]
  This is cool, that is needed, and the other is just awesome.
\item[Verification]
  Inspection, test
\end{description:}
\end{codeblock}
Except for the auto-incremented requirement number, the output generated by
the above is identical to that of requirement~\ref{reqt:unique_label}:
\def\ModelPrefix{Example}
\requirement{Handling of Incoming Events}
\label{reqt:another_label}
\begin{description:}
\item[Requirement]
  The model shall do this, that, and the other upon receipt of an event.
\item[Rationale]
  This is cool, that is needed, and the other is just awesome.
\item[Verification]
  Inspection, test
\end{description:}

There is a bigger problem with this requirement. It is a good example
of a bad requirement.
The problem is that the requirement specifies three distinct things.
One alternative is to split this poorly written requirement into three separate
requirements. Another is to use the \command{subrequirement} macro:
\begin{codeblock}
\requirement[Event Handling]{Handling of Incoming Events}
\label{reqt:yet_another_label}
\begin{description:}
\item[Requirement]
 Upon receipt of an event,
 \subrequirement{This.}\label{reqt:yet_another_label_this}
 The model shall do this.
 \subrequirement{That.}\label{reqt:yet_another_label_that}
 The model shall do that.
 \subrequirement{Other.}\label{reqt:yet_another_label_other}
 The model shall do the other.
\item[Rationale]
  This is cool, that is needed, and the other is just awesome.
\item[Verification]
  Inspection, test
\end{description:}
\end{codeblock}

\def\ModelPrefix{Example}
\requirement[Event Handling]{Handling of Incoming Events}
\label{reqt:yet_another_label}
\begin{description:}
\item[Requirement]
 Upon receipt of an event,
 \subrequirement{This.}\label{reqt:yet_another_label_this}
 The model shall do this.
 \subrequirement{That.}\label{reqt:yet_another_label_that}
 The model shall do that.
 \subrequirement{Other.}\label{reqt:yet_another_label_other}
 The model shall do the other.
\item[Rationale]
  This is cool, that is needed, and the other is just awesome.
\item[Verification]
  Inspection, test
\end{description:}


\subsection{Inspections}\label{sec:inspections}
The |\inspection| macro is used in the Inspections section of the Inspections,
Tests, and Metrics chapter of a model document to introduce a new inspection.
An ``inspection'' of a product somehow verifies without using the product that
the product satisfies some of the requirements levied on it.
Examples include peer reviews,
a desk check comparison of the mathematics as implemented in the code against
the mathematical description in the model document, and a desk check comparison
of the model interfaces versus some external specification.

The |\inspection| macro takes one argument, the name of the inspection.
After introducing an inspection, give it a label via the |\label| macro and then
describe the inspection in some way. This description can vary in form.
The JEOD project has not prescribed a rigid form for these inspections.
The inspection must indicate the application requirements and must indicate
whether the model passed or failed the inspection.

\begin{codeblock}
\inspection{Event Processing Inspection}
\label{inspect:events}
The model processes events via the method \verb#EventHandler::process_event#.
This function in turn calls three methods to process the incoming event:
\begin{itemize}
\item \verb#EventHandler::do_this#, which does this.
\item \verb#EventHandler::do_that#, which does that.
\item \verb#EventHandler::do_the_other_thing#, which does the other thing.
\end{itemize}
By inspection, the model satisfies
requirement~\traceref{reqt:yet_another_label}.
\end{codeblock}
The above generates the following:
\inspection{Event Processing Inspection}
\label{inspect:events}
The model processes events via the method \verb#EventHandler::process_event#.
This function in turn calls three methods to process the incoming event:
\begin{itemize}
\item \verb#EventHandler::do_this#, which does this.
\item \verb#EventHandler::do_that#, which does that.
\item \verb#EventHandler::do_the_other_thing#, which does the other thing.
\end{itemize}
By inspection, the model satisfies
requirement~\traceref{reqt:yet_another_label}.

\subsection{Tests}\label{sec:tests}
The |\test| macro is used in the Tests section of the Inspections,
Tests, and Metrics chapter of a model document to introduce a new inspection.
A ``test'' of a product uses the product in some way to demonstrate that the
product satisfies some of the requirements levied on it.
Some tests involve simulations while others involve unit tests.
The test should exercise the model in a manner
that allows the output to be compared against some expected result.

The |\test| macro takes one argument, the name of the test.
inspection. After introducing a test, give it a label and then describe
the test in some way. As with inspections, this description can vary.
A widely used approach is to use a description environment. Typical labels are:
\begin{description}
\item[Background] Provides background information regarding the test.
\item[Test description] Provides details on what was done.
\item[Test directory] Specifies where the test code resides.
\item[Success criteria] Specifies how the test is deemed to be successful.
\item[Test results] Describes the results of running the test.
\item[Applicable requirements] Identifies the requirement(s) tested.
\end{description}

\subsection{Specifying Applicable Requirements}
Words


\subsection{Traceability}
Each inspection and test should trace to one or more requirements, and each
requirement should be verified/validated by one or more inspection or test.
Every model document is supposed to contain a requirements
traceability table\footnote{
  The requirements traceability table is typically located in the
  ``Requirements Traceability'' section of the ``Inspections, Tests, and
  Metrics'' chapter of a model document.} %
that lists the mapping from requirements to inspections and tests.
 

\subsection[Adding Requirements etc. to the TOC]
{Adding Requirements, Inspections, and Tests to the Table of Contents}
\label{sec:add_to_toc}
The \command{dynenvaddtotoc} and \command{dynenvaddtocitems} macros
provide the ability to make the requirements, inspections, and tests
specified in document listed in that document's table of contents.

Words

\section{Commands and Environments}
\subsection{Commands}
\begin{description}
\item[\command{addmodel}\cmdarg{NAME}\cmdarg{dir}\cmdarg{name}\cmdarg{title}]
  \hfill [dynenvupfront.sty] \\
  Word
\item[\command{backmatter}\cmdopt{options}\cmdarg{list,of,files}]
  \hfill [dynenvmatter.sty] \\
\item[\command{boilerplatechapterone\cmdarg{description}\cmdarg{history}}]
  \hfill [dynenvboilerplate.sty] \\
  Argument \#1 - Contents of section 1.1 \\
  Argument \#2 - Table entries for history table. \\
  See section~\ref{sec:chapterone} for details.
\item[\command{boilerplateinventory}] \hfill [dynenvboilerplate.sty] \\
  Word
\item[\command{boilerplatemetrics}] \hfill [dynenvboilerplate.sty] \\
  Word
\item[\command{boilerplatetraceability}] \hfill [dynenvboilerplate.sty] \\
  Word
\item[\command{escapeus}\cmdarg{text\_with\_underscores}]
  \hfill [dynenvupfront.sty] \\
  Word
\item[\command{iflabeldefined}\cmdarg{label}\cmdarg{commands}]
  \hfill [dynenvboilerplate.sty] \\
  Argument \#1 - Label to be tested.\\
  Argument \#2 - Code to be expanded. \\
  Expands second argument if label specified by first argument is defined.
\item[\command{JEODHOME}] \hfill [paths.def] \\
  Path to \$JEODHOME.
\item[\command{frontmatter}\cmdopt{options}\cmdarg{abstractOrSummary}]
  \hfill [dynenvmatter.sty] \\
\item[\command{inspection}\cmdarg{name}] \hfill [dynenvreqt.sty] \\
  Words.
\item[\command{longentry}] \hfill [dynenvboilerplate.sty] \\
  Causes broken lines in a |longtable| environment to be printed with a
  hanging indent.
\item[\command{mainmatter}\cmdopt{options}\cmdarg{list,of,files}]
  \hfill [dynenvmatter.sty] \\
\item[\command{MODELDIR}] \hfill [paths.def] \\
  Model directory, with underscores escaped.
\item[\command{MODELDIRx}] \hfill [dynenv.sty] \\
  |{\MODELDIR\xspace}|
\item[\command{MODELDOCS}] \hfill [paths.def] \\
  Relative path to model documentation directory.
\item[\command{MODELGROUP}] \hfill [paths.def] \\
  Model group (same as type except for utils).
\item[\command{MODELGROUPx}] \hfill [dynenv.sty] \\
  |{\MODELGROUPx\xspace}|
\item[\command{ModelHistory}] \hfill [\inanglebrackets{model\_name}.sty] \\
  History table entries.
\item[\command{MODELHOME}] \hfill [paths.def] \\
  Relative path to model directory.
\item[\command{MODELNAME}] \hfill [paths.def] \\
  Model directory, underscores not escaped.
\item[\command{MODELPATH}] \hfill [paths.def] \\
  Path to model directory from \$JEODHOME.
\item[\command{MODELPATHx}] \hfill [dynenv.sty] \\
  |{\MODELPATH\xspace}|
\item[\command{ModelPrefix}] \hfill [\inanglebrackets{model\_name}.sty] \\
  One-word prefix for requirements, etc.
\item[\command{MODELTITLE}] \hfill [paths.def] \\
  All-caps model name command.
\item[\command{MODELTITLEx}] \hfill [dynenv.sty] \\
  |{\MODELTITLE\xspace}|
\item[\command{MODELTYPE}] \hfill [paths.def] \\
  Model type (e.g., dynamics, environment).
\item[\command{MODELTYPEx}] \hfill [dynenv.sty] \\
  |{\MODELTYPE\xspace}|
\item[\command{simpletracetable}\cmdarg{one}\cmdarg{two}\cmdarg{three}]
  \hfill [dynenvreqt.sty] \\
  Words.
\item[\command{requirement}\cmdarg{name}] \hfill [dynenvreqt.sty] \\
  Words.
\item[\command{subrequirement}\cmdarg{name}] \hfill [dynenvreqt.sty] \\
  Words.
\item[\command{test}\cmdarg{name}] \hfill [dynenvreqt.sty] \\
  Words.
\item[\command{traceref}\cmdarg{label}]
  \hfill [dynenvreqt.sty] \\
  Words.
\item[\command{tracerefrange}\cmdarg{label1}\cmdarg{label2}]
  \hfill [dynenvreqt.sty] \\
  Words.
\item[\command{tracetable}\cmdarg{one}\cmdarg{two}\cmdarg{three}]
  \hfill [dynenvreqt.sty] \\
  Words.
\end{description}

\subsection{Environments}
\begin{description}
\item[\option{abstract}] \hfill [dynenvmatter.sty] \\
  The abstract environment defined in |report.cls| issues an almost
  harmless |\titlepage| command,  the harm being that it resets the page
  number to one (or rather, roman i). Redefining this environment avoids
  this problem.
\item[\option{codeblock}] \hfill [dynenvcode.sty] \\
  The codeblock environment is a specialization of the Verbatim environment.
  Use this environment for code to be printed inline with the text. The
  printed code will be indented with respect to the current indentation level.
\item[\option{codebox}] \hfill [dynenvcode.sty] \\
  The codebox environment is a specialization of the Verbatim environment.
  Use this environment for code to be printed in an exhibit. The code will
  be framed in a box and is indented slightly with respect to page boundaries.
\item[\option{description:}] \hfill [dynenvreqt.sty] \\
  The description: environment is a specialization of the description
  environment intended for use with requirements.
\item[\option{exhibit}] \hfill [dynenvcode.sty] \\
  The exhibit environment is similar to the table environment.
  Exhibits, like tables, can have captions and labels. The caption should
  be placed above the exhibited items. The exhibit will be listed as a part
  of the list of tables.
\item[\option{stretchlongtable}] \hfill [dynenvboilerplate.sty] \\
  A | stretchlongtable | is a |longtable| environment that takes an optional
  |arraystretch| (default: 1.2). If specified, the |arraystretch| value must be
  specified in angle brackets and must precede the optional arguments to the
  |longtable| environment.
\end{description}


\end{document}
