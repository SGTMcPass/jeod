\section{Packages and Files Loaded by the \dynenv Package}
\label{sec:packages}
The \dynenv package loads a number of standard \LaTeX\ packages plus additional
packages and files as indicated by the options to the \dynenv class or package.
These packages and files are described below, in the order in which they
are loaded by the \dynenv package. The descriptions below are brief.
Users who want to obtain more information on one of the standard packages
are encouraged to read the documentation provided by the package developer
by typing |texdoc <package_name>| on the command line.

\subsection{Package \package{jeodspec}}
This package identifies the current JEOD release,
defines some simple shortcut macros,
and specifies the models that comprise the current release.
See section~\ref{sec:jeodspec} for details.

\subsection{Package \package{geometry}}
Specifying page layout in \LaTeX\ is a bit tricky and error-prone.
The \package{geometry} package solves these issues. The \dynenv package
uses the \package{geometry} package to specify a standard page layout
for JEOD documents. 

\subsection{Package \package{xspace}}
Consider the shortcut macro |\newcommand{fbz}{foo bar baz}|. Users of this
macro have to take care to add a hard space after invoking the macro in the
middle of a sentence but never to do so before punctuation. The \package{xspace}
package solves this problem. Define the macro as |\newcommand{fbz}{foo bar baz\xspace}| and voila, there is no need for that hard space.

\subsection{Package \package{xkeyval}}
The \package{xkeyval} package provides a convenient mechanism for specifying
and processing options to a class, package, or macro. The \dynenv package
uses the capabilities provided by \package{xkeyvals} to specify and process
the options to the document production macros
(see section~\ref{sec:doc_production}), plus a few other macros.

\subsection{Package \package{longtable}}
The \package{longtable} package provides the ability to create tables that
span multiple pages. Several of the automatically generated tables
in JEOD model documents are long tables.

\subsection{Package \package{appendix}}
The \package{appendix} package alters some of the base \LaTeX\ mechanisms
for presenting appendices and also provides the ability to have end of chapter
appendices.

The \package{appendix} package is not a part of the standard or extended sets of
packages loaded by the \dynenv package. The package is loaded only if the
|appendix| option is provided to the \dynenv class or package.
The \package{appendix} package should not be used in documents with parts.

\subsection{AMS Math Packages \package{amsmath}, \package{amssymb}, and \package{amsfonts}}
The \package{amsmath} package improves upon and augments the base
\LaTeX\ mechanisms for representing and displaying mathematical formulas
and is the basis for the American Mathematical Society \LaTeX\ suite.
The \package{amssymb} package defines miscellaneous mathematical symbols
beyond those provided by the \package{amsmath} package, and the
\package{amsfonts} package provides things such as
blackboard bold letters, Fraktur letters, and other miscellaneous symbols.

The AMS math packages are not a part of the standard set of packages
loaded by the \dynenv package but they are a part of the extended set
of packages (those enabled by specifying the \dynenv |complete| option).

\subsection{Package \package{graphicx}}
The \package{graphicx} package provides a range of capabilities. In JEOD,
the graphics bundle is used primarily to include graphics into a document
via the |includegraphics| macro.
The \package{graphicx} package is not a part of the standard set of packages
loaded by the \dynenv package but it is part of the extended set
of packages, enabled by specifying the \dynenv |complete| option.

\subsection{User-Specified Packages}
All options provided to the \dynenv package other than those explicitly
handled by a option handler specify some non-standard package to be
loaded. See section~\ref{sec:options} for a list of options that have
explicit handlers.

One side effect of this handling of \dynenv package options is that
options that do have explicit handlers cannot be used to load a package.
That said, only three packages in the \LaTeX\ distribution are shadowed by
\dynenv package options:
\begin{description}
\item[|appendix|] The \dynenv package loads the \package{appendix} package
  when the |appendix| option is specified, so this shadowing doesn't count.
\item[|hyperref|] The \dynenv package explicitly makes |hyperref| an illegal
  option because the \package{hyperref} package by design needs to be the
  very last package loaded in a document. That the \package{hyperref} package
  is shadowed by the |hyperref| option is intentional.
\item[|section|] The \package{section} package changes the appearance of
  sectioning commands. JEOD documents should not use this package.
\end{description}

The use of the extended options capability is optional.
The mechanism is a bit opaque. Some authors would rather explicitly
invoke the |\RequirePackage| macro in the model-specific package file.
Another issue is that the packages loaded via this mechanism are loaded without
options. If options need to be specified those packages must be loaded
in the model-specific package file.

\subsection{Package \package{dynmath}}
The \package{dynmath} package provides macros that implement the
recommended practice for displaying vectors, matrices, and quaternions.
This package is not a part of the \LaTeX\ distribution but is a part
of the standard set of packages loaded by \dynenv.
Documentation for this package is available in the JEOD file
|docs/templates/jeod/dynmath.pdf|.

\subsection{Package \package{dyncover}}
The \package{dyncover} package creates cover pages for JEOD documents.
This package is not a part of the \LaTeX\ distribution but is a part
of the standard set of packages loaded by \dynenv.
Documentation for this package is, ummm, RTFC.

\subsection{File \package{paths.def}}
The standard model document makefile automatically generates the file
|paths.def|. This file defines a standard set of macros whose values depend
on the model and on the make target. (The relative paths lose one level of indirection when the file is built with |make install| versus |make|.)
The names defined in this file are used in |dynenv.sty| in various places.

As a fictitious example, consider the model located in
|$JEOD_HOME/models/interactions/psychoceramics|.
The contents of the |paths.def| file
for this model given the command |make| will be:
\begin{codeblock}
\ifx\model@pathsdef\endinput\endinput\else\let\model@pathsdef\endinput\fi|\\
\newcommand\JEODHOME{../../../../..}
\newcommand\MODELHOME{../..}
\newcommand\MODELDOCS{..}
\newcommand\MODELPATH{models/interactions/psychoceramics}
\newcommand\MODELTYPE{interactions}
\newcommand\MODELGROUP{interactions}
\newcommand\MODELDIR{psychoceramics}
\newcommand\MODELNAME{psychoceramics}
\newcommand\MODELTITLE{\PSYCHOCERAMICS}
\endinput
\end{codeblock}

The \dynenv package loads this file via |\input{paths.def}| when the
package is operating in |complete| mode. The burden of loading this file
otherwise falls on the model document author. Note that the file can safely
be included multiple times because the first line of |paths.def| is the
\TeX\ equivalent of the one-time include protection used in
\Cplusplus header files.

\subsection{Model-Specific Package}
One such place where the \dynenv package uses one of names defined in
|paths.def| is the very next operation the \dynenv package performs
when the package is operating in |complete| mode: It loads the model-specific
package file via |\RequirePackage{\MODELNAME}|. The model-specific
package file defines a standard set of macro names whose definitions are
model-specific, plus additional macros that are specific to the model.
See section~\ref{sec:model.sty} for details.
\subsection{Package \package{hyperref}}
The \package{hyperref} package makes all cross-references in a PDF document
into hyperlinks and adds the ability to insert hyperlinks to other documents.
The \package{hyperref} package is not a part of the standard set of packages
loaded by the \dynenv package but it is part of the extended set
of packages, enabled by specifying the \dynenv |complete| option.

To make the content linkable, the package needs to modify many macros created
by other packages. The \package{hyperref} is typically the very last package
loaded in a \LaTeX\ document to ensure that it can make those alterations.
The \dynenv package loads \package{hyperref} when the \dynenv package is
used per the recommended practice. This means that a |.tex| file that
follows the recommended practice must by necessity have a very simple
\LaTeX\ preface. There should be no |\usepackage| statements in the main
|.tex| file for a model document that follows the recommended practice.


\section{Package \package{jeodspec}}
\label{sec:jeodspec}
The \package{jeodspec} package defines commands that need to be updated with
every release and with additions to or deletions from the JEOD model suite.
The maintainers of this package will be JEOD administrators and model developers
rather than \TeX\ programmers. The contents of |jeodspec.sty| are designed
with this in mind. Extensive knowledge of \TeX\ is not required.

The first set of definitions in |jeodspec.sty| specify items that need
to be changed with each release:
\begin{description}
\item[\command{JEODid}] The name of the current (or upcoming) JEOD release.
\item[\command{RELEASEMONTH}] The name of the month when the release will occur.
\item[\command{RELEASEYEAR}] The four-digit year when the release will occur.
\end{description}

The next set of definitions specify items that need less frequent changes:
\begin{description}
\item[\command{JEODMANAGERS}] The names and titles of the people who
need to approve the JEOD documentation suite.
\item[\command{JEODORG}] The JSC organization under which JEOD is developed
and maintained.\item[\command{JEODJSCNUM}] The official JSC document number
for the JEOD documentation suite.
\end{description}

The third set of definitions create shortcut commands for text phrases used in
multiple documents. Some comments:
\begin{itemize}
\item Model authors:
Please do not add model-specific shortcuts to |jeodspec.sty|.
These shortcuts should be for widely-used terminology.
The model-specific .sty file is the place to define such shortcuts.
\item The expansion of the older items in this set do not end with
\command{xspace}. Please consider using \command{xspace} in future definitions.
\end{itemize}

The final set of definitions in |jeodspec.sty| identify the models that
comprise the current JEOD release. Each JEOD model is specified via the
\command{addmodel} macro. \command{addmodel} takes four arguments, as follows:
\begin{itemize}
\item The model name,
by convention an all-caps, letters-only identifier.
For example, this argument is |ATMOSPHERE| for the atmosphere model.
\item The model parent directory name, one of dynamics, environment,
interactions, or utils. The atmosphere model is located in
|$JEOD_HOME/models/environment/atmosphere|, so in this case the parent
directory name is |environment|.
\item The model directory name. For the atmosphere model this is |atmosphere|.
Do not escape underscores in the model directory name. The model directory
name for the Earth Lighting Model is |earth_lighting|, not
|earth\_lighting|.
\item The model title. This is often an initial caps version of the
model name, plus "Model". For example, the title of the atmosphere model
is |Atmosphere Model|.
\end{itemize}
