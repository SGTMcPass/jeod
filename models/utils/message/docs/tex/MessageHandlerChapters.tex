\setcounter{chapter}{0}

%----------------------------------
\chapter{Introduction}\hyperdef{part}{intro}{}\label{ch:intro}
%----------------------------------


\section{Model Description}
%%% Incorporate the intro paragraph that used to begin this Chapter here.
%%% This is location of the true introduction where you explain why this 
%%% model exists.
%%% Identify the Model context within JEOD.
The JEOD \MessageHandlerDesc\ provides a convenient, 
flexible, consistent facility for users and developers alike to
communicate information and program status back to the user through a TTY 
interface.  The \MessageHandlerDesc\ offers
parameters which control the format of messages and the level of screening desired, thereby 
allowing flexibility for users and developers to specify which messages will
be sent and how they should be formatted under various conditions.

Most of the JEOD models and simulations make liberal use of the \MessageHandlerDesc\ 
in order to convey information regarding the state of the simulation, 
errors, warnings, debug information 
and failure conditions.  The control capabilities allow output to 
range from maximum debugging verbosity to a clean run of a production sim.

The parent document to this model document is the
JEOD Top Level Document~\cite{dynenv:JEOD}.



\section{Document History}
%%% Status of this and only this document.  Any date should be relevant to when 
%%% this document was last updated and mention the reason (release, bug fix, etc.)
%%% Mention previous history aka JEOD 1.4-5 heritage in this section.
%%% Mention that JEOD.pdf is the parent document.

\begin{tabular}{||l|l|l|l|} \hline
\DocumentChangeHistory
\end{tabular}

\section{Document Organization}
This document is formatted in accordance with the 
NASA Software Engineering Requirements Standard~\cite{NASA:SWE} 
and is organized into the following chapters:

\begin{description}
%% longer chapter descriptions, more information.

\item[Chapter 1: Introduction] - 
This introduction contains three sections: description of model, document history, and organization.  
The first section provides the introduction to the \MessageHandlerDesc\ and its reason 
for existence.  It contains a brief description of the interconnections with other models, and 
references to any supporting documents. It lists the document that is parent to this one.
The second section displays the history of this document which includes
author, date, and reason for each revision.  The final
section contains a description of how the document is organized.

\item[Chapter 2: Product Requirements] - 
Describes requirements for the \MessageHandlerDesc.

\item[Chapter 3: Product Specification] - 
Describes the underlying theory, architecture, and design of the \MessageHandlerDesc\ in detail.  It is organized in 
three sections: Conceptual Design, Mathematical Formulations, and Detailed Design.

\item[Chapter 4: User Guide] - 
Describes how to use the \MessageHandlerDesc\ in a Trick simulation.  It is broken into three sections to represent the JEOD 
defined user types: Analysts or users of simulations (Analysis), Integrators or developers of simulations (Integration),
and Model Extenders (Extension).

\item[Chapter 5: Verification and Validation] -  
Contains \MessageHandlerDesc\ verification and validation procedures and results.

\end{description}

%----------------------------------
\chapter{Product Requirements}\hyperdef{part}{reqt}{}\label{ch:reqt}
%----------------------------------
This model shall meet the JEOD project requirements specified in the 
\hyperref{file:\JEODHOME/docs/JEOD.pdf}{part1}{reqt}{JEOD} top-level document~\cite{dynenv:JEOD}.

%%% Format for the model Requirements is open.  It should include requirements for this model 
%%% only and use requirment tags like the one below.
%\requirement{...}
%\label{reqt:...}
%\begin{description}
%  \item[...]\ \newline
%    The documentation for the model shall include
%
%    \subrequirement{}
%    \label{reqt:...}
%      Software requirements specification.
%      
%    ...
%   
%  \item[title]\ \newline
%    text
%
%  ...
%
%\end{description}

%----------------------------------
%----------------------------------
\requirement{Failure Messages}
\label{reqt:FAIL}
\begin{description}
  \item[Requirement:]\ \newline
     The \MessageHandlerDesc\ shall be capable of accepting a failure message; conveying it to the user in a console window; and terminating the simulation.  Failure messages are never suppressed.
  \item[Rationale:]\ \newline
A key function of the \MessageHandlerDesc\ is to convey information to the user
regarding fatal errors followed by immediate termination of the simulation.

  \item[Verification:]\ \newline
    Test
\end{description}

\requirement{Error Messages}
\label{reqt:ERROR}
\begin{description}
  \item[Requirement:]\ \newline
     The \MessageHandlerDesc\ shall be capable of accepting an error message and
conveying it to the user in a console window.  Error messages can be 
conditionally suppressed by setting the suppression level parameter.
  \item[Rationale:]\ \newline
A key function of the \MessageHandlerDesc\ is to convey error messages to the user conditionally depending
on a user-settable suppression level parameter.

  \item[Verification:]\ \newline
    Test
\end{description}

\requirement{Warning Messages}
\label{reqt:WARN}
\begin{description}
  \item[Requirement:]\ \newline
     The \MessageHandlerDesc\ shall be capable of accepting a warning message and
conveying it to the user in a console window.  Warning messages can be 
conditionally suppressed by setting the suppression level parameter.
  \item[Rationale:]\ \newline
A key function of the \MessageHandlerDesc\ is to convey warning messages to the user conditionally depending
on a user-settable suppression level parameter.

  \item[Verification:]\ \newline
    Test
\end{description}

\requirement{Informational Messages}
\label{reqt:INFORM}
\begin{description}
  \item[Requirement:]\ \newline
     The \MessageHandlerDesc\ shall be capable of accepting an informational 
message and conveying it to the user in a console window.  Informational 
messages can be conditionally suppressed by setting the suppression level 
parameter.
  \item[Rationale:]\ \newline
A key function of the \MessageHandlerDesc\ is to convey informational messages 
to the user conditionally depending
on a user-settable suppression level parameter.

  \item[Verification:]\ \newline
    Test
\end{description}

\requirement{Debugging Messages}
\label{reqt:DEBUG}
\begin{description}
  \item[Requirement:]\ \newline
     The \MessageHandlerDesc\ shall be capable of accepting a debugging 
message and conveying it to the user in a console window.  Debugging 
messages can be conditionally suppressed by setting the suppression level 
parameter.
  \item[Rationale:]\ \newline
A key function of the \MessageHandlerDesc\ is to convey debugging messages 
to the user conditionally depending
on a user-settable suppression level parameter.

  \item[Verification:]\ \newline
    Test
\end{description}

\requirement{Generic Messages}
\label{reqt:GENERIC}
\begin{description}
  \item[Requirement:]\ \newline
     The \MessageHandlerDesc\ shall be capable of accepting a generic 
message and conveying it to the user in a console window.  Generic 
messages can be conditionally suppressed by setting the suppression level 
parameter.
  \item[Rationale:]\ \newline
A key function of the \MessageHandlerDesc\ is to convey generic messages 
to the user conditionally depending
on a user-settable suppression level parameter.

  \item[Verification:]\ \newline
    Test
\end{description}


\chapter{Product Specification}\hyperdef{part}{spec}{}\label{ch:spec}
%----------------------------------
\section{Conceptual Design}
The main purpose of the \MessageHandlerDesc\ is to provide a common interface
for display or logging of debugging, informational, warning, error or failure messages.  
The \MessageHandlerDesc\ allows format and verbosity of messages to be
tailored to the needs of the user by means of input variables.  
MessageHandler is a base class with one pure virtual method 
(MessageHandler::process\_message) which must be implemented in a derived
class in order to create a functional \MessageHandlerDesc.  The 
TrickMessageHandler is such a derived class and
provides an implementation of process\_message using the Trick methods
exec\_terminate for fatal errors and send\_hs for all other messages.
If other avenues of output were needed, then a different subclass of
MessageHandler could be substituted for TrickMessageHandler in order
to provide the desired functionality.



\section{Mathematical Formulations}
The \MessageHandlerDesc\ includes no mathematical algorithms.
\section{Detailed Design}
The \MessageHandlerDesc\ includes a protected field,
MessageHandler::suppression\_level, which 
the user can set in the input file in order to control which messages will be suppressed.  There are also public static integer constants 
which indicate levels of severity.
\begin{table}
\begin{tabular}{lll}
Constant & Value & Meaning \\
MessageHandler::Fail & -1 & Prints a message and terminates simulation \\
MessageHandler::Error & 0 & Most severe error which does not terminate the simulation \\
MessageHandler::Warning & 9 & Indicates results are suspect \\
MessageHandler::Notice & 99 & Indicates an informational message \\
MessageHandler::Debug & 999 & Intended for messages to inform users of expected events
\end{tabular}
\caption{Symbolic Constants for Severity Levels}
\end{table}
\newpage
The public methods of the \MessageHandlerDesc\ are:
\begin{itemize}
\item Fail -- deliver a failure message and terminate the simulation
\item Error -- deliver an error message of severity MessageHandler::Error
\item Warn -- Deliver a warning message with severity MessageHandler::Warning
\item Inform -- Deliver an informational message with severity MessageHandler::Notice
\item Debug -- Deliver a debugging message with severity MessageHandler::Debug
\item Send\_message -- Deliver a generic message with severity
and labeling specified by the user
\item Va\_send\_message -- Just like send\_message except that an argument
list defined by starg.h replaces the variadic arguments of send\_message
\end{itemize}
All of these methods are declared static, thus, there is effectively
only one MessageHandler in any given simulation.  These static methods
are all wrappers for the virtual method process\_message, which, due
to limitations of C++ cannot be a static method.  
The programming technique used to circumvent this issue is to include
a static field (handler, of type MessageHandler) in the MessageHandler
class.  The default initialization of handler is NULL, and upon first
instantiation of MessageHandler, handler is set to this instance.
Subsequent instantiations of the MessageHandler class have no effect on the value of the 
static field handler.
With the introduction of the 
sim\_interface model, the MessageHandeler is constructed automatically.  
See Chapter ~\ref{ch:user} for details.

For the user, the implications are as follows:
\begin{itemize}
\item There will only be one instance of
the MessageHandler, and that one is created
automatically with the construction
of the JeodSimulationInterface.
\item Changes made to instance fields such as suppression\_level, 
suppress\_id and suppress\_location are only effective if performed on
the initial instance of MessageHandler
\end{itemize}

Additional details of the \MessageHandlerDesc\ design can be found in
\hyperref{file:refman.pdf}{}{}
{JEOD Message Model Reference Manual}~\cite{message_refman}.

.	\section{Inventory}

All \MessageHandlerDesc\ files are located in the directory
\$JEOD\_HOME/models/utils/message.
Tables~\ref{tab:source_files} to~\ref{tab:verification_files}
list the configuration-managed files in this directory.

\input{inventory}


\chapter{User Guide}\hyperdef{part}{user}{}\label{ch:user}
The User Guide is composed of three sections. The Analysis section is 
intended primarily for users of pre-existing simulations.  
The Integration section of the user guide is intended for simulation developers.  
The Extension section of the user guide is intended primarily for developers 
needing to extend the capability of the \MessageHandlerDesc.  Such users 
should have a thorough understanding of how the model is used in the preceding 
Integration section, and of the model 
specification (described in Chapter \ref{ch:spec}).

\section{Analysis}
The primary purpose of the \MessageHandlerDesc\ is to provide analysts
(i.e., sim users) useful information regarding the status of a sim
while it is running. Some messages are related to the 
intentional termination of a sim. Other messages may occur without
sim termination. Sim users can use the information conveyed by the
\MessageHandlerDesc\ to help ensure their sim is running
correctly or to report possible problems in the JEOD models to 
the JEOD development team. When reporting a message be sure
to capture the entire message.  This will aid with the
determination and correction of the problem.

An example of a message from the JEOD Gravity Model is:
\begin{verbatim}
SIMULATION TERMINATED IN
  PROCESS: 1
  JOB/ROUTINE: 2/gravity_source.cc line 551
DIAGNOSTIC:
Error environment/gravity ord_exceeds_deg_error:
Gravity field order (80) is greater than gravity field degree (70).
\end{verbatim}
In this case, the user requested a gravity order which is larger
than the gravity degree. 

The analyst has three inputs to control the types of messages displayed
and the information related to those messages:

{\bf suppression\_level} - Default value: 10 (warnings and non-fatal errors).
All messages have an associated severity level, with increasingly positive
values indicating warnings of decreasing severity. Fatal errors have a
negative severity level. Messages whose severity equals or exceeds the
value of the global message handler's suppression\_level are suppressed.
Note that fatal errors cannot be suppressed.

{\bf suppress\_id} - Default value: false.
This flag indicates whether the message ID is printed for unsuppressed
messages. The ID is not printed if this flag is set to true.
The message ID is always printed for errors.

{\bf suppress\_location} - Default value: false.
This flag indicates whether the message source file and line number
are printed for unsuppressed messages. The location is not printed if this
flag is set to true.
The message location is always printed for errors.

The first Trick sim\_object created should contain an instance of 
JeodSimulationInterface, which
will automatically construct a singleton MessageHandler for the sim.
Here is a typical definition of such a sim\_object.
\begin{verbatim}
/* ===================== JEOD System Simulation Object ===================== **/
/* ========================================================================= **/
sim_object { /* jeod_sys
---------------------------------------------------*/
   /*=========================================================================
    =  This is the JEOD executive model, this model should be basically
    =  the same for all JEOD-based Trick applications.
=
    =========================================================================*/
   /*==================
    = DATA Structures =
    ==================*/
   utils/sim_interface/include: JeodTrickSimInterface sim_interface;
} jeod_sys;
\end{verbatim}


An example of how these three controls might be set in an input file is:
\begin{verbatim}
jeod_sys.sim_interface.message_handler.suppression_level = MessageHandler::Warning;
jeod_sys.sim_interface.message_handler.suppress_id       = true;
jeod_sys.sim_interface.message_handler.suppress_location = false;
\end{verbatim}
In this example case, only errors will be reported; warnings and other
messages with a higher severity level will be suppressed.
For any message that is not suppressed, the ID number will be suppressed and location
of any messages will be displayed.

%%%%%%%%%%%%%%%%%%%%%%%%%%%%%
\section{Integration}
The use of the \MessageHandlerDesc\ by sim integrators is essentially the
same as for sim analysts. The severity level of suppressed messages
can be changed to suit the needs of the sim integrator. See previous section
for more information.

%%%%%%%%%%%%%%%%%%%%%%%%%%%%%
\section{Extension}
Sim developers can add messages to a model. The messages are normally defined
in the model class in the {\bf include} directory.  For example,
messages for the JEOD Gravity Model are defined in {\bf models/environment/gravity/include}
in the file gravity\_messages.hh as shown below:
\begin{verbatim}
   // Errors
   static char const * max_deg_error; /* --
      Error issued when requested gravity field degree exceeds maximum */

   static char const * max_ord_error; /* --
      Error issued when requested gravity field order exceeds maximum */

   static char const * ord_exceeds_deg_error; /* --
      Error issued when requested gravity field order is greater than degree */


   // Warnings
   static char const * radial_distance_warning; /* --
      Warning issued when radial distance is less than equatorial radius of
      gravity model */

\end{verbatim}

The same messages must also be added to a corresponding file in the 
{\bf src} directory, such as:
\begin{verbatim}
// Errors
char const * GravityMessages::max_deg_error =
    PATH " max_deg_error";

char const * GravityMessages::max_ord_error =
    PATH " max_ord_error";

char const * GravityMessages::ord_exceeds_deg_error =
    PATH " ord_exceeds_deg_error";

// Warnings
char const * GravityMessages::radial_distance_warning =
    PATH " radial_distance_warning";
\end{verbatim}

Finally, the code that actually triggers the message must be included
in the proper source code file.  For the Gravity Model example, the
message trigger code appears in the gravity\_body.cc file.  An
example of a {\bf failure} and a {\bf warning} trigger are:
\begin{verbatim}
      // check if maximum order to be used for computations is greater than
      // maximum degree
      if (controls.order > controls.degree) {
         MessageHandler::fail (
         __FILE__, __LINE__, GravityMessages::ord_exceeds_deg_error,
         "Gravity field order (%i) is greater than gravity field degree (%i).\n",
         controls.order,controls.degree);
         return;

      }

      // check if radial distance is less than equatorial radius
      if (r_mag < radius) {
         MessageHandler::warn (
            __FILE__, __LINE__, GravityMessages::radial_distance_warning,
            "Radial distance (%E) is less than equatorial radius of gravity model (%E).\n",
            r_mag,radius);
      }

\end{verbatim}

%----------------------------------


%----------------------------------
\chapter{Verification and Validation}\hyperdef{part}{ivv}{}\label{ch:ivv}
%----------------------------------
\section{Verification}
%%% code imported from old template structure
%\inspection{<Name of Inspection>}\label{inspect:<label>}
% <description> to satisfy  
% requirement \ref{reqt:<label>}.

A verification sim was built to test the various combinations of message
suppression. The sim is located at:  {\bf message/verif/SIM\_message\_handler\_verif }
This simulation is constructed in such a way that almost all the 
meaningful output is routed to standard error.  There is a 
bin directory which contains a script (run\_test) to run the sim 
through its tests.
The output of the verification sim is shown below.  :
\begin{verbatim}
Tests 1-5: Tests with default suppression settings

Test 1: send_message()
Message utils/message/verif/Message at message_handler_verif_driver.cc line 105:
Test of MessageHandler::send_message()


Test 2: debug()

Test 3: inform()

Test 4: warn()
Warning utils/message/verif/Warning at message_handler_verif_driver.cc line 83:
Test of MessageHandler::warn()


Test 5: error()
Error utils/message/verif/Error at message_handler_verif_driver.cc line 76:
Test of MessageHandler::error()



Tests 6-10: Suppression level set to 9

Test 6: send_message() (Not suppressed)
Message utils/message/verif/Message at message_handler_verif_driver.cc line 105:
Test of MessageHandler::send_message()


Test 7: debug() (Suppressed)
Test 8: inform() (Suppressed)

Test 9: warn() (Suppressed)

Test 10: error() (Not suppressed)
Error utils/message/verif/Error at message_handler_verif_driver.cc line 76:
Test of MessageHandler::error()



Tests 11-12: Supression level set to 0

Test 11: send_message() (Suppressed)
Test 10: error() (Suppressed)


Tests 13-15: Tests of suppressing ID, location

Test 13: error() (ID suppressed)
message_handler_verif_driver.cc line 76
Test of MessageHandler::error()


Test 14: error() (location suppressed)
Error utils/message/verif/Error:
Test of MessageHandler::error()


Test 15: error() (ID and location suppressed)
Test of MessageHandler::error()


Test 16: fail() (Never suppressed)
\end{verbatim}


\section{Validation}
(No validation tests were conducted for the \MessageHandlerDesc .)

\section{Metrics}
\subsection{Code Metrics}
Table~\ref{tab:coarse_metrics} presents coarse metrics on the source
files that comprise the model.

\input{coarse_metrics}

Table~\ref{tab:metrix_metrics} presents the extended cyclomatic complexity
(ECC) of the methods defined in the model.
\input{metrix_metrics}

%%% code imported from old template structure
%\test{<Title>}\label{test:<label>}
%\begin{description}
%\item[Purpose:] \ \newline
%<description>
%\item[Requirements:] \ \newline
%By passing this test, the universal time module 
%partially satisfies requirement~\ref{reqt:<label1>} and 
%completely satisfies requirement~\ref{reqt:<label2>}.
%\item[Procedure:]\ \newline
%<procedure>
%\item[Results:]\ \newline
%<results>
%\end{description}

