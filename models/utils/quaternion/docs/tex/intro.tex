\chapter{Introduction}\hyperdef{part}{intro}{}
\label{ch:intro}

\section{Purpose and Objectives of the \ModelDesc}
\label{sec:purp}
Sir William Rowan Hamilton developed a four dimensional extension to the complex
numbers called quaternions in the mid 1800s. Like the reals and the complex
numbers, quaternions can be added, subtracted, multiplied, and divided.
Unlike the reals and complex numbers, quaternion multiplication and division
is not commutative. Care must be taken in specifying the order of the
factors when multiplying quaternions.

Quaternions, like complex numbers, can be represented as comprising real
imaginary parts. In the case of quaternions, there are three distinct roots
of -1, $i$, $j$, and $k$. The imaginary part of a quaternion can be viewed
as a three vector. (In fact, the use of $\hat i$, $\hat j$, and $\hat k$ as
canonical unit vectors in mathematics and physics was motivated by Hamilton's
quaternions.)

One widely used application of the quaternions is to represent rotations and
transformations in Euclidean three dimensional space.
That application is the subject of the \ModelDesc.

\section{Context within JEOD}
The following document is parent to this document:
\begin{itemize}
\item \hyperJEOD
\end{itemize}

The \ModelDesc forms a component of the utilities suite of
models within \JEODid. It is located at
models/utils/quaternion.


%%%%%%%%%%%%%%%%%%%%%%%%%%%%%%%%%%%%%%%%%%%%%%%%%%%%%%%%%%%%%%%%%%%%%%%%%%%%%%%%
% change_history.tex
% History of orientation_manager.pdf
% Add a new line to the table for each update. 
%%%%%%%%%%%%%%%%%%%%%%%%%%%%%%%%%%%%%%%%%%%%%%%%%%%%%%%%%%%%%%%%%%%%%%%%%%%%%%%%

\section{Documentation History}
\begin{tabular}{||l|l|l|l|} \hline
{\bf Author } & {\bf Date} & {\bf Revision} & {\bf Description} \\ \hline \hline
 Blair Thompson & November, 2009 & 1.0 & Initial Version \\ \hline
 David Hammen & October, 2010 & 2.0 & JEOD 2.1 \\ \hline
\end{tabular}



Quaternions are used throughout JEOD. These uses include:
\begin{itemize}
\item The \hypermodelref{REFFRAMES},
which uses left transformation unit quaternions
(along with transformation matrices) to represent the rotational
relationships between reference frames.
\item The \hypermodelref{MASS},
which uses left transformation unit quaternions
(along with transformation matrices) to represent the rotational
relationships between mass bodies.
\item The \hypermodelref{DYNBODY},
which uses quaternionic propagation to
model the time evolution of a dynamic body's rotational state.
\item The \hypermodelref{ORIENTATION},
which uses left transformation unit quaternions
as one of several mechanisms by which the user can specify
a rotational relationship.
\end{itemize}


\section{Documentation Organization}
\label{sec:docorg}
This document is formatted in accordance with the 
NASA Software Engineering Requirements Standard~\cite{NASA:SWE}.

The document comprises chapters organized as follows:

\begin{description}
%% longer chapter descriptions, more information.

\item[Chapter 1: Introduction] -
This introduction describes the objective and purpose of the \ModelDesc.

\item[Product Requirements] -
The requirements chapter describes the requirements on the \ModelDesc.

\item[Chapter 3: Product Specification] - 
The specification chapter describes
the architecture and design of the \ModelDesc.

\item[Chapter 4: User Guide] - 
The user guide chapter describes
how to use the \ModelDesc.

\item[Chapter 5: Verification and Validation] -  
The inspection, verification, and validation (IV\&V) chapter describes
the verification and validation procedures and results for the \ModelDesc.

\end{description}
