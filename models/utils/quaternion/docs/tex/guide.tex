\chapter{User Guide}\hyperdef{part}{user}{}\label{ch:user}
This chapter describes how to use the \quaternionDesc model from the
perspective of a simulation user, a simulation developer,
and a model developer.


\section{Analysis}
The Quaternion class is not intended to be used directly at the \Sdefine level.
Several JEOD models do however contain Quaternion objects as data elements.
Whether these elements can be set in the input or
logged to some log file depends on the object.

As the Quaternion data elements comprise the scalar and vector parts of the
quaternion, the standard Trick forms for setting / logging a scalar and a
three-vector double variable can be used to initialize  / log a Quaternion
element that embedded inside some larger data structure.

One very important caveat regarding the initialization and analysis of a
quaternion:
The user must be cognizant that different people use quaternions
in different ways represent the orientation of an object or reference frame.
In addition to the obvious sense issue (e.g., body to inertial versus
inertial to body), some use quaternions to represent rotation while
others use quaternions to represent transformations, and some use left
quaternions while others use right quaternions. The JEOD Quaternion model
uses left unit transformation quaternions.

Because multiple representation schemes exist, a quaternion provided by some
external source may need to be converted to the left unit transformation
quaternion form used by the JEOD Quaternion model. This is a simple
operation: negate the imaginary (vectorial) part of the quaternion.
 For example, a quaternion received from an external source that
uses right unit transformation quaternions will need to be conjugated
before it can be used within JEOD.


\section{Integration}
The Quaternion class is not intended to be used directly at the \Sdefine level.
Several JEOD models do however contain Quaternion objects as data elements.
The simulation developer should understand how quaternions are used in these
various models.

The ability to convert a quaternion to an eigen rotation provides a  
useful capability for the analysis of quaternions used to represent  
rotations. Suppose two alternate expressions such as the true versus  
navigated values of a quaternion are available. Analysts will want to  
know whether these quaternions differ significantly. Designating the  
two quaternions as Q1 and Q2, if the two were exactly equal, the  
product Q1 Q2* would be the pure real unit quaternion. An eigen  
decomposition of this product will yield an eigen angle and eigen unit  
vector. The eigen angle is very useful in assessing the difference  
between the quaternions. Embedding the member function eigen\_compare in the
\Sdefine file can provide this critical analytical capability.


\section{Extension}

The JEOD 2.0 Quaternion model lacks the full functionality described
in Appendix~\ref{sec:app_math}. One obvious extension
to the Quaternion class is to provide this full set of capabilities.
Other missing capabilities include:
\begin{itemize}
\item Support for right transformation quaternions, and
\item Support for additional charts on SO(3) such as 
Euler angles, Rodrigues parameters, and modified Rodrigues parameters.
\end{itemize}

The primary purpose of the Quaternion class is to be embedded as data
element within some other class. The JEOD Orientation, Reference Frame,
and Mass models exemplify this use. A model developer who wishes to
make such a use of the JEOD Quaternion model is encouraged to look
at these other models as a template.
