%%%%%%%%%%%%%%%%%%%%%%%%%%%%%%%%%%%%%%%%%%%%%%%%%%%%%%%%%%%%%%%%%%%%%%%%
%
% Purpose: Summarize the purpose of the LVLH Frames model
%
% 
%
%%%%%%%%%%%%%%%%%%%%%%%%%%%%%%%%%%%%%%%%%%%%%%%%%%%%%%%%%%%%%%%%%%%%%%%%%

\begin{abstract}

The JEOD \LvlhFrameDesc\ provides a simple standalone model for defining
a reference frame with respect to the position and velocity vectors of
an on-orbit object. This capability is extremely useful for certain
commonly encountered situations in spaceflight, such as tracking the
relative motion of multiple orbital vehicles. Making \LvlhFrameDesc\
distinct from the standard JEOD reference frames model allows the
\LvlhFrameDesc\ to be anchored to objects that are not DynBodies.

When used in concert with the LvlhRelativeDerivedState derived
class introducted with JEOD 3.2, \LvlhFrameDesc\ can be used to track
relative motion in either rectilinear or curvilinear coordinates.

\end{abstract}
