\chapter{Symplectic Euler's Method}\label{app:symplectic_euler}

\section{Overview}

The symplectic Euler method (also known as the semi-implicit Euler method,
the semi-explicit Euler method, the Euler-Cromer method)
is the simplest of integrators for second order ODE initial value problems.
The method propagates position and velocity by assuming a constant acceleration
and constant velocity across the integration interval. The technique advances
velocity to the end of the integration interval using the acceleration at the
start of the interval and advances position using the updated velocity.

\section{Classification}

The symplectic Euler method is a single step, single stage, single cycle
integrator for scalar-valued second order ODEs.
The only derivatives required are those at the start of an integration cycle,
which makes this method a single step integrator. 
The algorithm proceeds immediately to the end of the cycle,
which makes this method a single stage integrator.
The algorithm does not subdivide an integration tour into multiple cycles,
which makes this method a single cycle integrator.

As is the case for other solvers for scalar-valued second order ODEs, the
symplectic Euler method can easily be adapted to solving vector-valued
second order ODE initial value problems. In cannot be applied to initial value
problems of arbitrary order.

Like the basic Euler method, the symplectic Euler exhibits a local truncation
error that is proportional to the square of the step size and thus exhibits a
global truncation error that is proportional to the step size.
The symplectic Euler method is thus a first order integrator in terms of
accuracy.

\section{Implemented and Provided Techniques}

Table \ref{tab:symplectic_euler_method_problems} lists the types of initial value problems
that can be solved using the symplectic Euler method.
As noted in the table, the symplectic Euler method provides
solvers for all four types of initial value problems addressed by the
ER7 Utilities integration suite.

\begin{table}[htp]
\centering
\caption{Symplectic Euler Method Initial Value Problems}
\label{tab:symplectic_euler_method_problems}
\vspace{1ex}
\begin{tabular}{ll}\hline
\bf{Problem Type} & \bf{Support} \\
\hline\hline
\tt{FirstOrderODE} & Provided \\
\tt{SimpleSecondOrderODE} & Implemented \\
\tt{GeneralizedDerivSecondOrderODE} & Implemented \\
\tt{GeneralizedStepSecondOrderODE} & Implemented \\
\hline
\end{tabular}
\end{table}

\section{Mathematical Description}

The symplectic Euler method propagates velocity from time $t$ to $t+\Delta t$
by assuming a constant acceleration between $t$ and $t+\Delta t$
that is equal to the value at the start of the integration interval.
The method then propagates positon by assuming a constant velocity over the
integration interval that is equal to the value at the end of the interval:

\begin{equation}
\label{eqn:symplectic_euler_scalar_2nd_order_ode_step}
\begin{split}
v(t+\Delta t) &= v(t) + \Delta t f(t, x(t), v(t)) \\
x(t+\Delta t) &= x(t) + \Delta t v(t+\Delta t)
\end{split}
\end{equation}

\section{First Order ODE}

The symplectic Euler method explicitly addresses second order ODEs,
meaning that it cannot be used to solve an initial value problem
for a first order ODE.
The basic Euler method is used as a surrogate for symplectic
Euler.

\section{Second Order ODE}

The \erseven implementation of the symplectic Euler method applies
equation~(\ref{eqn:symplectic_euler_scalar_2nd_order_ode_step})
to each element of the position and velocity vectors.

\section{Generalized Position / Generalized Velocity ODE}

TBS

\section{Lie Group Second Order ODE}

TBS

\section{Usage Instructions}

To use the symplectic Euler method to integrate the dynamics of a monolithic
JEOD simulation
(\emph{i.e.}, one in which the dynamics manager performs the integration),
set the {\tt{jeod\_integ\_opt}} element of the simulation's
{\tt{DynManagerInit}} object to {\tt{trick.Integration.SymplecticEuler}}
in the appropriate Python input file.
