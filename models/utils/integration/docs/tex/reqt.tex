\chapter{Product Requirements}\hyperdef{part}{reqt}{}\label{ch:reqt}

\requirement{Project Requirements}
\label{reqt:toplevel}
\begin{description}
\item[Requirement:]\ \newline
  This model shall meet the JEOD project requirements specified in the 
  \hyperref{file:\JEODHOME/docs/JEOD.pdf}{part1}{reqt}{JEOD} top-level document.

\item[Rationale:]\ \newline
  This is a project-wide requirement.

\item[Verification:]\ \newline
  Inspection
\end{description}


\requirement{\erseven Framework} \label{reqt:use_er7_utils}
\begin{description}
\item[Requirement:]\ \newline
 The \ModelDesc shall provide extensions of the \erseven integration
 framework that enable it to be used in JEOD.

\item[Rationale:]\ \newline
 The \erseven integration framework was made intentionally generic so that
 it can be used in a Trick and/or a JEOD context.

\item[Remarks:]\ \newline
 The intent of this requirement is to extend the \erseven concept of an
 integration group and of a time interface to a JEOD-specific setting.

\item[Verification:]\ \newline
 Inspection, test
\end{description}


\requirement{\erseven Techniques} \label{reqt:supported_er7_techniques}
\begin{description}
\item[Requirement:]\ \newline
 The \ModelDesc shall provide access to all integration techniques defined in the
 \erseven integration module.

\item[Rationale:]\ \newline
 The intent of creating the \erseven was to provide a common framework for
 integration and to provide integration techniques formerly defined in Trick and in
 JEOD. Not providing access to all those techniques would be counterproductive.

\item[Verification:]\ \newline
 Inspection, test
\end{description}


\requirement{Long-Arc Integration} \label{reqt:long_arc_integration} \begin{description} \item[Requirement:]\ \newline
 The \ModelDesc shall provide techniques that accurately and efficiently
 integrate translational state over long spans of time.

\item[Rationale:]\ \newline
 This capability is needed for a number of human and automated simulations.

\item[Verification:]\ \newline
 Inspection, test
\end{description}


\requirement{Extensibility}
\label{reqt:extensibility}
\begin{description}
\item[Requirement:]\ \newline
 The \ModelDesc architecture shall support extensibility to enable the use of
 techniques not provided by the \erseven integration module.

\item[Rationale:]\ \newline
 Providing all of the plethora of numerical integration techniques
 that exist is not practical. However, the architecture should be
 extensible so as to accommodate techniques not provided.

\item[Verification:]\ \newline
 Inspection, test
\end{description}


\requirement{Support JEOD ODEs} \label{reqt:integration_problems} \begin{description} \item[Requirement:]\ \newline
 The \ModelDesc will provide support for the three classes
 of integration problems encountered in JEOD:
 \begin{enumerate}
   \item Scalar first order ordinary differential equations,
   \item Three-vector second order ordinary differential equations, and
   \item The Lie group SO3 as a second order ordinary differential equation.
 \end{enumerate}
 The support shall be in the form of classes that simplify the use of
 the broader classes of problems supported by the \erseven and
 that automatically allocate and deallocate resources per the
 Resource Allocation Is Initialization (RAII) scheme widely used
 across the C++ community.

\item[Rationale:]\ \newline
 These integration support constructs simplify the use of the
 integration elsewhere and reduce the chances of lost resources.

\item[Verification:]\ \newline
 Inspection, test
\end{description}


\requirement{Multiple States}
\label{reqt:multiple_states}
\begin{description}
\item[Requirement:]\ \newline
 The \ModelDesc shall provide the ability to integrate multiple states.

\item[Rationale:]\ \newline
 A simulation can involve multiple vehicles, each with its own translational
 and rotational state.

\item[Verification:]\ \newline
 Inspection, test
\end{description}


\requirement{Multiple Integrators}
\label{reqt:multiple_integrators}
\begin{description}
\item[Requirement:]\ \newline
 The \ModelDesc shall provide the ability to simultaneously integrate states
 using disparate integration rates and techniques.

\item[Rationale:]\ \newline
 A simulation can involve multiple vehicles, and each vehicle may have its
 own requirements on the propagation of its state.  Particularly in
 situations where the behaviors are governed by different scales of time
 (e.g. one vehicle in a geo-synchronous orbit and one in a low-Earth-orbit),
 the optimal integration rates (and, in some cases, methods) may differ from
 vehicle to vehicle.

\item[Verification:]\ \newline
 Inspection, test
\end{description}
