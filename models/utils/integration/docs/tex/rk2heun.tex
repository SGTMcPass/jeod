\chapter{Heun's Method}\label{app:rk2heun}

The RK2 method is the simplest of the predictor-corrector methods.  The
prediction of the final state is made using Euler's method, then the
derivatives computed for the final state are combined with the derivatives of
the initial state to produce a mean over the interval.  This mean value is
then used to correct the final state.

 This is a second-order, two-stage, single-step, single-cycle algorithm.

The first stage involves an Euler step to the end of the interval.
The second stage computes the mean value as the average of
the state derivative at the start of the interval and
the state derivative at the end of the interval
(as computed with the state from the first step).
Table~\ref{tab:heun_butcher} specifies the Butcher tableau for Heun's method.

\begin{table}[htp]
\centering
\caption{Heun's Method Butcher Tableau}
\label{tab:heun_butcher}
\vspace{1.5ex}
\begin{tabular}{c|cc}
0 && \\
1 & 1 & \\
\hline
& 1/2 & 1/2
\end{tabular}
\end{table}

Equations~(\ref{eqn:rk2_heun_0}) and~(\ref{eqn:rk2_heun_1}) describe
the equations that govern propagation using Heun's method.

Stage 0 (predictor):
\begin{equation}
\label{eqn:rk2_heun_0}
\begin{split}
t_1 &= t_f = t_i + \Delta t \\
\dot{\vect s}_0 &= \dot{\vect s}(t_i,\vect s(t_i)) \\
\vect s_1(t_1) &= \vect s(t_i) +  \Delta t\, \dot{\vect s}_0
\end{split}
\end{equation}
Stage 1 (corrector):
\begin{equation}
\label{eqn:rk2_heun_1}
\begin{split}
t_2 &= t_f = t_1 = t_i + \Delta t \\
\dot{\vect s}_1 &= \dot{\vect s}(t_1,\vect s(t_1)) \\
\bar {\dot{\vect s}} &= \frac{\dot{\vect s}_0+\dot{\vect s}_1} 2 \\
\vect s(t_f) &= \vect s(t) + \Delta t\, \bar {\dot{\vect s}}
\end{split}
\end{equation}

There are an infinite number of Runge Kutta methods for
any given order and number of stages.
For example, another second order two stage Runge Kutta method
is the midpoint method.
The midpoint method takes an Euler step to the midpoint of the
interval and uses the derivative at this midpoint as the mean value.

