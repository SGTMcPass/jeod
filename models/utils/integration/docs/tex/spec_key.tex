This section summarizes key algorithms used in the implementation
of the \ModelDesc.

\subsection{Integration Techniques}

Each integration technique is described in detail in an appendix to this
document. This section summarizes key characteristics of the techniques.
Table~\ref{tab:integ_technique_interfaces} presents a summary of
the integration techniques.

\newcommand\enumnamesboth[2]{%
  \multicolumn{1}{l}{{\tt #1}} &
  \multicolumn{1}{l|}{{\tt #2}}}
\newcommand\enumnamesfirst[1]{%
  \multicolumn{1}{l}{{\tt #1}} &
  \multicolumn{1}{l|}{N/A}}
\newcommand\enumnamesnone{%
  \multicolumn{1}{l}{N/A} &
  \multicolumn{1}{l|}{N/A}}

\begin{landscape}
\begin{table}[htp]
\begin{minipage}{\textwidth}
\centering
\caption{Integration Technique Characteristics}
\label{tab:integ_technique_interfaces}
\vspace{1ex}
\begin{tabular}{||l|ccccll|}
\hline
{\bf Technique} &
\tilt{\bf Accuracy (order)} &
\tilt{\bf Needs Priming} &
\tilt{\bf Number Stages Per Cycle
   \footnote{%
   This column indicates the number of steps per cycle after the technique
   is primed. Techhnique that needs priming typically exhibit different
   behavior during priming.}} &
\tilt{\bf First Step Derivatives
   \footnote{%
   This column indicates whether derivative data must be provided to the
   integrator on the first step of the integration cycle.
   Some integration techniques do not use the derivatives on the first step
   of an integration cycle. As computing derivatives is typically a rather
   expensive computational process, the computation of those derivatives can be
   bypassed if they aren't used.}} &
{ \bf er7\_utils Enum Value} &
{\bf Trick Enum Value}
\\ \hline \hline
Euler &
  $O(\Delta t)$ & No & 1 & Yes &
  \enumnamesboth{Euler}{Euler} \\
Symplectic Euler &
  $O(\Delta t)$ & No & 1 & Yes &
  \enumnamesboth{SymplecticEuler}{Euler\_Cromer} \\ 
Beeman's Algorithm &
  $O(\Delta t^2)$ & Yes & 2 & Yes &
  \enumnamesfirst{Beeman} \\
Second order Nystr\"{o}m-Lear &
  $O(\Delta t^2)$ & Yes & 2 & No &
  \enumnamesboth{NystromLear2}{Nystrom\_Lear\_2} \\
Position verlet &
  $O(\Delta t^2)$ & No & 2 & No &
  \enumnamesfirst{PositionVerlet} \\
Heun's method &
  $O(\Delta t^2)$ & No & 2 & Yes &
  \enumnamesboth{RK2Heun}{Runge\_Kutta\_2} \\
Midpoint method &
  $O(\Delta t^2)$ & No & 2 & Yes &
  \enumnamesfirst{RK2Midpoint} \\
Velocity verlet &
  $O(\Delta t^2)$ & No & 2 & Yes &
  \enumnamesfirst{VelocityVerlet} \\
Modified midpoint method &
  $O(\Delta t^3)$ & No & 2 & Yes &
  \enumnamesboth{ModifiedMidpoint4}{Modified\_Midpoint\_4} \\
4th order Adams-Bashforth-Moulton &
  $O(\Delta t^4)$ & Yes & 2 & Yes &
  \enumnamesboth{AdamsBashforthMoulton4}{ABM\_Method} \\
Runge-Kutta 4 &
  $O(\Delta t^4)$ & No & 4 & Yes &
  \enumnamesboth{RungeKutta4}{Runge\_Kutta\_4} \\
Runge-Kutta-Gill 4 &
  $O(\Delta t^4)$ & No & 4 & Yes &
  \enumnamesboth{RKGill4}{Runge\_Kutta\_Gill\_4} \\
 Runge-Kutta-Fehlberg 4/5 &
  $O(\Delta t^5)$ & No & 6 & Yes &
  \enumnamesboth{RKFehlberg45}{Runge\_Kutta\_Fehlberg\_45} \\
 Runge-Kutta-Fehlberg 7/8 &
  $O(\Delta t^8)$ & No & 12 & Yes &
  \enumnamesboth{RKFehlberg78}{Runge\_Kutta\_Fehlberg\_78} \\
Gauss-Jackson &
  var. & Yes & var. & Yes &
  \enumnamesnone \\
LSODE &
  var. & var. & var. & Yes &
  \enumnamesnone \\
\hline
\end{tabular}
\end{minipage}
\end{table}
\end{landscape}

% \textbf{First Step Derivatives} identifies whether derivative data must be
% provided to the integrator on the first step of the integration cycle.
% Some integration techniques do not use the derivatives on the first step
% of an integration cycle. As computing derivatives is typically a rather
% expensive computational process, the computation of those derivatives can be
% bypassed if they aren't used.
% 
% \textbf{Supply 1st Derivative} identifies whether the time-derivative of the
% zeroth-derivative-state must be passed in to the integrator (from outside).
% For the symplectic Euler and Beeman algorithms, the propagation of the
% zeroth-derivative-state (e.g. position) is based on an internally-generated
% mean-value of its derivative.

\subsection{Generalized Derivative Second Order ODEs}

\subsection{Lie Group Integration}
