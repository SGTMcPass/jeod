%%%%%%%%%%%%%%%%%%%%%%%%%%%%%%%%%%%%%%%%%%%%%%%%%%%%%%%%%%%%%%%%%%%%%%%%%%%%%%%%
% exec_summary.tex
% Memory model executive summary
%
%
%%%%%%%%%%%%%%%%%%%%%%%%%%%%%%%%%%%%%%%%%%%%%%%%%%%%%%%%%%%%%%%%%%%%%%%%%%%%%%%%

\chapter*{Executive Summary}

The \ModelDesc forms a component of the dynamics suite of
models within \JEODid. It is located at
models/utils/memory.

This model provides mechanisms in the form of macros and classes
for allocating and deallocating memory.
Several factors drove the design of the \ModelDesc.

\paragraph*{Environment.}
The model is to work within and outside of the Trick environment.
(Only the Trick-based capabilities are used in JEOD 2.0.)

\paragraph*{Flexibility.}
The model is to work with a wide variety of data types, including
pointers, C++ primitive types, and classes.

\paragraph*{C++ Concerns.} The C++ allocation operators
\verb|new| and \verb|new[]| have distinct semantics,
as do \verb|delete| and \verb|delete[]|.

\paragraph*{Ease of Use.} All of the above concerns should
be hidden from the programmers who use the model.
