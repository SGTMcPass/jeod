%%%%%%%%%%%%%%%%%%%%%%%%%%%%%%%%%%%%%%%%%%%%%%%%%%%%%%%%%%%%%%%%%%%%%%%%%%%%%%%%
% reqt.tex
% Toplevel requirements on the Memory Model
%
% 
%%%%%%%%%%%%%%%%%%%%%%%%%%%%%%%%%%%%%%%%%%%%%%%%%%%%%%%%%%%%%%%%%%%%%%%%%%%%%%%%

%----------------------------------
\chapter{Product Requirements}\hyperdef{part}{reqt}{}
\label{ch:reqt}
%----------------------------------

\requirement{Top-level requirement}
\label{reqt:toplevel}
\begin{description}
\item[Requirement:]\ \newline
  The \ModelDesc shall meet the JEOD project requirements specified in
  the \JEODid\
  \hyperref{file:\JEODHOME/docs/JEOD.pdf}{part1}{reqt}{ top-level
  document}.
\item[Rationale:]\ \newline
  This model shall, at a minimum,  meet all external and
  internal requirements 
  applied to the \JEODid\ release.
\item[Verification:]\ \newline
  Inspection 
\end{description}


\requirement{Memory Allocation}
\label{reqt:alloc}
\begin{description}
\item[Requirement:]\ \newline
  The \ModelDesc shall provide the ability to allocate and initialize
  memory of the data types described below.
  \subrequirement{Array of pointers to some data type.}
  \label{reqt::alloc_pointer_array}
  The model shall provide the ability to allocate an array of a specified
  number of pointers to a specified type, with each element of the allocated
  array initialized to the null pointer.
  \subrequirement{Array of structured objects.}
  \label{reqt::alloc_object_array}
  The model shall provide the ability to allocate an array of a specified
  number of objects of a specified structured (non-primitive) type, with each
  element of the allocated array initialized using the default constructor for
  the specified type.
  \subrequirement{Array of primitive objects.}
  \label{reqt::alloc_prim_array}
  The model shall provide the ability to allocate an array of a specified
  number of objects of a specified primitive data type, with each element of
  the allocated array initialized as zero.
  \subrequirement{Single structured object.}
  \label{reqt::alloc_object}
  The model shall provide the ability to allocate a single instance of a
  specified structured type, with the allocated object initialized using
  the constructor that matches a specified set of constructor arguments.
  \subrequirement{Single primitive object.}
  \label{reqt::alloc_prim}
  The model shall provide the ability to allocate a single instance of a
  specified primitive type, with the allocated object initialized to a
  specified value.
  \subrequirement{String duplication.}
  \label{reqt::strdup}
  The model shall provide the ability to create a copy of
  a specified null-terminated string.

\item[Rationale:]\ \newline
  The primary objective of the \ModelDesc is to make allocated memory visible to
  the simulation engine (e.g., Trick) in a manner that enables the simulation
  engine's data input and output mechanisms to operate on the memory.
  The simulation engine must be notified of memory allocations to the use of
  these input and output mechanisms. Encapsulating memory allocations is the
  first step in accomplishing this primary objective.
  
\item[Verification:]\ \newline
  Inspection, Test
\end{description}


\requirement{Memory Registration}
\label{reqt:registration}
\begin{description}
\item[Requirement:]\ \newline
  The \ModelDesc shall use the capabilities provided by the Simulation
  Interface Model to register all memory allocated using the capabilities
  described in requirement~\ref{reqt:alloc} with the simulation engine.

\item[Rationale:]\ \newline
  This requirement is the second step toward accomplishing the primary objective
  of making allocated memory visible to the simulation engine.

  The reason for using the Simulation Interface Model is that JEOD 2.1
  encapsulates all interactions with the simulation engine in that model.

\item[Verification:]\ \newline
  Inspection, Test
\end{description}


\requirement{Memory Deallocation}
\label{reqt:free}
\begin{description}
\item[Requirement:]\ \newline
  The \ModelDesc shall provide the ability to destroy and release memory
  previously allocated via the capabilities described in requirement
  \ref{reqt:alloc}.

\item[Rationale:]\ \newline
  Memory allocated by JEOD models must eventually be released to
  avoid memory leaks.

\item[Verification:]\ \newline
  Inspection, Test
\end{description}


\requirement{Freeing from a Base Class Pointer}
\label{reqt:base_class_pointer_free}
\begin{description}
\item[Requirement:]\ \newline
  The \ModelDesc shall properly destroy and release an object previously
  allocated by the model given a base class pointer to the object in question.

\item[Rationale:]\ \newline
  Derived class and base class pointers to the same object do not necessarily
  point to the same location in memory. The \ModelDesc machinery must be
  capable of handling this fact of life.

\item[Verification:]\ \newline
  Inspection, Test
\end{description}


\requirement{Memory Deregistration}
\label{reqt:deregistration}
\begin{description}
\item[Requirement:]\ \newline
  The \ModelDesc shall use the capabilities provided by the Simulation
  Interface Model to remove the registration of memory that is deallocated
  using the capabilities described in requirement~\ref{reqt:free}.

\item[Rationale:]\ \newline
  Memory that is released can be reused for other purposes. Failing to remove
  the registration for deallocated memory would make the simulation engine's
  memory model invalid.

\item[Verification:]\ \newline
  Inspection, Test
\end{description}


\requirement{Thread Safety}
\label{reqt:threads}
\begin{description}
\item[Requirement:]\ \newline
  The implementation of the \ModelDesc shall be thread-safe when used in
  a POSIX-complaint environment.

\item[Rationale:]\ \newline
  Trick makes extensive use of threads, as do many other simulation engines.

\item[Verification:]\ \newline
  Inspection, Test
\end{description}


\requirement{Memory Tracking}
\label{reqt:tracking}
\begin{description}
\item[Requirement:]\ \newline
  The \ModelDesc shall keep track of all outstanding memory allocations made
  with the model (memory allocated using the capabilities described in
  requirement~\ref{reqt:alloc} and not yet freed using the capabilities
  described in requirement~\ref{reqt:free}).

\item[Rationale:]\ \newline
  This capability is a nice-to-have for the purposes of debugging memory leaks
  but will be essential in future releases of JEOD to fully support
  checkpoint/restart.

\item[Verification:]\ \newline
  Inspection, Test
\end{description}


\requirement{Memory Reporting}
\label{reqt:reporting}
\begin{description}
\item[Requirement:]\ \newline
  The \ModelDesc shall optionally report individual memory transactions and
  shall optionally report outstanding memory allocations at simulation end time.

\item[Rationale:]\ \newline
  This capability provides a debugging aid. This capability needs to be
  optional as the output is of little interest except when needed.

\item[Verification:]\ \newline
  Inspection, Test
\end{description}


\requirement{Checkpoint/Restart (FUTURE)}
\label{reqt:checkpoint_restart}
Reserved.
