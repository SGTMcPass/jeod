%%%%%%%%%%%%%%%%%%%%%%%%%%%%%%%%%%%%%%%%%%%%%%%%%%%%%%%%%%%%%%%%%%%%%%%%%%%%%%%%
% guide.tex
% User Guide for the Container Model.
%
% 
%%%%%%%%%%%%%%%%%%%%%%%%%%%%%%%%%%%%%%%%%%%%%%%%%%%%%%%%%%%%%%%%%%%%%%%%%%%%%%%%

\chapter{User Guide}\hyperdef{part}{user}{}\label{ch:user}
The typical JEOD model User Guide looks at the model at hand from the
perspectives of a simulation user,
a simulation developer, and a model developer.
The \ModelDesc is not the typical JEOD model User Guide.
The model is designed to be transparent to the typical
simulation user and simulation developer.

\section{Instructions for Simulation Users}
The container model is not intended for use by simulation users.

\section{Instructions for Simulation Developers}
The container model is not intended for use by simulation developers.

\section{Instructions for Model Developers}
This section describes the use of the model from the perspectives of
a user of the STL sequence container replacements,
a model developer who wishes to write a class that derives from
one of the provided checkpointable/restartable classes,
and a simulation interface developer who is porting the model outside of Trick.

\subsection{Model Header Files}
The \ModelDesc contains 23 header files,
categorized into four major groups:
\begin{itemize}
\item Headers that define incomplete classes that form the
  basis for the checkpoint and restart. External users are
  free to create their own extensions of these incomplete classes.
\item Headers that define non-checkpointable replacements for
  the STL sequence containers. These are incorporated into
  checkpointable replacements in other headers.
\item Headers that define usable, but rather hard to name,
  checkpointable and restartable replacements for the STL containers.
  For example, a list that contains pointers to objects of type Foo
  is a \template{JeodPointerContainer}
  {\template{JeodList}{Foo$\ast$}, Foo}.
\item Headers that simplify the process of defining
  checkpointable and restartable replacements for the STL containers.
  For example, to change a data member declared as
  \template{std::list}{Foo$\ast$} foo\funder{}ptr\funder{}list
  to its JEOD equivalent,
  change the type to  \template{JeodPointerList}{Foo}::type and
  \#include ``utils/container/include/pointer\_list.h'' instead of
  \inanglebrackets{list}.
\end{itemize}
Only the first and last group are discussed in this User Guide.
Table~\ref{tab:guide_headers} describes these header files.


\begin{table}[htp]
\centering
\caption{Model Header Files}
\label{tab:guide_headers}
\vspace{1.5ex}
\centering
\begin{tabular}{|| p{1.4in} p{1.7in} p{2.9in}|} \hline
{\bf Header} & {\bf Defines} & {\bf Description} \\ \hline\hline

\longentry checkpointable.hh &
  \longentry
  JeodCheckpointable &
  \raggedright
  Base class for something that can be
  checkpointed and restored from a checkpoint.
\tabularnewline[4pt]

\longentry simple\funder{}checkpointable.hh &
  \longentry
  SimpleCheckpointable &
  \raggedright
  A simplified version of JeodCheckpointable.
  Extenders just need to define simple\funder{}restore().
\tabularnewline[30pt]

\longentry object\funder{}list.hh &
  \longentry
  \template{JeodObjectList}{ElemType}::type &
  \raggedright
  Use in lieu of \template{std::list}{ElemType}
  for structured ElemType data.
\tabularnewline[4pt]

\longentry object\funder{}set.hh &
  \longentry
  \template{JeodObjectSet}{ElemType}::type &
  \raggedright
  Use in lieu of \template{std::set}{ElemType}
  for structured ElemType data.
\tabularnewline[4pt]

\longentry object\funder{}vector.hh &
  \longentry
  \template{JeodObjectVector}{ElemType}::type &
  \raggedright
  Use in lieu of \template{std::vector}{ElemType}
  for structured ElemType data.
\tabularnewline[4pt]

\longentry pointer\funder{}list.hh &
  \longentry
  \template{JeodPointerList}{ElemType}::type &
  \raggedright
  Use in lieu of \template{std::list}{ElemType$\ast$}
  (list of pointers).
\tabularnewline[4pt]

\longentry pointer\funder{}set.hh &
  \longentry
  \template{JeodPointerSet}{ElemType}::type &
  \raggedright
  Use in lieu of \template{std::set}{ElemType$\ast$}
  (set of pointers).
\tabularnewline[4pt]


\longentry pointer\funder{}vector.hh &
  \longentry
  \template{JeodPointerVector}{ElemType}::type &
  \raggedright
  Use in lieu of \template{std::vector}{ElemType$\ast$}
  (vector of pointer).
\tabularnewline[4pt]

\longentry primitive\funder{}list.hh &
  \longentry
  \template{JeodPrimitiveList}{ElemType}::type &
  \raggedright
  Use in lieu of \template{std::list}{ElemType}
  for primitive ElemType data.
\tabularnewline[4pt]

\longentry primitive\funder{}set.hh &
  \longentry
  \template{JeodPrimitiveSet}{ElemType}::type &
  \raggedright
  Use in lieu of \template{std::set}{ElemType}
  for primitive ElemType data.
\tabularnewline[4pt]

\longentry primitive\funder{}vector.hh &
  \longentry
  \template{JeodPrimitiveVector}{ElemType}::type &
  \raggedright
  Use in lieu of \template{std::vector}{ElemType}
  for primitive ElemType data.
\tabularnewline[4pt]

\hline
\end{tabular}
\end{table}
\clearpage

\subsection{Using the Container Model}
This section describes how to use the data types defined by the
\ModelDesc in some other model.

\subsubsection{STL Sequence Containers}
Classes that contain one or more data members that are STL sequence containers
are not checkpointable or restartable as-is. The approach taken by JEOD is
to replace the STL types with their JEOD equivalents and to register the
objects with JEOD. The code changes that need to be made to accomplish this are:
\begin{itemize}
  \item In the header file for the class, replace the STL \#includes
    and the STL sequence types with the JEOD equivalents.
    See the last set of entries in table~\ref{tab:guide_headers}
    for the appropriate header files and data types.
    The following example illustrates a class that contains a list
    of pointers and a vector of doubles.
    \begin{verbatim}
    #include "utils/container/include/pointer_list.hh"
    #include "utils/container/include/primitive_vector.hh"
    class Foo;

    class Bar {
    public:
       Bar();
    protected:
       JeodPointerList<Foo>::type foo_list;
       JeodPrimitiveVector<double>::type double_vector;
    };
    \end{verbatim}
  \item In the constructor(s) for the class, register with the
    JEOD memory manager the class itself, the class in the list or
    vector (you don't need to register primitives), and the containers.
    The constructor for the class defined above is illustrated below.
    \begin{verbatim}
    #include "utils/memory/include/jeod_alloc.hh"
    #include "../include/foo.hh"
    #include "../include/bar.hh"

    Bar::Bar ()
    {
       JEOD_REGISTER_CLASS (Bar);
       JEOD_REGISTER_CLASS (Foo);
       JEOD_REGISTER_CHECKPOINTABLE (this, foo_list);
       JEOD_REGISTER_CHECKPOINTABLE (this, double_vector);
    }
    \end{verbatim}
  \item In the destructor for the class, deregister the containers
    with the JEOD memory manager.
    \begin{verbatim}
    Bar::~Bar ()
    {
       JEOD_DEREGISTER_CHECKPOINTABLE (this, foo_list);
       JEOD_DEREGISTER_CHECKPOINTABLE (this, double_vector);
    }
    \end{verbatim}
\end{itemize}

\subsection{Extending the Container Model}
This section describes how to extend the data types defined by the
\ModelDesc.


\subsubsection{Extending the JeodCheckpointable Class}
\label{sec:guide_JeodCheckpointable}
There are many things beyond STL containers that a simulation engine may not
be able to checkpoint or restart without some additional help. One way
to provide that additional help is to build a class that derives from
JeodCheckpointable.

The class JeodCheckpointable declares several virtual methods with empty
implementations and several pure virtual methods with no implementation
at all. A usable extension to JeodCheckpointable must at a minimum
provide implementations for these pure virtual methods.
Section~\ref{sec:spec_JeodCheckpointable} discusses the class in detail,
including descriptions of how to extend the class.

An instance of a class that derives from JeodCheckpointable will not checkpoint
or restore itself. Checkpointing and restoring checkpointable objects is
performed by the \SIMINTERFACE. A checkpointable object must be registered
with JEOD via the macro JEOD\_REGISTER\_CHECKPOINTABLE to make the
checkpointable object subject to checkpoint and restart.


\subsubsection{Extending the SimpleCheckpointable Class}
Directly extending JeodCheckpointable requires one to implement each of
the JeodCheckpointable pure virtual methods. This is a bit much in the
case of a simple, atomic checkpoint and restart process. The class
SimpleCheckpointable provides a much simpler checkpoint/restart interface that
may be used in the case of these easily restored objects.

For example, consider a model that uses random access to read binary data from some file. The underlying file stream is an opaque data type and is not directly
checkpointable or restorable. What can be done is to maintain the name of the
file and the location of the read pointer as data members. These are simple
data types that the simulation engine can checkpoint and restore.
A tiny bit of glue is needed to reopen the file stream. The example below uses a data member whose type derives from SimpleCheckpointable to make
the model restartable.
    \begin{verbatim}
    #include <cstdio>
    #include <string>
    #include "utils/container/include/simple_checkpointable.hh"
    #include "utils/memory/include/jeod_alloc.hh"

    class Foo {
    public:

       // This class calls Foo's reopen method on restart.
       class Restart : public SimpleCheckpointable {
       public:
          // All checkpointable classes need a default constructor.
          Restart () {}
       
          // This constructor creates a viable Foo::Restart object.
          explicit Restart (Foo & foo) : foo_ptr(&foo) {}

          // simple_restore() is called by the restart machinery.
          virtual void simple_restore ()
          {
             foo_ptr->reopen();
          }
          
       private:
          Foo * foo_ptr; // -- Pointer to owner
       };

       // Constructor.
       Foo()
       {
          // All checkpointable objects must be registered with JEOD.
          // First register the containing class, then the object.
          JEOD_REGISTER_CLASS (Foo);
          JEOD_REGISTER_CHECKPOINTABLE (this, restart);
       }
       
       // Destructor.
       ~Foo()
       {
          // All registered checkpointable objects must be deregistered
          // when they go out of scope.
          JEOD_DEREGISTER_CHECKPOINTABLE (this, restart);
       }
       
       // Reopen the file on restart. Code elided.
       void reopen ();

    protected:
       std::string fname; // -- The name of the file.
       long foffset; // -- The current offset within the file.

       std::FILE * fstream; // ** The input file stream.
       Restart restart; // ** The checkpoint/restart agent for this class.
    };
    \end{verbatim}
In the above example, the embedded class Foo::Restart derives from
SimpleCheckpointable. The class Foo contains an instance of Foo::Restart
as a data member. This member acts as a restart agent that restores the
containing Foo object's file pointer by calling that object's reopen method.

\subsubsection{Which Class to Extend}
A model developer that wishes to make some object checkpointable and
restartable using the mechanisms provided by this model has several choices
regarding which class to extend.\begin{itemize}
\item If the object is one of the STL containers such as std::deque for which a
JEOD implementation does not exist, the appropriate place to start is by
extending the JeodSequenceContainer or JeodAssociativeContainer class
as appropriate.
\item If the object can be restored by extending SimpleCheckpointable,
do so. One could also extend directly from JeodCheckpointable, but that entails
a lot more effort.
\item Finally, one can write an extension of JeodCheckpointable if
all other options are off the table.
\end{itemize}

\subsection{Implementing Checkpoint and Restart}
The JEOD/Trick simulation interface model implements checkpointing and
restarting of checkpointable objects in the Trick simulation environment.
When JEOD is used outside of the Trick simulation environment, the
implementation must provide an interface between JEOD and the simulation
engine. If this new interface is to provide checkpoint/restart capabilities,
it must do so in a manner consistent with the capabilities provided by
this model. See the \hypermodelref{SIMINTERFACE} for details on the
implementation of the checkpoint/restart capability in the
 JEOD/Trick simulation interface.

