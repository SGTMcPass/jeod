%%%%%%%%%%%%%%%%%%%%%%%%%%%%%%%%%%%%%%%%%%%%%%%%%%%%%%%%%%%%%%%%%%%%%%%%%%%%%%%%
% reqt.tex
% Toplevel requirements on the Container Model
%
% 
%%%%%%%%%%%%%%%%%%%%%%%%%%%%%%%%%%%%%%%%%%%%%%%%%%%%%%%%%%%%%%%%%%%%%%%%%%%%%%%%

%----------------------------------
\chapter{Product Requirements}\hyperdef{part}{reqt}{}
\label{ch:reqt}
%----------------------------------

\requirement{Top-level requirement}
\label{reqt:toplevel}
\begin{description}
\item[Requirement:]\ \newline
  The \ModelDesc shall meet the JEOD project requirements specified in
  the \JEODid\
  \hyperref{file:\JEODHOME/docs/JEOD.pdf}{part1}{reqt}{ top-level
  document}.
\item[Rationale:]\ \newline
  This model shall, at a minimum, meet all external and
  internal requirements 
  applied to the \JEODid\ release.
\item[Verification:]\ \newline
  Inspection 
\end{description}


\requirement{Checkpoint/Restart}
\label{reqt:chkpt}
\begin{description}
\item[Requirement:]\ \newline
  The \ModelDesc shall provide a generic framework for checkpointing resources
  that would otherwise be opaque to simulation engines and for restoring those
  resources from a checkpoint file.
  \subrequirement{Extensibility.}
  \label{reqt::checkpt_extensibility}
  The basic checkpoint/restart capability shall provide a means for extension
  such that developers of models that contain opaque resources can have these
  resources checkpointed and restarted from a checkpoint.

  \subrequirement{Transparency.}
  \label{reqt::checkpt_transparency}
  The implementation details of this extensibility framework shall be
  transparent to the models that perform the checkpointing of the
  checkpointable contents to a checkpoint file, and the restoration of
  resources from a previously recorded checkpoint file.

  In other words, the implementation shall define a set of virtual functions
  that specify the checkpoint/restart interface. The checkpoint/restart
  framework calls the generic interfaces without concern of how the
  extensions implement the required behavior.

  \subrequirement{Undefined behavior.}
  \label{reqt::checkpt_bad_scoobies}
  The effects of copying, swapping, or overwriting checkpoint information
  is undefined.

  This means that extenders should not write a true swap() method and should
  be careful not to copy or overwrite checkpoint information in their
  copy constructors and assignment operators. Note that this is not a
  requirement on the model. It is a requirement on model extenders:
  Do not invoke undefined behavior.

\item[Rationale:]\ \newline
  The primary objective of the \ModelDesc is to make all of JEOD checkpointable
  and restartable from a checkpoint. A standard framework enables the agents
  that perform the checkpointing and restoring to do so without knowledge of
  the contents to be checkpointed and restored. An extensible framework enables
  the developers of models with opaque content to specify the mechanisms for
  checkpointing and restoring such content.
  
\item[Verification:]\ \newline
  Inspection, Test
\end{description}


\requirement{STL Containers}
\label{reqt:stl_containers}
\begin{description}
\item[Requirement:]\ \newline
  The \ModelDesc shall provide checkpointable and restorable replacements for
  the STL list, set, and vector class templates.

  \subrequirement{Full functionality.}
  \label{reqt::stl_container_equiv}
  With the noted exceptions, each STL container replacement shall
  provide the full functionality of its STL counterpart as specified in
  the 2003 \Cplusplus Standard, ISO/IEC 14882:2003 \cite{cpp2003}.

  Exceptions:\begin{itemize}
  \item Support for non-default Compare or Allocator classes is not required.
  \item The implementation shall provide a default constructor and
        a copy constructor for each STL container replacement.
        The implementation does not need to provide the full suite of
        constructors defined by the Standard for the STL containers.
  \item Where the standard itself is known to be buggy, the implementation
        should follow the recommended correction to the standard.
        This exception is defined as reported issues that have
        reached at least CD1 status with the \Cplusplus Standards Committee
        (ISO/IEC JTC1/SC22/WG21).
  \item The implementation shall provide the ability to swap the STL contents
        of a JEOD container with another JEOD or STL container of the same base
        type.
        The JEOD equivalent of the STL swap method shall not touch the
        internal checkpoint information and thus shall not be named swap.
        See requirement~\ref{reqt::checkpt_bad_scoobies}.
  \end{itemize}

  \subrequirement{Transparency.}
  \label{reqt::stl_container_trans}
  Excluding the exceptions specified above, the JEOD equivalents of the
  STL containers shall use the function names and signatures as
  specified in the 2003 \Cplusplus Standard for the STL class templates.

\item[Rationale:]\ \newline
  This requirement exists because the STL containers were the number one
  culprit that formerly made JEOD unable to be fully checkpointed or restarted.

  The functionality and transparency requirements mean that only the
  data types of those STL container data members need to be changed.
  There should be very little, if any, ripple effect resulting from the
  change in type.

\item[Verification:]\ \newline
  Inspection, Test
\end{description}