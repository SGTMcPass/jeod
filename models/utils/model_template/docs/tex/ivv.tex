%%%%%%%%%%%%%%%%%%%%%%%%%%%%%%%%%%%%%%%%%%%%%%%%%%%%%%%%%%%%%%%%%%%%%%%%%%%%%%%%
%
% ivv.tex
%
% Inspection, verification, and validation of the <model name>
%
% Usage instructions:
% This file should comprise a
% - The \chapter command, which must not be changed.
% - A number of \section command. You can change the labels but do not
%   change the names, and do not add any sections of your own.
% - Text to fill in the contents of the sections (which you must provide).
%
% Feel free to:
% - Change the labels on the \section commands.
% - Use your own \subsection commands and below. Structure is good.
% - Use figures and tables.
% - Split the file into parts if it gets too big.
%
%%%%%%%%%%%%%%%%%%%%%%%%%%%%%%%%%%%%%%%%%%%%%%%%%%%%%%%%%%%%%%%%%%%%%%%%%%%%%%%%

\chapter{Inspection, Verification, and Validation}
\hyperdef{part}{ivv}{}\label{ch:ivv}

\section{Inspection}\label{sec:inspect}
This section describes the inspections of the \MODELTITLE.

% %%%%%%%%%%%%%%%%%%%%%%%%%%%%%%%%%%%%%%%%%%%%%%%%%%%%%%%%%%%%%%%%%%%%%%%%%%%%%%%%
% ivv_inspect.tex
% Inspections of the Simulation Interface Model
%
% 
%%%%%%%%%%%%%%%%%%%%%%%%%%%%%%%%%%%%%%%%%%%%%%%%%%%%%%%%%%%%%%%%%%%%%%%%%%%%%%%%

\inspection{Top-level Inspection}
\label{inspect:TLI}
This document structure, the code, and associated files have been inspected.
With the exception of the cyclomatic complexity of the
\verb|JeodTrickMemoryInterface::primitive attributes| method,
the \ModelDesc satisfies requirement~\traceref{reqt:toplevel}.
A waiver has been granted for this one exception.

\inspection{Design Inspection}
\label{inspect:design}
%\tracingall
Table~\ref{tab:design_inspection} summarizes the key elements of the
implementation of the \ModelDesc that satisfy requirements levied on the model.
By inspection, the \ModelDesc satisfies
requirements~\tracerefrange{reqt:hidden_data_visibility}{reqt:extensibility}.

\begin{longtable}%
  {||l @{\hspace{4pt}} %
   >{\raggedright\arraybackslash}p{1.38in} |%
   >{\raggedright\arraybackslash}p{3.95in}|}
\caption{Design Inspection}
\label{tab:design_inspection} \\[6pt]
\hline
\multicolumn{2}{||l|}{\bf Requirement} & \bf{Satisfaction}
\\ \hline\hline
\endfirsthead

\caption[]{Design Inspection (continued from previous page)} \\[6pt]
\hline
\multicolumn{2}{||l|}{\bf Requirement} & \bf{Satisfaction}
\\ \hline\hline
\endhead

\hline \multicolumn{3}{r}{{Continued on next page}} \\
\endfoot

\hline
\endlastfoot

\ref{reqt:hidden_data_visibility} & Hidden Data Visibility &
  The macro \verb|JEOD_MAKE_SIM_INTERFACES| provides the required
  simulation engine agnostic capability. The implementation of this macro
  is simulation engine specific. Trick-specific implementations make
  protected and private data visible to Trick.
\tabularnewline[4pt]

\ref{reqt:allocated_data_visibility} & Allocated Data Visibility &
  The member functions
  \verb|register_allocation| and \verb|deregister_allocation|
  in the abstract class \verb|JeodTrickMemoryInterface|
  specify the simulation engine agnostic capability. The implementations
  of these methods in the class \verb|JeodTrickMemoryInterface|
  make allocated data visible to Trick.
\tabularnewline[4pt]

\ref{reqt:sim_engine_interface} & Simulation Engine Interface &
  The only Trick dependencies outside of the \ModelDesc are the
  noted exception in the \DYNMANAGER. All other simulation engine
  interfaces are encapsulated within the \ModelDesc.
\tabularnewline[4pt]

\ref{reqt:integ_interface} & Integration Interface &
  The abstract class \verb|IntegratorInterface| provides the required
  simulation engine agnostic capabilities. Trick-specific
  classes that derive from this abstract class provide implementations
  of the required functionality in both the Trick 7 and Trick 10
  environments.
\tabularnewline[4pt]

\ref{reqt:job_cycle} & Job Cycle &
  The function \verb|JeodSimulationInterface::get_job_cycle|
  is the public interface to this required functionality.
  This invokes the pure virtual protected member function
  \verb|get_job_cycle_internal|.
  The class \verb|BasicJeodTrickSimInterface| implements this
  function in the context of a Trick-based simulation.
\tabularnewline[4pt]

\ref{reqt:trick_message_handler} & Trick Message Handler &
  The \ModelDesc provides the class \verb|TrickMessageHandler|, which
  derives from the class \verb|SuppressedCodeMessageHandler|
  and which implements the functionality required of a \verb|MessageHandler|
  using Trick's messaging system.
\tabularnewline[4pt]

\ref{reqt:checkpoint_restart} & Checkpoint/Restart &
  The \ModelDesc provides a generic checkpoint/restart capability in the
  form of classes that create and read from a sectioned checkpoint file.
  The class \verb|JeodTrick10MemoryInterface| uses these
  generic capabilities to provide the required ability to make
  JEOD-based simulations checkpointable and restartable in a
  Trick 10 environment.
\tabularnewline[4pt]

\ref{reqt:addr_name_xlate} & Address/Name Translation &
  The functions \verb|get_name_at_address| and \verb|get_address_at_name|
  in the classes \verb|JeodSimulationInterface| (static) and
  \verb|JeodMemoryInterface| (pure virtual)
  are the public interfaces to this required functionality.
  The class \verb|JeodTrickMemoryInterface| provides dummy
  implementations while \verb|JeodTrick10MemoryInterface| provides
  functional implementations of these functions.
\tabularnewline[4pt]

\ref{reqt:multiple_integ_groups} & Multiple Integration Groups &
  The \ModelDesc implements this requirement as a set of JEOD-agnostic
  classes (which may eventually be migrated out of JEOD) and the JEOD-aware
  class \verb|JeodDynbodyIntegrationLoop|.
\tabularnewline[4pt]

\ref{reqt:extensibility} & Extensibility &
  The \ModelDesc was carefully designed to have the public interfaces be in
  the form generic macros and abstract, simulation engine agnostic classes.
  The Trick independent demonstration and the test harness
  used in the JEOD unit tests illustrate to some extent
  that the model can be extended for use outside of the Trick environment.
\tabularnewline[4pt]

\end{longtable}

\newpage
\section{Test}
This section describes various tests conducted to demonstrate
that the \MODELTITLEx satisfies the requirements levied against it.
The tests described in this section
are archived in the JEOD directory FIXME.


% \test{Simulation Interface Simulation}
\label{test:local_verif}
\begin{description}
\item[Background]
This test is an extremely simple simulation which
creates a Trick sim\_object containing 
an instance of a JeodTrickSimInterface to be tested.
One other Trick sim\_object creates a test object
with a single scheduled job.  The 
test object calls JeodTrickSimInterface::get\_job\_cycle() as
a scheduled job, thus the output can be compared
to the cycle time of the scheduled job.

The test object also includes private fields which are initialized by the 
Trick input processor and are logged by Trick.

\item[Test directory] {\tt verif} \\
This is a standard verification directory containing
the simulation directory  {\tt SIM\_sim\_interface}
along with src and include directories for the test class code.

\item[Success criteria]
The simulation includes initialization and logging of private fields 
which also reflect the output of 
JeodTrickSimInterface::get\_job\_cycle(). The logged data should be
identical to the 
reference data in the SET\_test\_val directory.

\item[Test results]
Passed.

\item[Applicable requirements]
This test demonstrates the satisfaction of
requirements \traceref{reqt:hidden_data_visibility},
\traceref{reqt:sim_engine_interface},
and \traceref{reqt:job_cycle}.
\end{description}


\test{Container Simulation}
\label{test:container_model_sim}
\begin{description}
\item[Background]
This simulation, located in the \CONTAINER\ verification directory,
exercises the checkpoint/restart capabilities of the model.
For a complete description of this test, see the
\hypermodelrefinside{CONTAINER}{part}{ivv} for details.
\item[Test Directory]
\verb|models/utils/container/verif/SIM_container_T10|
\item[Test Results]
Passed.
\item[Applicable Requirements]
This test demonstrates the satisfaction of
requirements \traceref{reqt:allocated_data_visibility},
\traceref{reqt:sim_engine_interface},
\traceref{reqt:checkpoint_restart},
and \traceref{reqt:addr_name_xlate}.
\end{description}


\test{Memory Simulation}
\label{test:memory_model_sim}
\begin{description}
\item[Background]
This simulation, located in the \MEMORY\ verification directory,
tests the ability of the \MEMORY\ to allocate and deallocate
memory. This in turn tests the ability of this model to
make those allocations visible to the simulation engine.
 For a complete description of this test, see the
\hypermodelrefinside{MEMORY}{part}{ivv} for details.
\item[Test Directory]
\verb|models/utils/memory/verif/SIM_memory_T10|
\item[Test Results]
Passed.
\item[Applicable Requirements]
This test demonstrates the satisfaction of
requirements
\traceref{reqt:allocated_data_visibility},
\traceref{reqt:sim_engine_interface},
and \traceref{reqt:checkpoint_restart}.
\end{description}


\test{Message Handler Simulation}
\label{test:message_model_sim}
\begin{description}
\item[Background]
This simulation, located in the \MESSAGE\ verification directory,
exercises the Trick-based implementation of the abstract
class MessageHandler.
For a complete description of this test, see the
\hypermodelrefinside{MESSAGE}{part}{ivv} for details.
\item[Test Directory]
\verb|models/utils/message/verif/SIM_message_handler_verif_T10|
\item[Test Results]
Passed.
\item[Applicable Requirements]
This test demonstrates the satisfaction of
requirements \traceref{reqt:sim_engine_interface}
and \traceref{reqt:trick_message_handler}.
\end{description}


\test{Propagated Planet Simulation}
\label{test:sim_prop_planet}
\begin{description}
\item[Background]
This simulation, located in the \EPHEMERIDES\ verification directory,
tests the ability of the \EPHEMERIDES\ to propagate a planet as an
alternative to using an ephemeris model.
The Trick 10 version was constructed to serve as a test case
of the multiple integration group capability provided by this model. 
 For a complete description of this test, see the
\hypermodelrefinside{EPHEMERIDES}{part}{ivv} for details.
\item[Test Directory]
\verb|models/environment/ephemerides/verif/SIM_prop_planet_T10|
\item[Test Results]
Passed.
\item[Applicable Requirements]
This test demonstrates the satisfaction of
requirements \traceref{reqt:allocated_data_visibility},
\traceref{reqt:sim_engine_interface},
\traceref{reqt:integ_interface},
\traceref{reqt:checkpoint_restart},
\traceref{reqt:addr_name_xlate},
and \traceref{reqt:multiple_integ_groups}.
\end{description}



\newpage
\section{Metrics}

Table~\ref{tab:coarse_metrics} presents coarse metrics on the source
files that comprise the model.
\input{coarse_metrics}

Table~\ref{tab:metrix_metrics} presents the extended cyclomatic complexity (ECC)
of the methods defined in the model.
\input{metrix_metrics}


\newpage
\section{Requirements Traceability}
This section is intentionally left blank for this release.
%Table~\ref{tab:reqt_ivv_xref} summarizes the inspections and tests
that demonstrate the satisfaction of the requirements levied on the model.

\begin{table}[htp]
\centering
\caption{Requirements Traceability}
\label{tab:reqt_ivv_xref}
\vspace{1ex}
\centering
\begin{tabular}{||l @{\hspace{4pt}} l|l @{\hspace{2pt}} l @{\hspace{4pt}} l|} \hline
\multicolumn{2}{||l|}{\bf Requirement} &
\multicolumn{3}{l|}{\bf Inspection or test} \\ \hline\hline
\ref{reqt:toplevel} & Project Requirements &
     Insp. & \ref{inspect:TLI}     & Top-level Inspection
\tabularnewline[4pt]
\ref{reqt:MHI} & Message Handler Implementation &
     Insp. & \ref{inspect:message} & Code Reference \\
\tabularnewline[4pt]
\ref{reqt:MII} & Memory Interface Implementation &
     Insp. & \ref{inspect:memory} & Code Reference \\
\tabularnewline[4pt]
\ref{reqt:III} & Integrator Interface Implementation &
     Insp. & \ref{inspect:integrator} & Code Reference \\
\tabularnewline[4pt]
\ref{reqt:JSI} & Jeod Simulation Interface Implementation &
     Insp. & \ref{inspect:jtsi} & Code Reference \\
  && Test  & \ref{test:jtsi}     & test get\_job\_cycle sim \\
\tabularnewline[4pt]
\ref{reqt:DA} & Data Access &
     Test & \ref{test:jtsi} & init and logging test \\
\tabularnewline[4pt]
\hline
\end{tabular}
\end{table}

