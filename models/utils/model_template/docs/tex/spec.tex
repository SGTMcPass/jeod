%%%%%%%%%%%%%%%%%%%%%%%%%%%%%%%%%%%%%%%%%%%%%%%%%%%%%%%%%%%%%%%%%%%%%%%%%%%%%%%%
%
% spec.tex
%
% Specifications of the <model name>
%
% Usage instructions:
% This file should comprise a
% - The \chapter command, which must not be changed.
% - A number of \section command. You can change the labels but do not
%   change the names, and do not add any sections of your own.
% - Text to fill in the contents of the sections (which you must provide).
%
% Feel free to:
% - Change the labels on the \section commands.
% - Use your own \subsection commands and below. Structure is good.
% - Use figures and tables.
% - Split the file into parts if it gets too big.
%
%%%%%%%%%%%%%%%%%%%%%%%%%%%%%%%%%%%%%%%%%%%%%%%%%%%%%%%%%%%%%%%%%%%%%%%%%%%%%%%%

\chapter{Product Specification}\hyperdef{part}{spec}{}\label{ch:spec}

\section{Conceptual Design}
\label{sec:conceptual_design}

% Provide a conceptual description of the model.
% This section typically provides answers to the following:
% - What are the key concepts of the model?
% - What is the model architecture?
% - With what other models does this model interact?

% If needed, put a \clearpage at the end of the section to ensure that all
% floats (figures and tables) are printed before the start of the next section.

\section{Mathematical Formulation}
\label{sec:mathematics}

% Note: Do not delete this section for a non-mathematical model. Instead use N/A
% or some other designation to indicate that the section is blank.
% For all other models, this is where you should describe the mathematics that
% underlie your model. Do not put those mathematical descriptions in the
% conceptual design or detailed design sections.


\section{Detailed Design}
\label{sec:detailed_design}

% Provide a detailed description of the model.
% This section typically provides answers to the following:
% - What is the model API?
% - What design decisions are not captured in the API that some future
%   maintainer of the model or some programmatic user of the model should know?
% - With what other models does this model interact?

% As a starter you should refer to the API. If you have a bibtex entry
% tagged with api:\MODELNAME, you can use the following:
% The classes and methods of the \MODELTITLEx are described in detail in the
% \hyperapiref{\MODELNAME}. This section describes architecture details that
% are not present in the API document.

% If the conceptual design did not take the model to the point where the
% API alone suffices to provide the necessary understanding you should
% continue the discussion of the details of the design.

% As with the conceptual design, make sure all of your floats are printed.


% Inventory instructions: Tailoring instructions are embedded in the
% itemized list.
\section{Inventory}
All \MODELTITLEx files are located in {\tt \$\{JEOD\_HOME\}/\MODELPATH}.
Relative to this directory,
\begin{itemize}
\vspace{-0.2\baselineskip}
%
% Header and source files.
% Tailoring is needed for
% - Models with headers only
% - Models with headers and source in model subdirectories.
\item Header and source files are located
in model {\tt include} and {\tt src} subdirectories.
See table~\ref{tab:source_files}
for a listing of the configuration-managed files in these directories.
%
% Data files and other stuff.
% This item prints only if your model has data files and other stuff. Do not
% delete this item if your model doesn't have any data or miscellaneous files.
% Nothing will be printed in such a case.
% Tailoring is needed for
% - Models that have other stuff such as a makefile_overrides file.
\iflabeldefined{tab:data_files}{%
\vspace{-0.1\baselineskip}
\item Data files are located in the model data subdirectory.
See table~\ref{tab:data_files}
for a listing of the configuration-managed files in this directory.
}%
%
% Documentation files.
% This item should be the same for all models.
%
\vspace{-0.1\baselineskip}
\item Documentation files are located in the model {\tt docs} subdirectory.
See table~\ref{tab:documentation_files}
for a listing of the configuration-managed files in this directory.
%
% Verification files.
% This item should be the same for all models.
% This item prints only if your model has verification files. Do not
% delete this item if your model doesn't have a verif directory.
% Nothing will be printed in such a case.
%
\iflabeldefined{tab:verification_files}{%
\vspace{-0.1\baselineskip}
\item Verification files are located in the model {\tt verif} subdirectory.
See table~\ref{tab:verification_files}
for a listing of the configuration-managed files in this directory.
}%
\end{itemize}

\input{inventory}
