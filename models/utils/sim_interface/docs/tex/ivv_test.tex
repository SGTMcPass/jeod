\test{Simulation Interface Simulation}
\label{test:local_verif}
\begin{description}
\item[Background]
This test is an extremely simple simulation which
creates a Trick sim\_object containing 
an instance of a JeodTrickSimInterface to be tested.
One other Trick sim\_object creates a test object
with a single scheduled job.  The 
test object calls JeodTrickSimInterface::get\_job\_cycle() as
a scheduled job, thus the output can be compared
to the cycle time of the scheduled job.

The test object also includes private fields which are initialized by the 
Trick input processor and are logged by Trick.

\item[Test directory] {\tt verif} \\
This is a standard verification directory containing
the simulation directory  {\tt SIM\_sim\_interface}
along with src and include directories for the test class code.

\item[Success criteria]
The simulation includes initialization and logging of private fields 
which also reflect the output of 
JeodTrickSimInterface::get\_job\_cycle(). The logged data should be
identical to the 
reference data in the SET\_test\_val directory.

\item[Test results]
Passed.

\item[Applicable requirements]
This test demonstrates the satisfaction of
requirements \traceref{reqt:hidden_data_visibility},
\traceref{reqt:sim_engine_interface},
and \traceref{reqt:job_cycle}.
\end{description}


\test{Container Simulation}
\label{test:container_model_sim}
\begin{description}
\item[Background]
This simulation, located in the \CONTAINER\ verification directory,
exercises the checkpoint/restart capabilities of the model.
For a complete description of this test, see the
\hypermodelrefinside{CONTAINER}{part}{ivv} for details.
\item[Test Directory]
\verb|models/utils/container/verif/SIM_container_T10|
\item[Test Results]
Passed.
\item[Applicable Requirements]
This test demonstrates the satisfaction of
requirements \traceref{reqt:allocated_data_visibility},
\traceref{reqt:sim_engine_interface},
\traceref{reqt:checkpoint_restart},
and \traceref{reqt:addr_name_xlate}.
\end{description}


\test{Memory Simulation}
\label{test:memory_model_sim}
\begin{description}
\item[Background]
This simulation, located in the \MEMORY\ verification directory,
tests the ability of the \MEMORY\ to allocate and deallocate
memory. This in turn tests the ability of this model to
make those allocations visible to the simulation engine.
 For a complete description of this test, see the
\hypermodelrefinside{MEMORY}{part}{ivv} for details.
\item[Test Directory]
\verb|models/utils/memory/verif/SIM_memory_T10|
\item[Test Results]
Passed.
\item[Applicable Requirements]
This test demonstrates the satisfaction of
requirements
\traceref{reqt:allocated_data_visibility},
\traceref{reqt:sim_engine_interface},
and \traceref{reqt:checkpoint_restart}.
\end{description}


\test{Message Handler Simulation}
\label{test:message_model_sim}
\begin{description}
\item[Background]
This simulation, located in the \MESSAGE\ verification directory,
exercises the Trick-based implementation of the abstract
class MessageHandler.
For a complete description of this test, see the
\hypermodelrefinside{MESSAGE}{part}{ivv} for details.
\item[Test Directory]
\verb|models/utils/message/verif/SIM_message_handler_verif_T10|
\item[Test Results]
Passed.
\item[Applicable Requirements]
This test demonstrates the satisfaction of
requirements \traceref{reqt:sim_engine_interface}
and \traceref{reqt:trick_message_handler}.
\end{description}


\test{Propagated Planet Simulation}
\label{test:sim_prop_planet}
\begin{description}
\item[Background]
This simulation, located in the \EPHEMERIDES\ verification directory,
tests the ability of the \EPHEMERIDES\ to propagate a planet as an
alternative to using an ephemeris model.
The Trick 10 version was constructed to serve as a test case
of the multiple integration group capability provided by this model. 
 For a complete description of this test, see the
\hypermodelrefinside{EPHEMERIDES}{part}{ivv} for details.
\item[Test Directory]
\verb|models/environment/ephemerides/verif/SIM_prop_planet_T10|
\item[Test Results]
Passed.
\item[Applicable Requirements]
This test demonstrates the satisfaction of
requirements \traceref{reqt:allocated_data_visibility},
\traceref{reqt:sim_engine_interface},
\traceref{reqt:integ_interface},
\traceref{reqt:checkpoint_restart},
\traceref{reqt:addr_name_xlate},
and \traceref{reqt:multiple_integ_groups}.
\end{description}

