%%%%%%%%%%%%%%%%%%%%%%%%%%%%%%%%%%%%%%%%%%%%%%%%%%%%%%%%%%%%%%%%%%%%%%%%%%%%%%%%
% Orientation model requirements
%%%%%%%%%%%%%%%%%%%%%%%%%%%%%%%%%%%%%%%%%%%%%%%%%%%%%%%%%%%%%%%%%%%%%%%%%%%%%%%%

\chapter{Product Requirements}\hyperdef{part}{reqt}{}\label{ch:reqt}

\requirement{Project Requirements}
\label{reqt:toplevel}
\begin{description}
\item[Requirement:]\ \newline
  This model shall meet the JEOD project requirements specified in the 
  \hyperref{file:\JEODHOME/docs/JEOD.pdf}{part1}{reqt}{JEOD} top-level document.

\item[Rationale:]\ \newline
  This is a project-wide requirement.

\item[Verification:]\ \newline
  Inspection
\end{description}


\requirement{Supported Representations}
\label{reqt:representations}
\begin{description}
\item[Requirement:]\ \newline
  The \ModelDesc shall provide the ability to represent an orientation
  in any of the following forms:
  \subrequirement{}3x3 transformation matrix.
  \subrequirement{}Left unit transformation quaternion.
  \subrequirement{}Eigen rotation.
  \subrequirement{}Euler angles.

\item[Rationale:]\ \newline
  There are many different but equivalent ways of representing the orientation
  of an object. Some are more intuitive than others while some are more
  amenable to analysis. An external source may specify orientation in any
  one of these forms (and others as well).  For these reasons, JEOD must be
  able to model the orientation of an object in a number of representation
  schemes.

\item[Verification:]\ \newline
  Inspection, Test
\end{description}


\requirement{Data Access}
\label{reqt:data_access}
\begin{description}
\item[Requirement:]\ \newline
  \subrequirement{Member Access.}
  The \ModelDesc shall provide public access to its representations of
  an orientation.
  \subrequirement{Data Consistency.}
  The \ModelDesc shall provide mechanisms to ensure that the member data
  for some representation of an orientation is consistent with input data.
  \subrequirement{Programmatic Assignment.}
  The \ModelDesc shall provide functional mechanisms to initialize or update
  an orientation.
  \subrequirement{Programmatic Access.}
  The \ModelDesc shall provide functional mechanisms to access representations
  of an orientation.

\item[Rationale:]\ \newline
  Public access is provided for backward compatibility. The model would
  fail to satisfy the requirement to represent multiple representation schemes
  without the consistency requirement. The latter two sub-requirements support
  a more modern, object-oriented view.

\item[Verification:]\ \newline
  Inspection, Test
\end{description}


\requirement{Euler Angles}
\label{reqt:euler_angles}
\begin{description}
\item[Requirement:]\ \newline
  With regard to Euler angles,
  \subrequirement{Supported Sequences.}
    The \ModelDesc shall support all twelve of the Euler rotation sequences.
  \subrequirement{Sequence Identifiers.}
    The numerical values of the identifiers used in the \ModelDesc to
    identify the six aerodynamics angles shall be equal to the corresponding
    identifiers used in Trick.

\item[Rationale:]\ \newline
  JEOD 2.0 used the Trick math library Euler angle identifiers and only
  supported the six aerodynamics Euler angles supported by Trick.
  The project requirement to make JEOD 2.1 Trick-independent means
  that the identifiers must now be defined by JEOD. The requirement to use the
  same numerical values as in Trick makes these new identifiers backwards
  compatible with JEOD 2.0. The requirement to be Trick-independent also
  provided the opportunity to support all twelve Euler sequences.

\item[Verification:]\ \newline
  Inspection, Test
\end{description}


\requirement{Conversions between Representation Schemes}
\label{reqt:conversions}
\begin{description}
\item[Requirement:]\ \newline
  The \ModelDesc shall provide the following conversion schemes:
  \subrequirement{}\label{reqt:eigen_to_mat}
  Eigen rotations to transformation matrices.
  \subrequirement{}\label{reqt:mat_to_eigen}
  Transformation matrices to eigen rotations.
  \subrequirement{}\label{reqt:euler_to_mat}
  Euler angles to transformation matrices.
  \subrequirement{}\label{reqt:mat_to_euler}
  Transformation matrices to Euler angles.
  \subrequirement{}\label{reqt:euler_to_quat}
  Euler angles to quaternions.
\item[Rationale:]\ \newline
  JEOD 2.0 used Trick math library functions to perform the first four
  of the specified conversions. The project requirement to make JEOD 2.1
  Trick-independent means that these conversion methods must now be implemented
  in JEOD. The final conversion makes it possible to directly compute a
  quaternionic representation from all supported representations.
  
\item[Verification:]\ \newline
  Inspection, Test
\end{description}
