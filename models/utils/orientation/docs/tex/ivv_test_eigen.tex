\test{Eigen Rotation Test}\label{test:eigen}
\begin{description}
\item[Background]
The purpose of this test is to determine whether the two static
methods that convert eigen rotations to and from transformation matrix
are implemented correctly. The transformation from an eigen rotation to a matrix
is straight line code. This method involves no branching and has little if any numerical sensitivity.

The true challenge is to verify the correctness of the conversion from
transformation matrices to eigen rotations. The underlying algorithm splits
rotations into four main groups: Null rotations, small rotations (up to 45
degrees), intermediate rotations (45 to 135 degrees), and large rotations
(135 to 180 degrees). The test needs to ensure each case is well-covered, with
special attention to rotations very close to 0 and 180 degrees.
The algorithm shouldn't be sensitive to which of {\em x}, {\em y}, or {\em z} is
the dominant axis, but that selection is the subject of a branch. This needs to
be well-covered as well. The test achieves this coverage by overkill.

\item[Test description]
The test generates 340 unit vectors spread around the unit sphere. A set of 19
eigen rotations are performed for each unit vector. Nine of these rotations
are from 0 to 180 degrees in steps of 22.5 degrees. The remaining ten involve
rotations very close to 0 and 180 degrees.

Each individual test proceeds by
\begin{enumerate}
\item Constructing a left transformation quaternion from the eigen rotation and
then constructing a transformation from the quaternion. This provides a means
of testing the correctness of the eigen rotation to matrix computation.
\item Constructing a transformation matrix from the eigen rotation using
the Orientation model.
\item Comparing the results from steps 1 and 2. The two matrices should be
very close to one another. The threshold used is $10^{-15}$ radians, or
numerical error.
\item Computing an eigen rotation from the matrix constructed in step 1.
This should reproduce the original rotation angle and unit vector. The
combined angular error in the two should once again represent numerical
error, or $10^{-15}$ radians.
\end{enumerate}

\item[Test directory] {\tt verif/unit\_tests/eigen\_rotation} \\
This is a standard unit test directory with two configuration managed items,
main.cc and makefile. Simply type {tt make} to build the test article and run the
test in default mode. This default mode summarizes the results of the test.
To see individual output, issue the command {\tt ./test\_program -verbose} after
have made the test article.

\item[Success criteria]
As mentioned above, each of the 6460 tests must pass both the matrix and
eigen tests to within numerical error. An individual test passes only if both
subtests pass. All 340*19=6460 tests must pass for the test to pass as a whole.

\item[Test results]
The test passes.

\item[Relevant requirements]
This test demonstrates the satisfaction of requirement~\ref{reqt:conversions}
sub-requirements~\ref{reqt:eigen_to_mat} to~\ref{reqt:mat_to_eigen}
and partially demonstrates the satisfaction of
requirement~\ref{reqt:representations}.

\end{description}
