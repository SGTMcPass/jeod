\setcounter{chapter}{0}

%----------------------------------
\chapter{Introduction}\hyperdef{part}{intro}{}\label{ch:intro}
%----------------------------------


\section{Model Description}
%%% Incorporate the intro paragraph that used to begin this Chapter here.
%%% This is location of the true introduction where you explain why this 
%%% model exists.
%%% Identify the Model context within JEOD.

This documentation describes the design and testing of the routines in the JSC Engineering Orbital 
Dynamics (JEOD) \ModelDesc.  These routines 
are derived from well-known conversions between the various
attitude representations (i.e., quaterion to matrix, etc.).

Included in this documentation are verification and validation 
cases that describe tests done on the algorithms to verify that 
they are working correctly and computing the correct values for given 
input data. There is also a User Guide which describes how to incorporate the above 
mentioned routines as part of a Trick simulation.

The parent document to this model document is the
JEOD Top Level Document~\cite{dynenv:JEOD}.

%%%%%%%%%%%%%%%%%%%%%%%%%%%%%%%%%%%%%%%%%%%%%%%%%%%%%%%%%%%%%%%%%%%%%%%%%%%%%%%%
% change_history.tex
% History of orientation_manager.pdf
% Add a new line to the table for each update. 
%%%%%%%%%%%%%%%%%%%%%%%%%%%%%%%%%%%%%%%%%%%%%%%%%%%%%%%%%%%%%%%%%%%%%%%%%%%%%%%%

\section{Documentation History}
\begin{tabular}{||l|l|l|l|} \hline
{\bf Author } & {\bf Date} & {\bf Revision} & {\bf Description} \\ \hline \hline
 Blair Thompson & November, 2009 & 1.0 & Initial Version \\ \hline
 David Hammen & October, 2010 & 2.0 & JEOD 2.1 \\ \hline
\end{tabular}



\section{Document Organization}
This document is formatted in accordance with the 
NASA Software Engineering Requirements Standard~\cite{NASA:SWE} 
and is organized into the following chapters:

\begin{description}
%% longer chapter descriptions, more information.

\item[Chapter 1: Introduction] - 
This introduction contains three sections: description of model, document history, and organization.  
The first section provides the introduction to the \ModelDesc and its reason 
for existence.  It contains a brief description of the interconnections with other models, and 
references to any supporting documents. It also lists the document that is parent to this one.
The second section displays the history of this document which includes
author, date, and reason for each revision.  The final
section contains a description of the how the document is organized.

\item[Chapter 2: Product Requirements] - 
Describes requirements for the \ModelDesc.

\item[Chapter 3: Product Specification] - 
Describes the underlying theory, architecture, and design of the \ModelDesc in
detail.  It is organized in 
three sections: Conceptual Design, Mathematical Formulations, and Detailed Design.

\item[Chapter 4: User Guide] - 
Describes how to use the \ModelDesc in a Trick simulation.  It is broken into three sections to represent the JEOD 
defined user types: Analysts or users of simulations (Analysis), Integrators or developers of simulations (Integration),
and Model Extenders (Extension).

\item[Chapter 5: Verification and Validation] -  
Contains \ModelDesc verification and validation procedures and results.

\end{description}
