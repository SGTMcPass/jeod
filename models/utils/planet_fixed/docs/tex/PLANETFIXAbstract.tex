%%%%%%%%%%%%%%%%%%%%%%%%%%%%%%%%%%%%%%%%%%%%%%%%%%%%%%%%%%%%%%%%%%%%%%%%
%Planet Fixed Utility
% Purpose:
%
% 
%
%
%%%%%%%%%%%%%%%%%%%%%%%%%%%%%%%%%%%%%%%%%%%%%%%%%%%%%%%%%%%%%%%%%%%%%%%%%

\begin{abstract}
This model utility provides two coordinate transformations:
\begin{itemize}
\item A module for transformation from planet centered planet fixed Cartesian coordinates to
      planetospherical or planetoellipsoidal coordinates (a longitude, latitude, altitude set)
      and vice versa.(Note: in the model the word ellipse occurs in place of ellipsoidal coordinates.
      The reader should note that ellipsoidal is the standard usage for a given planetary body.
       See the Explanatory Supplement to the Astronomical Almanac \cite{Seidelmann}.
\item A module for transformation from planet centered planet fixed to a tangent plane system.
      This is a horizon system and here is chosen to be the North East Down specification with
      coordinates of altitude , latitude (or elliptical latitude) and longitude chosen as the non
      cartestian coordinates.(This is the reversed version of Local North East Up coordinates, where height
      (or altitude), right ascension and declination are chosen.)
\end{itemize}
The term planet\_fix refers to the two utility modules planet\_fixed\_posn and north\_east\_down.\\
These coordinates and coordinate transformation apply to any body in the solar system. The reader however
will find difference nomenclature in the literature for these as noted above.
\end{abstract}
