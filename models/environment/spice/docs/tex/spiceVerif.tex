\section{Inspection}\label{sec:inspect}
This section describes the inspections conducted on the \SpiceDesc\ to examine
its compliance with the inspection requirements levied against it.

\inspection{Top-level Inspection}
\label{inspect:TLI}
This document structure, the code, and associated files have been inspected,
and together satisfy requirement~\traceref{reqt:toplevel}.


\section{Tests}\label{sec:tests}
This section describes the tests conducted to verify and validate
that the \SpiceDesc\ satisfies the requirements levied against it.
All verification and validation test source code, simulations and procedures
are archived in the JEOD directory
{\tt models/environment/spice/verif}.\relax


\test{Ephemeris and Rotation}\label{test:spice_ephem_rot}
\begin{description}

\item[Background]\ \newline
The purpose of this test is to demonstrate the ability of the \SpiceDesc\
to represent both the ephemerides and orientation of any solar system
body.


\item[Test description]\ \newline
This test utilizes two simulations -- one that utilizes the \SpiceDesc\
for ephemeris and rotational state representation of several
solar system bodies, and one that uses the legacy JEOD De4xx ephemeris and
RNP models for analogous calculations for a subset of the same bodies. The
results of the first will be compared against the second for the bodies that
are common to both, in order to build confidence in the correct operation of
the \SpiceDesc\ even for the bodies that are not. 

The SPICE-only simulation contains the solar system bodies Sun, Earth, Moon,
Mars, Itokawa (an asteroid), and Phobos (a moon of Mars). It can provide
ephemeris data for all of them, as well as rotation data for Earth, Moon,
and Mars.

The simulation using legacy JEOD models contains the bodies Sun, Earth,
Moon, and Mars. It provides ephemeris for all of them, along with rotation
for Earth, Moon, and Mars. The legacy JEOD models are unable to provide
ephemeris or rotation information for Itokawa or Phobos, which illustrates
the primary benefit of the new \SpiceDesc: the ability to include many
more solar system bodies in a simulation.


\item[Test directories] {\tt SIM\_spice and SIM\_de4xx}


\item[Success criteria]\ \newline
A comparison of the output of the SPICE-only simulation with the legacy
simulation should show good agreement between the two for both
translational and rotational states of the bodies being compared. Due
to the nature of the data and models being employed, translation
should agree to within approximately machine precision, while rotation
will be less accurate but still agree to near machine precision.


\item[Test results]\ \newline
All output data confirmed expectations.

\item[Applicable Requirements]\ \newline
This test satisfies the requirements~\traceref{reqt:rep_ephem}
and~\traceref{reqt:rep_rnp}.

\end{description}
