The User Guide chapters for many JEOD models contain sections for simulation
users, simulation builders, and model extenders.  This chapter will primarily
describe the verification sim provided for the \SpiceDesc\ and how to obtain
SPICE and the data files to use with it.  Thus, most of the content is
intended for simulation users.  However, there is also a short section at the
end of the chapter which describes briefly how to use the \SpiceDesc\ in a
Trick and JEOD simulation more generally, which should be useful for simulation
builders. The \SpiceDesc\ was not designed to be extended, therefore, there is
no section describing extension.

The verification directory for the \SpiceDesc\ includes two simulations: the
\SpiceDesc\ verification sim based on the SPICE ephemeris toolkit, and a
classic JEOD DE4xx-based ephemerides sim. These exercise comparable
capabilities, though it will be demonstrated how to expand SIM\_spice to
provide capabilities not available from the DE4xx model. Both simulations
should basically run ``out-of-the-box'', except that the SPICE simulation
(``SIM\_spice'') requires the C implementation of SPICE library, named CSPICE,
in order to run.


\section {The CSPICE Library}
SIM\_spice requires an external library in order to access the capabilities of
SPICE. In order to run SIM\_spice, it is necessary to download and compile
\href{https://naif.jpl.nasa.gov/naif/toolkit.html}{the SPICE toolkit}, then
define the environment variable ``JEOD\_SPICE\_DIR'' to point to the directory
where it was compiled.

The default build for the SPICE toolkit creates the static library
\begin{verbatim}
${JEOD_SPICE_DIR}/lib/cspice.a.
\end{verbatim}
SIM\_spice includes a file containing make overrides (S\_overrides.mk) which
uses the environment variable \verb|$JEOD_SPICE_DIR| to locate the SPICE library.
Unfortunately, the default build process for SPICE does not prepend the ``lib''
prefix to the name of the library file cspice.a.  To fix this, either rename
the file or provide an appropriately named symbolic link. The shell commands
for each alternative are given below.
\begin{verbatim}
mv ${JEOD_SPICE_DIR}/lib/cspice.a ${JEOD_SPICE_DIR}/lib/libcspice.a

ln -s ${JEOD_SPICE_DIR}/lib/cspice.a ${JEOD_SPICE_DIR}/lib/libcspice.a
\end{verbatim}


\section{Obtaining SPICE kernels}
This section describes the process used to obtain SPICE data files such as
the ones for Sun, Earth, Moon, and Mars used in SIM\_spice.  Two major sources
of SPICE data will be discussed: NAIF and Horizons.


\subsection {NASA's Navigation and Ancillary Information Facility (NAIF)}
\label{subsec:naif}
Data files which are used by the SPICE toolkit are called ``kernels''.  To
obtain many pre-made kernels including those for the Sun and planets, go to
the~\href{http://naif.jpl.nasa.gov/naif}{NAIF homepage}. Follow the links to
Data $\to$ Generic Kernels $\to$ Generic Kernels.  The resulting page should
show a list of directories and a file named ``aareadme.txt''.  At this point
it would be a good idea to read the file, which contains a concise explanation
of the various generic spice kernels.

Kernels for planetary ephemerides which describe translational motion of
astronomical objects are referred generically in SPICE as ``spk''
(SPICE Planetary Kernel) files. There are special high-fidelity versions of
such files which are in binary format and have the extension ``.bsp''.
The spk file that used to be used in SIM\_spice for ephemerides of the Sun and
planets is named ``de421.bsp'' and can be found in the directory
spk/planets/a\_old\_versions. The DE421 ephemeris file was used in order to
provide a consistent comparison with what was available with the classic JEOD
ephemerides model, but any other file containing analogous data could have
been used. The spk file currently used in SIM\_spice is ``de440.bsp'' in order to
come into alignment with other models being used for lunar operations using DE440.

Kernel files for natural satellites of other planets can be found in the
folder spk/satellites. For example, data for Phobos and Deimos, the moons of
Mars, are contained in the file ``mar097.bsp.''  Planets with large moon
systems are broken into separate files; however, since Mars has only two
moons, there is only a single file for the Martian system. The ``readme''
files in that folder can assist in determining which files are needed for
a given object.

Besides de421.bsp, the other files needed for SIM\_spice are the
high-precision orientation models for Earth and Moon.  These can be found in
the~\href{http://naif.jpl.nasa.gov/pub/naif/generic_kernels/pck}
{``pck'' directory} on the ``Generic Kernels'' page.  The specific files
used are ``earth\_000101\_171024\_170805.bpc'' and
``moon\_pa\_de421\_1900-2050.bpc.'' These files provide the high-fidelity
orientation models for Earth and Moon. When using DE440, the files used are
 ``earth\_000101\_240604\_240312.bpc'' and ``moon\_pa\_de440\_200625.bpc.''
The directory contains other versions which may be a better match for some
situations.  The file ``aareadme.txt' contains descriptions of each file which
can aid in determining which best suits the desired application. The files used
with SIM\_spice were chosen because they provide capabilities similar to JEOD's
existing RNP model.

Note that the text kernels
``pck00010.tpc'' and ``moon\_080317.tf'' are also found in this directory,
for use with DE421. For DE440, the files used are ``pck00011.tpc'' and
``moon\_de440\_220930.tf.'' ``moon\_080317.tf '' includes definitions for
standard lunar reference frames while ``pck00010.tpc'' contains
low fidelity (IAU) orientation models for planets and natural satellites in the
solar system. The orientation model for Mars that SPICE offers is one of these
relatively low fidelity IAU models; however, it agrees reasonably well with
the JEOD model for Mars orientation.


\subsection {The Jet Propulsion Laboratory Horizons System}
\label{subsec:horizons}
While the NAIF website includes ephemerides for Sun, Planets, natural
satellites, and barycenters, as well as orientation models for planets and
satellites, it is generally necessary to obtain kernel files for objects
such as comets and asteroids from the JPL Horizons system.  There are both
web ~\href{http://ssd.jpl.nasa.gov/?horizons}{(Horizons main page)} and telnet
interfaces to Horizons; however, high-fidelity binary files are only
available through the telnet interface.  The following example describes the
process for obtaining a high-fidelity ephemeris file for the asteroid Itokawa.

First type the following from the command line:
\begin{verbatim}
telnet ssd.jpl.nasa.gov 6775
\end{verbatim}
The first task is to find the body of interest. In the case of Itokawa, this
is very easy. Just type ``Itokawa'' at the ``Horizons$>$'' prompt. Once Horizons
finds Itokawa, press Return to confirm. The screen will fill with information
about Itokawa, and at the bottom the user will be prompted for input.  Type
``s'' to request an ``spk'' file.  The next step will request an email address.
Enter it and confirm.  Then, Horizons will ask whether a text file is desired.
Since binary is desired instead, respond with ``n''.

Next, the start and stop dates for the file are needed. For instance,
SIM\_spice starts and ends on Jan. 30, 2009, so any range which includes that
day would be fine.

Finally, the system creates a binary spk file and provides instructions on
how/where to retrieve it by anonymous ftp. The name of the file is a somewhat
random string of characters; the user will likely find it helpful to rename
it something more human-friendly after retrieval, such as ``itokawa.bsp''.
Follow the instructions and retrieve the file.


\section{SIM\_spice}
Once SPICE has been downloaded and installed, SIM\_spice is ready to be run.
The simulation directory already contains the SPICE kernels needed to operate the
basic simulation which includes Sun, Earth, Moon, and Mars.  These files were
retrieved using the process described in section~\ref{subsec:naif}.

The standard RUN case delivered with SIM\_spice is named ``RUN\_01''. That run
will create files that log the position and velocity of Sun, Earth, Moon and
Mars in Solar System Barycenter coordinates. The rotational states of Earth,
Moon, and Mars are also logged in their respective files. The exact same
information is logged by SIM\_de4xx and can be used for comparison.

The generalized version of SIM\_spice includes two bodies, Phobos and Itokawa,
which are not part of the JEOD DE4xx model. In order to run the generalized sim,
one merely needs to uncomment a few lines in the input deck and download the
necessary files from NAIF and Horizons.  The appropriate file for Phobos is
named ``mar097.bsp'' and is located in the directory ``spk/satellites'' on
the NAIF Generic Kernels page; see section~\ref{subsec:naif} for further
information on retrieval.  The appropriate file for Itokawa can be retrieved
from the JPL Horizons system; the process is thoroughly described in
section~\ref{subsec:horizons}.

Once those files are retrieved, place them in the data directory of
SIM\_spice. Then open the file SET\_test/RUN\_01/input.py and uncomment
line 19. The additional bodies will now be actively updated upon subsequent
re-runs of SIM\_spice.

(Aside: If one is interested in seeing how the SPICE files are unpacked
and the resulting JEOD reference frame tree is built, then uncomment line 7 of
SET\_test/RUN\_01/input.py. Note that doing so uncorks a spew of other debug
output from JEOD as well, so it might be the kind of thing best left for
debugging purposes.)


\section{Using SPICE in a Trick Simulation}\label{sec:builder}
In order to have planets or other celestial bodies in a Trick simulation,
one should instantiate a SimObject for each object. The typical architecture
includes the following:
\begin{itemize}
\item A SimObject to manage the overall environment.
\item Planetary SimObjects, one for each celestial body, containing Planet
and GravitySource classes to represent the associated celestial body's reference
frames, characteristics, and gravity.
\item ``Default data'' classes to initialize the Planet and GravitySource objects
within each planetary SimObject with the characteristics of the associated
celestial body.
\end{itemize}


\subsection{Planetary SimObjects}\label{subsec:planet.sm}
Planetary SimObjects are the S\_define level classes which encapsulate a planet
or other celestial object. JEOD provides several predefined S\_modules of this
sort for Earth, Moon, Mars, and the Sun; they can be found in
\verb|${JEOD_HOME}/lib/jeod/JEOD_S_modules|.

In addition to the standard JEOD S\_modules for Earth, Moon, Sun, and Mars,
SIM\_spice also includes two custom S\_modules, itokawa\_basic.sm and
phobos\_basic.sm, which are used in the generalized simulation. These
S\_modules were built by inheriting from a generic planet module found
in the directory \verb|${JEOD_HOME}/lib/jeod/JEOD_S_modules/Base| which is
there for that purpose. Thus, itokawa\_basic.sm and phobos\_basic.sm
are excellent prototypes for any new S\_modules that may need to be created
for celestial objects driven by SPICE.


\subsection{Planetary Initialization Classes}\label{subsec:planetary_init}
JEOD provides default data classes to initialize Planet and GravitySource
objects for Earth, Moon, Sun, Mars, and Jupiter. These standard JEOD
initialization classes are found in the ``data'' directories for the planet
and gravity models respectively.  For the purposes of demonstrating the
capabilities of the SPICE model via SIM\_spice, analogous initialization
classes were created for Phobos and Itokawa.  However, these classes are
only for demonstration and should not be used when accurate planet or gravity
models are needed.


\subsection{Working with SPICE Ephemerides}\label{subsec:spice_ephelmerides}
Starting with JEOD version 3.3, the standard JEOD module library includes
an S\_module named environment\_spice.sm which defines an environment
SimObject for the SPICE ephemerides. Thus, including SPICE in a sim is
as simple as adding the line
\begin{verbatim}
#include "JEOD_S_modules/environment_spice.sm"
\end{verbatim}
to an S\_define file.

Unlike the classic DE4xx model, the \SpiceDesc\ has no concept of activation
or deactivation of an individual celestial object once declared.  If a
SPICE ephemeris object exists in a sim, then SPICE assumes that the object
should be active.  So, all that is necessary in order for a sim to contain
celestial bodies driven by the \SpiceDesc\ is to define their S\_modules,
and include environment\_spice.sm as above.

The rest of what is needed is supplied in the input deck. The following
examples assume one is using the standard JEOD S\_module environment\_spice.sm,
which instantiates a SimObject named ``env'' containing the data member
``spice'', which is of type ``SpiceEphemeris'' (i.e., the \SpiceDesc).

First, introduce the kernel files downloaded from NAIF, containing data
for Sun, Earth, Moon, and Mars:
\begin{verbatim}
env.spice.metakernel_filename = "data/kernels_440.tm"
\end{verbatim}
The file kernels\_440.tm is a so-called ``metakernel'' file, meaning it contains
a list of kernels for SPICE to load. This file can be used as a template for
tailoring to one's own needs. The previous DE421 configuration can be accessed
by using kernels\_421.tm instead. For the generalized sim containing Phobos and
Itokawa, please refer to the file data/more\_kernels.tm.

Next, load the planetary ephemerides (see Modified\_data/spice.py):
\begin{verbatim}
env.spice.add_planet_name("Sun")
env.spice.add_planet_name("Earth")
env.spice.add_planet_name("Moon")
env.spice.add_planet_name("Mars")
\end{verbatim}

Finally, request orientation calculations for Earth, Moon, and Mars:
\begin{verbatim}
env.spice.add_orientation("Earth")
env.spice.add_orientation("Moon")
env.spice.add_orientation("Mars")
\end{verbatim}
