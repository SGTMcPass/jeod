%%%%%%%%%%%%%%%%%%%%%%%%%%%%%%%%%%%%%%%%%%%%%%%%%%%%%%%%%%%%%%%%%%%%%%%%
%
% Purpose: Summarize the purpose of the SPICE model
%
% 
%
%%%%%%%%%%%%%%%%%%%%%%%%%%%%%%%%%%%%%%%%%%%%%%%%%%%%%%%%%%%%%%%%%%%%%%%%%

\begin{abstract}

The JEOD \SpiceDesc\ provides equivalent functionality to the existing DE4xx
ephemerides and Earth orientation (RNPJ2000) models, while expanding upon
their capabilities. As those models can, it can supply high-fidelity
ephemerides for the Sun, planets, Pluto, and the Moon, as well as
high-fidelity RNP for the Earth and Moon. However, the \SpiceDesc\ can also
provide high-fidelity ephemeris and simple orientation models for all of the
planets, natural satellites (moons), asteroids, comets, and other objects and
locations of interest in the solar system.  Thus, the \SpiceDesc\ offers
significant advantages over existing JEOD implementations of DE4xx and RNPJ2000:
\begin{itemize}
\item Compatibility -- The SPICE Toolkit, which is implemented by the \SpiceDesc,
is an agency standard supported and maintained by the Jet Propulsion
Laboratory (JPL).
\item Extended coverage -- JPL offers a vast selection of data files which can
provide custom ephemerides and orientation information for most objects and
locations of interest in the solar system.
\item Custom data -- JPL offers data files which are fitted for specific
applications where high accuracy or other special characteristics are required.
Additionally, with the proper know-how, users can create their own
SPICE-compatible data files which would then be usable with the JEOD \SpiceDesc.
\end{itemize}

Users are encouraged to convert existing simulations to use the \SpiceDesc\ as
it may prove necessary in the future to deprecate and ultimately remove the
classic JEOD implementations of DE4xx and RNPJ2000.

\end{abstract}
