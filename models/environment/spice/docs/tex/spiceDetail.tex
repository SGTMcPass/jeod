\subsection{Process Architecture}
The \SpiceDesc\ is an extension of the JEOD ephemeris model framework
and is organized into three primary classes. The first is SpiceEphemeris,
which extends the EphemerisInterface class and thus is the orchestrator
of the entire \SpiceDesc. The second is SpiceEphemPoint, which is a child
class of EphemerisPoint.  There must be a separate SpiceEphemPoint for each
celestial body for which SPICE is to update the translational state in a
simulation. The final class is SpiceEphemOrientation, which extends
EphemerisOrientation. It performs a similar function as SpiceEphemPoint,
but for the rotational state of celestial bodies; one must exist in the
sim for each body that SPICE is to keep updated with respect to rotational
state.

The usual types of methods used for such things as initializing and updating
the model exist for the \SpiceDesc, just as for most JEOD models.  Since
this is a model to interface with SPICE, much of the functionality
involves either creating connections to SPICE or obtaining data from it in
accordance with the setup implemented in the simulation.  Since the
\SpiceDesc\ fits into the existing JEOD ephemeris model framework, the
Dynamics Manager takes care of orchestrating the state updates automatically.


\subsection{Functional Design}
This section describes the functional operation of the methods of the
\SpiceDesc.

The \SpiceDesc\ contains the classes SpiceEphemeris, SpiceEphemPoint, and
SpiceEphemOrientation. While there are several methods for each class, both
new and inherited, this section will focus on only a few key SpiceEphemeris
methods. Discussion of these methods will provide good insight into the inner
workings of the \SpiceDesc.  For an exhaustive treatment of the methods for
all three classes, see the \href{file:refman.pdf}{Reference Manual}\cite{api:spice}.

The SpiceEphemeris class contains the following methods, among others:
\begin{enumerate}

\funcitem{add\_planet\_name}
This is an S\_define or input file method which is used to add a new
SpiceEphemPoint to the \SpiceDesc's list of ones to keep updated.

\funcitem{add\_orientation}
This is an S\_define or input file method which is used to add a new
SpiceEphemOrientation to the \SpiceDesc's list of ones to keep updated.

\funcitem{initialize\_model}
This is an S\_define-level method, which sets the \SpiceDesc\ up
properly.  It has several subroutines and performs critical tasks such as
loading the given SPICE kernels, finding the desired celestial bodies and
orientations within them, figuring out what barycenters need to exist in
the simulation based upon the points and orientations loaded, creating
said barycenters, and making connections to the appropriate ephemeris
point and orientation objects in the SPICE kernels.

\funcitem{ephem\_build\_tree}
This method is called automatically by the Dynamics Manager whenever it
is determined that the reference frame tree needs to be rebuilt.

\funcitem{ephem\_update}
This method is called automatically by the Dynamics Manager and is the
routine that actually performs the state updates for the loaded
SpiceEphemPoints and SpiceEphemOrientations.

\end{enumerate}

