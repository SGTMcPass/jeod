In order for a given reference frame $B$ (which need not initially be known
to JEOD) to properly mesh with the JEOD reference frame system, it is necessary
to define the following quantities:
\begin{itemize}
\item The parent frame, denoted generically as $A$. This needs to be
a frame that is already part of the JEOD reference frame tree and which is
also known to SPICE. 
\item The position and velocity of the origin of frame $B$ expressed in $A$.
\item T\_parent\_this. This is the $3 \times 3$ transformation from $A$ to
$B$.
\item ang\_vel\_this. This is the angular velocity of $B$ with respect to
$A$, expressed in $B$.
\end{itemize}


\subsection{Nomenclature}
Subscripts will indicate the reference frame to which a quantity is relative
and the coordinate system in which it is expressed. For example,
$\vec x_{B|A:A}$ represents a vector position of frame $B$ with
respect to frame $A$ expressed in frame $A$.


\subsection{Relevant Functions from CSPICE} \label{subsec:functions}
The SPICE kernels and functions provide means of determining the 
information necessary to define $B$ as a JEOD reference frame. This is
accomplished by interfacing with the C-language implementation of SPICE,
known as CSPICE.

\paragraph{Translation}
In order to obtain $\vec x_{B|A:A}$ and $\dot{\vec{x}}_{B|A:A}$,
the position and velocity of the origin of $B$ expressed in $A$, JEOD employs
the CSPICE function spkez\_c. This function provides the position and velocity
of a ``target'' with respect to an ``observer'' in a specified coordinate system
at a given time.  For JEOD, the coordinate system to use is J2000, since JEOD
tracks the origins of all celestial bodies in J2000. The values
$\vec x_{B|A:A}$ and $\dot{\vec{x}}_{B|A:A}$ are returned by
the function in a single $6$-array from which JEOD will then need to extract
them.  The call to spkez\_c by JEOD is:
\begin{verbatim}
spkez_c (int Frame_B_ID,
         double ephemeris_time,
         "J2000",
         "none",
         int Frame_A_ID,
         double state[6],
         double *lt);
\end{verbatim}
The arguments for the function call are explained as follows:
\begin{itemize}
\item The Frame\_x\_ID arguments are the SPICE-defined integer codes for $A$
and $B$. For a description of these codes, see section~\ref{subsec:codes}.
\item The ``ephemeris\_time'' parameter is the time at which the relative state
between $A$ and $B$ is desired, in the Barycentric Dynamic Time (TDB) time scale.
\item Parameter ``J2000'' identifies the coordinate system in which position and
velocity should be expressed.
\item The ``none'' specifies that no aberration correction is to be applied, which is
consistent with the geometric interpretation of the ephemeris understood by JEOD.
\item Parameter ``state'' is the previously mentioned $6$-array from which the
position and velocity will be extracted following the function call.
\item The last parameter is unused by JEOD and thus is populated by a dummy
variable. It is the light travel time.
\end{itemize}
Upon completion of the call to spkez\_c, JEOD then extracts the desired quantities
$\vec{x}_{B|A:A}$ and $\dot{\vec{x}}_{B|A:A}$ from ``state'' by:
\begin{eqnarray}
\label{eq:state}
\vec{x}_{B|A:A} & = & \mbox{state[$i$] for }i = 0 \ldots 2 \\ \nonumber
\dot{\vec{x}}_{B|A:A} & = & \mbox{state[$i$] for }i = 3 \ldots 5
\end{eqnarray}

\paragraph{Orientation and Rotation}
The C implementation of SPICE (CSPICE) includes  the function sxform\_c which
retrieves a $6 \times 6$ matrix mapping position and velocity expressed in a
given frame to its representation in a different frame.  This matrix accounts
for orientation and relative rotation of the frames. (In the context of
ephemeris, this matrix is the mapping between the standard JEOD
pseudo-inertial frame (JEOD\_Planet\_Name.inertial)
centered on a celestial body and its planet-fixed counterpart.)  The sxform\_c
function assumes that both frames share a common origin. The relation
between the frames can be written as:
\begin{equation}
\left [\begin{array}{c}\vec x_B \\ \dot{\vec{x}}_B \end{array}\right ] =
M_{B|A} \left [\begin{array}{c} \vec x_A \\ \dot{\vec{x}}_A \end{array}
\right ]
\label{eq:m6}
\end{equation}
\noindent where
$\left [\begin{array}{c}\vec x_A \\ \dot{\vec{x}}_B \end{array}\right ]$
and
$\left [\begin{array}{c}\vec x_B \\ \dot{\vec{x}}_B \end{array}\right ]$
are position and velocity of a point expressed in frame $A$ and $B$
respectively.

The matrix $M_{B|A}$ can be partitioned into four $3 \times 3$
sub-matrices to facilitate extracting the orientation and angular velocity
between the standard pseudo-inertial and planet-fixed frames of the celestial
body of interest:
\begin{equation}
M_{B|A} = \left [\begin{array}{cc}
T_{B|A} & 0 \\
-\Omega_{B|A:B}T_{B|A} & T_{B|A}
\end{array} \right ]
\label{eq:m33}
\end{equation}
\noindent where $T_{B|A}$ is the $3 \times 3$ transformation from
frame $A$ to frame $B$ (T\_parent\_this in JEOD, because in the JEOD
reference frame tree the planet-fixed frame for a body is child to the
standard pseudo-inertial frame for the same body), and $\Omega_{B|A:B}$
is the skew-symmetric matrix satisfying
\begin{equation}
\Omega_{B|A:B} \vec{x} = \vec{\omega}_{B|A:B} \times \vec{x}
\label{eq:omega_cross_r}
\end{equation}
\noindent where $\vec{\omega}_{B|A:B}$ is the vector angular velocity of $B$ with
respect to $A$ expressed in $B$ (ang\_vel\_this in JEOD).
From ~\ref{eq:m33} it follows that
\begin{equation}
T_{i, j B|A} = M_{i,j} \mbox{ for }i, j = 0\ldots2
\label{eq:T}
\end{equation}
\noindent and
\begin{equation}
\Omega_{B|A:B} = -M_{i,j}T^t_{B|A} \mbox{ for }i = 3\ldots5, j = 0\ldots2
\label{eq:Omega}
\end{equation}
Finally, $\vec{\omega}_{B|A:B}$ is realized as
\begin{equation}
\vec{\omega}_{B|A:B} = \left [\begin{array}{c}
-\Omega_{1,2 B|A:B} \\
\Omega_{0,2 B|A:B} \\
-\Omega_{0,1 B|A:B} \end{array} \right ]
\label{eq:VecOmega}
\end{equation}

The call to sxform\_c by JEOD is:
\begin{verbatim}
sxform_c (char * from,
          char * to,
          double ephemeris_time,
          double trans[6][6]);
\end{verbatim}
The parameter ``from'' should be set to ``J2000'' in order to obtain the
relation between the standard pseudo-inertial frame centered at the
origin of the body of interest and the associated planet-fixed frame.  This is
because in JEOD all standard pseudo-inertial frames are J2000-oriented.  Next,
``to'' is the name of $B$, the planet-fixed frame, which JEOD constructs
consistent with SPICE nomenclature.  The time parameter ``ephemeris\_time'' is
the time at which the rotation state between $A$ and $B$ is desired in the TDB
time scale. Finally, ``trans'' is the previously discussed $6 \times 6$
matrix $M_{B|A}$.


\subsection {Integration with the JEOD Reference Frame System}
Following the calls to spkez\_c and sxform\_c described in the previous
subsections, all information needed to create a new JEOD frame representing
$B$ is now available.  The data members in the new Frame\_B correspond to the
quantities obtained in ~\ref{eq:state}, ~\ref{eq:T} and ~\ref{eq:VecOmega}
as follows:
\begin{eqnarray}
\mbox{Frame\_B.state.trans.position} & = & \vec{x}_{B|A:A} \\ \nonumber
\mbox{Frame\_B.state.trans.velocity} & = & \dot{\vec{x}}_{B|A:A} \\ \nonumber
\mbox{Frame\_B.state.rot.T\_parent\_this} & = & T_{B|A} \\ \nonumber
\mbox{Frame\_B.state.rot.ang\_vel\_this} & = & \vec{\omega}_{B|A:B}
\label{eq:FrameBDefinition}
\end{eqnarray}


\subsection {NAIF Integer ID Codes} \label{subsec:codes}
Every object in the SPICE inventory is assigned an integer identification code.
These codes follow some simple rules which are exploited in order to determine
parent-child relationships in the JEOD reference frame tree.  Specifically, the
solar system and planetary barycenters are numbered $0 \ldots 9$ in order of
distance from the sun. For purposes of this identification system,
the Pluto-Charon barycenter ID is $9$. Each planet is assigned the code $x99$
where $x = 1 \ldots 9$. The Sun is numbered $10$.  The planets Mercury and Venus
are moonless, thus there is no distinction between $1$ and $199$ or $2$ and $299$.
Codes larger than $999$ are reserved for comets and asteroids.  Planetary moons
are ordered from inner to outer, so, for example, the code for Phobos is $401$,
and the code for Deimos is $402$. The code system treats Pluto as a planet, thus
Pluto is $999$ and Charon is $901$. For full documentation,
see \href{https://naif.jpl.nasa.gov/naif/documentation.html}{NAIF Documentation}.

With this information, it is possible to write simple rules to determine the
code of the parent of any solar system object using the object's own code.
\begin{itemize}
\item $Code > 999$ (comet or asteroid): $Parent = 0$ (Solar System Barycenter).
\item $Code = 199$ or $299$ (Mercury or Venus): $Parent = 0$ (Solar System
Barycenter).
\item Otherwise, $Parent = \left\lfloor\frac{Code}{100}\right\rfloor$.
\end{itemize}
