%%%%%%%%%%%%%%%%%%%%%%%%%%%%%%%%%%%%%%%%%%%%%%%%%%%%%%%%%%%%%%%%%%%%%%%%%%%%%%%%%
%
% Purpose:  Extension part of User's Guide for the time model
%
% 
%
%%%%%%%%%%%%%%%%%%%%%%%%%%%%%%%%%%%%%%%%%%%%%%%%%%%%%%%%%%%%%%%%%%%%%%%%%%%%%%%%

 \section{Extension}\label{sec:User_Extension}

Several new files will be needed when extending the capability of the
\timeDesc\ with new Standard Times (note that JEOD allows for an
indefinite number of UDE Times, adding UDE times is as simple as adding
them to the S\_define, and initializing them).

Perhaps the easiest way to do this is to copy an existing file of
comparable purpose, and replace all instances of the old type with the
new type.  Remember to do this for both lowercase and uppercase.

Then edit the new file as needed.

\subsection{How to add a new time-type:}
\label{ref:Howtoaddanewtimetype}

An example of adding a new time type is verified at
\textref{SIM\_6\_extension}{test:sim6} in the Verification section of this
document.  The code to support this verification is found at
\textit{time/verif/SIM\_6\_extension}.

\begin{enumerate}


\item{Add it to a header file in the format:}

\begin{verbatim}
  class TimeABC : public TimeStandard 
{
  TimeABC( );
  ~TimeABC( );
}
\end{verbatim}


\item{Make a constructor for it:}

The \textit{name} is required, any other values are optional

\begin{verbatim}
  TimeABC::TimeABC()
  {
    name = "ABC";
    // optional content
    // e.g., set_epoch();
  }
\end{verbatim}

\item{Make a destructor for it}
\item{Add a new converter class(optional)} 

Presumably, if this is a new time-type, a new converter will be needed. 
See the next section on how to add a new TimeConverter Class.

\item{Add it to the S\_define in 2 places}

First, declare it
\begin{verbatim}
  class YourTimeSimObject : public Trick::SimObject
  {
    
    ...

    TimeABC  abc;
    
    ...

    YourTimeSimObject()
    {
       ...
       P_TIME ("initialization") time_manager.register_time( abc );
       ...
    }
  };
\end{verbatim}

Second, register it.

\begin{verbatim}
  P_TIME (initialization) environment/time:     time.manager.register_time( 
           In        JeodBaseTime  & time_reg = time.abc);
\end{verbatim}

\item{Update input file (optional)}
Add values to the input file to describe the location in the
initialization and update tree

\end{enumerate}



\subsection{How to Add a New TimeConverter Function}\label{ref:howtoaddanewconverterfunction}
In some cases, the standard release of \JEODid\ is equipped with one-way
converters, but the functionality was not provided for both directions.
 For example, it is possible to convert UT1 to GMST, but not vice
versa.  If there is need for the reverse functionality, that function
can easily be added by editing two files.




\subsubsection{Change the Constructor}
In the appropriate TimeConverter\_xxx\_yyy function, there are two areas
to change.  The first is in the constructor, where a pair of
functionality flags will likely be set to False.  Let us assume that
the \textit{a\_to\_b} converter is defined, and the \textit{b\_to\_a}
converter is only available at runtime.  Then the following flags should be set:

\begin{verbatim}
  TimeConverter_XXX_YYY::TimeConverter_XXX_YYY()
  {
    valid_directions = A_TO_B | B_TO_A_UPDATE;
    // other code
  }
\end{verbatim}

\subsubsection{Add the function}
Further down in this file, you will find the code for the defined
converter function
(\textit{TimeConverter\_XXX\_YYY::convert\_a\_to\_b()}), 

and for the reverse function
(\textit{TimeConverter\_XXX\_YYY::convert\_b\_to\_a()}).




\subsubsection{Declare the function}
The default (inherited from \textit{TimeConverter}) version of an
undefined converter function is intended to cause execution to stop
immediately.  The appropriate header file must be changed to declare
that this new function exists, and prevent inheritance of the
termination version.

In file \textit{include/time\_converter\_xxx\_yyy.hh, }add the following
line:

\begin{verbatim}
 // convert_b_to_a: Apply the converter in the reverse direction
    void convert_b_to_a(void);
\end{verbatim}

The function is now ready to go; to use it, add it to the definition of
the appropriate tree in the input file (see Analysis section of this
User Guide).







\subsection{How to Add a TimeConverter Class}
These steps are for users intending to add completely new converter
functionality

\subsubsection{Write a header file for the new class}
In the format

\begin{verbatim}
class TimeConverter_XYZ_ABC : public TimeConverter {
 
   JEOD_MAKE_SIM_INTERFACES(TimeConverter_TAI_GPS)
 
// Member Data
private: 
 TimeXYZ * xyz_ptr; /* ** 
  Converter parent time, always a TimeXYZ for this converter. */
 TimeABC * abc_ptr; /* ** 
  Converter parent time, always a TimeABC for this converter. */

// Member functions:
public: 
// Constructor
  TimeConverter_TAI_GPS();
// Destructor
  ~TimeConverter_TAI_GPS();

private:
// Initialize the converter
  void initialize( JeodBaseTime * parent, JeodBaseTime * child, const int direction);

// convert_a_to_b: Apply the converter in the forward direction
  void convert_a_to_b(void);  // optional

// convert_b_to_a: Apply the converter in the reverse direction
  void convert_b_to_a(void);  // optional
}
\end{verbatim} 

Either or both of the converter functions may be included.  Having
neither is not illegal, but is redundant (the converter has no
functionality).  Having both is also usually redundant, unless one is
intended for use during initialization and the other during updates.

\subsubsection[\ Write the source code:]{ Write the source code:}
The \textit{initializer} function must be defined, along with
\textit{convert\_a\_to\_b( ) }and \textit{convert\_b\_to\_a( ) } if
included in the header file.  




\paragraph{Initializer}
The \textit{initializer} sets the pointers and initializes any data
values required in the conversion process.  In the case that the
conversion process is itself a function of time, it must be initialized
with respect to some time.  This function is called as needed during
the initialization process: when the time-types are being initialized,
or when the update tree is being generated.  If a converter is needed
for either of those tasks, this function is called.  It can be assumed
that the \textit{parent }type has been initialized already, and that
the time is known in that representation.  It should be assumed that
the \textit{child} type has not.  Because either type (\textit{a }or
\textit{b, }i.e.\textit{ xyz }or\textit{ abc})\textit{ }can be known a
priori, there are two {\textquotedblleft}directions{\textquotedblright}
available.

A simple example may look like:

\begin{verbatim}
void TimeConverter_XYZ_ABC::initializer( // Return: -- Void
   JeodBaseTime * const parent_ptr, 
   JeodBaseTime * const child_ptr,  
   const int int_dir)  
{ 
 verify_setup(parent_ptr,child_ptr,int_dir);

 if (int_dir == 1) { 
   xyz_ptr = dynamic_cast<TimeXYZ *> (parent_ptr);
   abc_ptr = dynamic_cast<TimeABC *> (child_ptr);
 }

 else if (int_dir == -1) {
   xyz_ptr = dynamic_cast<TimeXYZ *> (child_ptr);
   abc_ptr = dynamic_cast<TimeABC *> (parent_ptr);
 }
 a_to_b_offset = some_value_or_function;
 initialized = true;  
 return; 
}
\end{verbatim}

\paragraph{Converter functions}
The converter functions are described in more detail in the previous
section, \textref{How to Add a New Converter Function}{ref:howtoaddanewconverterfunction}.

\paragraph{Constructor}
The master class, \textit{TimeConverter}, sets \textit{a\_to\_b\_offset = 0.0},
and \textit{initialized = false. } These values are inherited by
default; \textit{initialized} will be reset to t\textit{rue} in the
initialization function, and \textit{a\_to\_b\_offset} may be
calculated in the initialization function if necessary.  The
functionality flags, \textit{a\_to\_b\_initialization, 
a\_to\_b\_runtime, b\_to\_a\_initialization,} and
\textit{b\_to\_a\_runtime} should be set in the specific class
constructor.  The two time-type pointers should be initialized to NULL
(they are reset in the initializer function).  The identification names
of the two time types between which this converter operates must be \textbf{identical to} the names of those time-types as
they appear in their respective classes.

\begin{verbatim}
  a_name = "XYZ";
  b_name = "ABC";
\end{verbatim}

\paragraph[Destructor]{Destructor}
Releasing the memory allocated in the name definition is no longer necessary in JEOD 4.0, as the a\_name and b\_name members have converted to STL strings.

\subsubsection[Add it to the S\_define in 2 places]{\rmfamily\bfseries Add
it to the S\_define in 2 places}
\ First, declare it. Second Register it.

\begin{verbatim}
  class YourTimeSimObject : public Trick::SimObject
  {
    
    ...

    TimeConverter_XYZ_ABC xyz_abc;
    
    ...

    YourTimeSimObject()
    {
       ...

       P_TIME ("initialization") manager.register_converter( xyz_abc );
      
       ...
    }
  };
\end{verbatim}

