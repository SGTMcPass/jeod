%%%%%%%%%%%%%%%%%%%%%%%%%%%%%%%%%%%%%%%%%%%%%%%%%%%%%%%%%%%%%%%%%%%%%%%%%%%%%%%%%
%
% Purpose:  Conceptual part of Product Spec for the time model
%
% 
%
%%%%%%%%%%%%%%%%%%%%%%%%%%%%%%%%%%%%%%%%%%%%%%%%%%%%%%%%%%%%%%%%%%%%%%%%%%%%%%%%


%\section{Conceptual Design}


% This file was converted to LaTeX by Writer2LaTeX ver. 1.0 beta3
% see http://writer2latex.sourceforge.net for more info

This section describes the key concepts found in the \timeDesc.  For an
architectural overview, see the 
\href{file:refman.pdf} {\em Reference Manual} \cite{timebib:ReferenceManual}.

\subsection{Overview}

The {\textquotedblleft}second{\textquotedblright} is a well-defined
quantity, counted to high precision by atomic clocks, and is
incorporated into many physical constants, even into the definition of
length.  In a Newtonian universe, all atomic clocks would tick at the
same rate; the fundamental unit of time may as well be the Earth-based
atomic time (TAI) as any other atomic-time.  In the vast majority of
applications of JEOD (near-Earth simulations with millisecond
precision), using TAI as the fundamental second is highly recommended,
and 1 second of simulation time should equal 1 second of TAI.  In rare
exceptions, it may be desirable to use a more heliocentric central
base-time (e.g. Barycentric Dynamic Time (TDB) or Ephemeris Time (not
included in this release)), and some local time to track each vehicle
independently.



In \JEODid, there are many different clocks available, each of which has
the following commonalities:




\begin{itemize}
\item A character-string name
\item An initial value (value at sim-start)
\item Identification (by name) of the time-type from which it is to be
initialized
\item Identification (by name) of the time-type from which it is to be
updated
\item A pointer to the converter function that will be used to update it
\item A pointer to the Time Manager (\textit{TimeManager})
\item A pointer to the Time Manager Initializer.
(\textit{TimeManagerInit})
\end{itemize}



\subsection{Simulator Time}
In previous versions of JEOD, the simulator time (in a Trick simulation,
this is \textit{sys.exec.out.time}) was used as the basis for all
dynamical processes.  Everything progressed at the rate at which
simulator time advanced.  This is highly inflexible, and the \JEODid\
model has moved away from using the simulator time for anything more
than an incremental counter.  Simulator Time is NOT a part of the Time
model.

\subsection{Dynamic Time}
To enhance flexibility, we have added an intermediary time to manipulate
the simulation time to a clock appropriate for integrating the dynamics
of the simulation.  While simulation time always advances forward at a
constant {\textquotedblleft}rate{\textquotedblright}, Dynamic Time
(class:\textit{TimeDyn}, object ID: \textit{time.dyn}) can be made to
speed up or slow down (thereby adjusting the temporal resolution
without adjusting the calling frequencies), or even to run in reverse. 
TimeDyn can, in principle, be linked to any clock (e.g. TAI, UTC, UT1)
with an appropriate converter, and thereby define that clock as the
fundamental second.  Because TAI is based on the SI second at
Earth's gravitational potential, it is highly
recommended that TAI be used, unless the user fully understands the
implications of switching to a different fundamental clock.




This is really important, so it is worth repeating: it is this Dynamic
Time, not the simulator time, that is to be used by the integrators. 
Simulator time has no physical significance whatsoever, it is purely
used as a {\textquotedblleft}placeholder{\textquotedblright} to monitor
the computational progress of the simulation.




There are some other important issues associated with the switch from a
\textit{sim-time} based simulation to a \textit{dyn-time} based
simulation, due to the separation between simulation management and
dynamic integration:


\begin{itemize}
\item Time resolution can now be adjusted trivially, while calls to
functions are maintained at a constant ratio.  For example, suppose
that Function \textit{X} is called every 3 seconds, and function
\textit{Y} called 3 times for every call to \textit{X }(i.e. every
second), and the user wishes higher resolution at a particular part of
the simulation.  It is sufficient to adjust the rate at which
\textit{dyn-time }advances. Adjusting by a factor of 10 would yield
calls to function \textit{X} every 0.3 seconds, and to function
\textit{Y} every 0.1 seconds.  There is no need to adjust the job-cycle
in the actual simulation.
\item Previous time-resolution limitations associated with simulation
management are no longer valid.  Increased resolution can be achieved
by setting \textit{dyn-time} to run at a small fraction of the rate of
\textit{sim-time, }thereby increasing the available resolution without
bound.
\item A consequence of this feature is that the calling rate defined in
the simulation will only be true if \textit{dyn-time} and
\textit{sim-time} are running at the same rate.  Again, setting the
calling rate to be every 3 units in the simulation will be every 0.3
seconds if \textit{dyn-time} runs at one-tenth the rate of
\textit{sim-time}.
\item Adjusting the time-resolution away from a 1-1 ratio can also have
unintended consequences in the operation of the simulation,
particularly if the simulation is interfacing with some other
application.  Adjusting the time-resolution could affect the rate at
which data is exchanged, and only one side of the interface would be
aware of the change.



\end{itemize}



Dynamic Time is automatically incorporated into every simulation as an
inherent part of the Time Manager.  Dynamic Time always starts the
simulation at 0.0.




From\textit{ dyn\_time, }the other times can be derived, although the
default converters that come with \JEODid mostly require that TAI be
derived from \textit{dyn-time} and everything else from TAI.  All of
the other times are therefore referred to as derived-times, which is
then subdivided into Standard Times, and User-Defined-Epoch Times.




\subsection{Derived Time}
Depending on the simulation, it may or may not be necessary for
additional time representations.  If the purpose of a simulation is to
observe some arbitrary dynamic event with no reference to any external
configuration, then no additional times are needed.  As soon as
time-varying environmental factors are introduced, or the user wishes
to use a clock that starts at some value other than 0.0, it becomes
necessary to add additional capabilities.  To convert between time
representations, Time Converters are then needed.  To manage all of the
time representations into a coherent system, a Time Manager is also
provided.

All additional times are derived from Dynamic Time (either directly, or
via other derived times).  

There are two fundamentally distinct types of these derived times -- one
that is well defined outside the simulation, and one that is not.  The
first is referred to as a Standard Time, the second as a
User-Defined-Epoch Time (UDE).  All derived times fall into one of
these two categories, the concept of derived time becomes incorporated
into both of these categories, and disappears as a stand-alone concept.

\subsection[Standard Time (STD)]{Standard Time (STD)}
The concept of a Standard Time is that it is commonly accepted, with no
ambiguity.  Picking a particular value on a particular clock (e.g. noon
on January 23, 2009, UTC) is well understood without additional
context.  Consequently, there can be only one clock running in a
simulation for each of these times (e.g. 2 clocks running UTC would
have to read exactly the same value, and would be entirely redundant).

The specific clocks all inherit from the base-class TimeStandard.  The 
commonality between the various subclasses is extensive, therefore TimeStandard
provides the vast majority of the structure and methods for those subclasses.
Nevertheless, there is no known application that could utilize an
instance of a TimeStandard (extension to add other clocks should be carried out
by \textit{inheriting} from TimeStandard, not by instantiating a TimeStandard).  
Consequently, the ability to instantiate a TimeStandard was deprecated in JEOD 
version 2.1, and removed in JEOD version 2.2.

\subsection{User-Defined-Epoch Time (UDE)}
The main distinction between a Standard Time and a User-Defined Time is
that User-Defined Times are ambiguous without additional context.  A
commonly used example of a UDE (User Defined Epoch) is Mission Elapsed
Time.  Simply stating that a simulation started 2 hours after the
mission started is not a very meaningful statement, unless we also have
information on when the mission started.  The \textit{epoch} of UDEs
(i.e. when their value was zero) must be defined before they make any
sense at all; that epoch must ultimately be anchored with a Standard
Time, or with the Dynamic Time.  A secondary difference arises from the
ambiguity of a UDE time; while Standard Times are restricted to one
instance per time-type, the UDEs have no such restriction.  A dozen or
more different clocks could be running simultaneously, representing
different time zones (ticking with UTC), or different mission-elapsed
times for simulations involving multiple vehicles.

\subsection{Epoch Definitions}
An epoch is an instant in time; for our purposes, it is a particular
time when the values of two or more clocks are known simultaneously. 
The most commonly used epoch in modern analysis is J2000, corresponding
to noon TT on January 1, 2000.  This is the default epoch used for all
Standard Times (except GPS, which has its own well-defined epoch), although users are free to define their own. 
Indeed, users must define their own epoch for UDE times (as the name suggests), 
since these have no fixed epoch.

J2000 is defined with respect to Terrestrial Time; the value of other clocks is known at J2000, but it is not the same value as for Terrestrial Time.  To tie clock values together, we use Truncated Julian Date.  Truncated Julian
Date is based on Julian Date, but with a more recent epoch; we use the
NASA definition of Truncated Julian Date with a epoch of midnight on
May 23/24, 1968.  This value is specific to the clock under consideration (so UTC TJT=0 occurred at 00:00:00 May 24, 1968 UTC, which is a different instant in time from TT TJT = 0, or TAI TJT = 0). 

It is important to differentiate between the two epoch concepts.  The
former is used to zero a time representation, while the latter is used
to provide continuity between types.  JEOD provides two variables --
\textit{seconds} and \textit{days} --  associated with the former
interpretation for each time representation; these measure the elapsed
time since that epoch.  Each Standard Time (except GMST) also carries a variable
\textit{trunc\_julian\_time}, and a constant \textit{tjt\_at\_epoch};
these are associated with the latter interpretation.




\subsection[\ Time{}-keeping]{ Time-keeping}
A decimal counter-type clock was chosen as the basic time-keeping method
in each time representation because it is more computationally
efficient than conducting the constant verification for passing through
boundaries that is inherent to a calendar-type clock.  For all times,
seconds-since-epoch, and days-since-epoch are both maintained.  For
most Standard Times, the epoch chosen is J2000 (12:00 noon TT on
January 1, 2000).  Additionally, for Standard Times, the NASA standard
Truncated Julian Time is also maintained.  The Truncated Julian
representation was chosen for its improved accuracy over the Julian or
Modified Julian representations, and for its versatility in comparing
current values of standard clocks, and for converting to calendar-type
representations.  The following example demonstrates the distinction
between maintaining Truncated Julian Time and seconds since epoch
(J2000).

\begin{quotation}
Terrestrial Time (TT) ticks in lockstep with atomic time (TAI), but they
are offset by a constant value.  The J2000 epoch occurred at the same
instant for both, so the respective seconds-since-epoch values will be
the same for both since they tick in lockstep.  Conversely, the
Truncated Julian Time {\textquotedblleft}epochs{\textquotedblright} are
clock-dependent, so they occurred at different instants.  Since TT and
TAI are offset from each other, their respective Truncated Julian Times
values will always differ by a constant amount and their reported times
will therefore differ, as is expected.
\end{quotation}

A calendar-based clock is also made available for most Standard Times;
maintaining this is more demanding than the simple decimal-type clocks,
so we have left this to the discretion of the user.  It is an easy task
to have the simulation routinely calculate the calendar representation,
but the default is to omit the calendar representation.  Similarly, for
the UDE Times, there is a clock capability (day::hour::min::sec) that
can be maintained if desired.




\subsection{Initializing the Simulation}
For maximum versatility, the code must be able to initialize the
simulation given a time in any convenient format referencing any
defined clock.  The resulting consequence is that there is plenty of
opportunity to \textit{incorrectly } specify the initialization time. 
Users should take care to read through the initialization instructions
found in the Analysis section of the User Guide (chapter \ref{ch:user}).




\subsection{Time Converters}
Time Converters are needed to calculate the values of derived times
(ultimately) from the Dynamic Time.

Converters between types are each in a separate class.  All converters
operate between two time-types,
{\textquotedblleft}type-a{\textquotedblright}, and
{\textquotedblleft}type-b{\textquotedblright}; their classes are named,
accordingly, TimeConverter\_AAA\_BBB.  All converters have the
following commonalities:


\begin{itemize}
\item The name of type-a
\item The name of type-b
\item Identification of the available functionality of the converter;
each converter has two directions available (a-to-b, and b-to-a), and
two potential operation times (initialization, and runtime).  Each
converter class has four flags, one for each combination. to indicate how
the converter may be used.
\item A numerical difference between type-a and type-b.  In some cases,
this is a constant, in others it is calculated, in others it is
obtained from data tables.
\end{itemize}



\subsection{Time Manager}
The Time Manager controls all of the time representations.

The initialization of the class TimeManager is handled by the class
TimeManagerInit.

The Time Manager contains a list of all the time representations that are to
be updated during the simulation.

The Time Manager first updates the Dynamic Time, based on the Simulator Time.  
Each of
the derived-time representations then update from their respective
parent time, starting from \textit{dyn-time} and propagating down the
time-update-tree as needed.




The TimeManager maintains two lists:


\begin{itemize}
\item All registered JeodBaseTime-derived time instances.

\begin{itemize}
\item This list is created at initialization, and after which it remains 
constant throughout the simulation.
\item At the end of the initialization cycle, all times and their 
conversions should be identified. At this point, the list of time objects is 
sorted in the order that their update must take place.
\item Each time instance knows where it is located in that list.
\item The TimeManager has a lookup function to find the location on that
list of any time object by name.
\item The Time Manager keeps all of these types updated throughout the
simulation.
\end{itemize}
\item All registered Time Converters.

\end{itemize}



\subsection{Time Update Tree}
This is a tree built at initialization that is stored in the TimeManager
class.  Coming from TimeDyn, the tree branches to any time classes
scheduled to be updated directly from DynTime.  Then from each of those
to the next layer down, etc.  For each branch, a Time Converter is
needed.  If the tree cannot be built at initialization (due to
insufficient Time Converters), the simulation will fail immediately. 
The hierarchical list of time types maintained by the TimeManager is
derived from this tree, then knowledge of the tree is discarded.




\subsection{Time Manager Initialization}
The TimeManagerInitializer performs the following tasks:


\begin{itemize}
\item Creates a tree-like structure for initialization and updating
hierarchy

\begin{itemize}
\item Includes auto-finding paths, and user-entered override
capabilities
\item Ensures that there are no circular dependencies or orphans
\end{itemize}
\item Identifies which converter functions are required for navigating
the tree of time-types
\item Initializes all of the time representations.
\end{itemize}

