%----------------------------------
\chapter{Product Requirements}\hyperdef{part}{reqt}{}\label{ch:reqt}
%----------------------------------

\section{Project Requirements}
\requirement{Top-level Requirement}
\label{reqt:toplevel}
\begin{description:}
\item[Requirement]
  The \ModelDesc shall meet the JEOD project requirements specified in
  the \JEODid\
  \hyperref{file:\JEODHOME/docs/JEOD.pdf}{part1}{reqt}{ top-level
  document}.
\item[Rationale]
  This model shall, at a minimum,  meet all external and
  internal requirements 
  applied to the \JEODid\ release.
\item[Verification]
  Inspection 
\end{description:}

\requirement{Ephemeris Items}
\label{reqt:ephem_items}
\begin{description:}
\item[Requirement]
  The \ModelDesc shall provide the ability to represent and operate on
  the ephemeris items described by some ephemeris model.
  \subrequirement{Basic capabilities.}
  \label{reqt::ephem_item_base}
  Each ephemeris item shall have a settable and gettable name and timestamp
  and shall be capable of being enabled/disabled and activated/deactivated.
  \subrequirement{Position and velocity of a point in space.}
  \label{reqt::ephem_point}
  An ephemeris point shall extend the ephemeris item capabilities by
  providing the ability to represent and update the translational state
  (position and velocity) of some point in space associated with some
  inertial frame of reference.
  \subrequirement{Position and velocity of a planet.}
  \label{reqt::ephem_planet}
  An ephemeris planet shall extend the ephemeris point capabilities
  for points in space associated with the center of mass of a planetary body.
  \subrequirement{Orientation and angular velocity of a planet.}
  \label{reqt::ephem_orientation}
  An ephemeris orientation shall extend the ephemeris item capabilities
  by providing the ability to represent and update the rotational state
  (orientation and angular velocity) of some planet-fixed reference frame.
\item[Rationale]
The JPL Developmental Ephemerides provide position and velocity of the Sun,
the planets and the Moon and also provide the orientation and angular velocity
of the Moon.
\item[Verification]
  Inspection, Test
\end{description:}


\requirement{Ephemeris Interface}
\label{reqt:ephem_interface}
\begin{description:}
\item[Requirement]
  An ephemeris model provides state information on one or more planetary
  object. A framework that characterizes in generic terms what constitutes
  a JEOD-compliant ephemeris model is needed to enable the use of
  different ephemerides in JEOD.
  \subrequirement{Base class.}
  \label{reqt:ephem_interface_class}
  The \ModelDesc shall define a framework that prescribes the behavior of
  any implementation of a JEOD-compliant ephemeris model.
  \subrequirement{Derived classes.}
  \label{reqt:ephem_interface_extensions}
  A JEOD-compliant implementation of an ephemeris model shall be
  capable of being initialized, activated, updated, and shall build the
  elements of the reference frame tree described by the ephemeris model
  consonant with the way in which JEOD calculates gravitational acceleration.
\item[Rationale]
  This requirement specifies an object-oriented scheme in which a base class
  defines but does not implement the behaviors levied on the derived classes.

  The requirement to construct the reference frame tree consonant with
  gravity calculations can be met in at least three ways:
  \begin{itemize}
  \item A flat scheme with the system barycenter at the root of the tree and all
    other ephemeris-based reference frames descending directly from this root,
  \item A hierarchical scheme with the system barycenter at the root, subsystem
    barycenters (planets plus moons) descending from the system barycenter,
    and a planet and its moons descending from the subsystem barycenter,
  \item A mixed scheme.
  \end{itemize}
  \item[Verification]
  Inspection
\end{description:}

\requirement{Ephemeris Manager}
\label{reqt:ephem_manager}
\begin{description:}
\item[Requirement]
  The JEOD reference frame manager provides a base capability for managing
  a set of reference frames that are connected in the form of a tree.
  The \ModelDesc ephemerides manager extends these base capabilities.
  \subrequirement{Registration Services.}
  The \ModelDesc shall provide registration and lookup capabilities for
  planets, ephemeris items, and integration frames.
  \subrequirement{Ephemeris Model Management.}
  The \ModelDesc shall provide the ability to register ephemeris models
  with the \ModelDesc and shall coordinate the initialization, activation,
  and updating of the registered ephemeris models.
\item[Rationale]
  The JEOD 2.0 implementation of the dynamics manager was overly complex.
  Functionality that pertained to reference frames only was moved to the
  reference frame manager sub-model while functionality that pertained to items
  related to ephemerides was moved to the ephemeris manager sub-model. 
  This is an internal requirement that reflects this architectural decision.
\item[Verification]
  Inspection, Test
\end{description:}

\requirement[Solar System]{Solar System Representation}
\label{reqt:solar_system}
\begin{description:}
\item[Requirement]
  The \ModelDesc shall provide the ability to represent:
  \subrequirement{Planetary position and velocity.}
  The model shall provide the ability to represent the position and velocity
  of the Sun, the planets, Pluto, and the Moon.
  \subrequirement{Lunar orientation and angular velocity.}
  The model shall provide the ability to represent the orientation and angular
  velocity of the Moon.
  \subrequirement{JPL Development Ephemerides.}
  The model shall use the JPL Development Ephemerides as the basis for
  these state representations.
\item[Rationale]
  These specified items are those represented in the JPL Development
  Ephemerides. The JPL DE4xx series represents the most recent of a very
  long history of planetary ephemeris models.
\item[Verification]
  Inspection, Test
\end{description:}


\requirement[DE4xx Module]{Development Ephemeris Implementation}
\label{reqt:de4xx_ephem_interface}
\begin{description:}
\item[Requirement]
  The implementation of the DE4xx ephemeris model shall satisfy the constraints
  levied in requirement~\ref{reqt:ephem_interface} on extensions to the base
  ephemeris interface capabilities.
\item[Rationale]
  This requirement is a distinguishing feature of JEOD 2.x versus JEOD 1.x.
  It changes the model from a set of routines into a set of classes
  that fits into the overall architecture.
\item[Verification]
  Inspection, Test
\end{description:}


\requirement{Propagated Planet}
\label{reqt:prop_planet}
\begin{description:}
\item[Requirement]
  The \ModelDesc shall provide the ability to:
  \subrequirement{Ephemeris Mode.}
  The model shall provide the ability to compute the position and velocity
  of a planetary body via some ephemeris model.
  \subrequirement{Propagated Mode.}
  The model shall provide the ability to compute the position and velocity
  of a planetary body by propagating the state over time per the
  gravitational forces that act on the body.
  \subrequirement{Mode Change.}
  The model shall provide the ability to dynamically switch the mechanism
  used to compute state from ephemeris mode to propagated mode.
\item[Rationale]
  This is an external requirement levied upon JEOD by its customers.
\item[Verification]
  Inspection, Test
\end{description:}
