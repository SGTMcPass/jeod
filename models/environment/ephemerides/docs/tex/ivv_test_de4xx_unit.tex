\test{DE4xx Unit Test}\label{test:de4xx_unit_test}
\begin{description}
\item[Purpose:] \ \newline
The purpose of this test is to verify that the \ModelDesc
correctly compute requested ephemerides for various solar system bodies.
\item[Requirements:] \ \newline
By passing this test, the \ModelDesc satisfies
requirements~\traceref{reqt:ephem_items},\\
\traceref{reqt:ephem_interface}, 
\traceref{reqt:solar_system},
and~\traceref{reqt:de4xx_ephem_interface}.
\item[Procedure:]\ \newline
The \ModelDesc include a test program for verifying proper routine 
function. To run the test program, go to the
\verb+verif/unit_tests/de4xx_ephem+ directory and type {\tt make}.
This will create the test article, the required binary ephemeris files,
and conduct a series of tests based on the
2004 Venus transit of the Sun (08:19:44 UTC, June 8, 2004)
and on the validation data supplied by JPL.
The test is run four times, covering the set product
(big endian, little endian)$\times$(DE405, DE421).
\item[Results:]\ \newline
The test program runs two tests for each ephemeris file, a Venus transit test
and a verification test.

The test program uses the ephemeris model to compute the position and velocity
of the Sun, Venus, and the Earth at the time of the 2004 transit of Venus.
These are compared to values obtained from the JPL HORIZONS system.
The angular separation between the position of Venus as seen from the Earth as
opposed to the position of the Sun as seen from the Earth is compared to the
known value at the time of this transit. All comparisons must pass defined
thresholds for the transit test to pass as a whole.

After the Venus transit tests, the test program then reads the appropriate
verification file ({\tt testpo.405} or {\tt testpo.421}). This file,
supplied by JPL, specifies various states for various bodies at various times.
The test program uses the model to compute the corresponding state. The
computed and provided values must agree to within a JPL-prescribed threshold
for the comparison to pass.
All comparisons must pass for this test to pass as a whole.

The results of one such test run are shown below. 
\begin{verbatim}
./test_program -model 405 -endian little
================================================================================

JEOD epehemeris model test
  Ephemeris file : JPL-DE405.little_endian.bin
  DE model       : 405
  File type      : little endian
  Machine type   : little endian
  Byteswapping   : off

-----------------Venus transit test - 08:19:44 UTC June 8, 2004-----------------

Earth-Moon barycenter position and velocity check:
  Position vector magnitude at 08:19:44 UTC June 8, 2004
    value from JPL Horizons             = 1.520e+11 m
    value from JEOD ephemeris model     = 1.520e+11 m
    relative error                      = -2.2e-13

  Velocity vector magnitude at 08:19:44 UTC June 8, 2004
    value from JPL Horizons             = 2.934e+04 m/s
    value from JEOD ephemeris model     = 2.934e+04 m/s
    relative error                      = 1.97e-13


Venus-Earth-Sun angular separation check:
  Time = 08:19:44 UTC June 8, 2004
   Expected value                      = 626.9 arc sec
   Computed value                      = 626.9 arc sec
   error                               = 0.010 arc sec

Venus transit test status: Passed

--------------------------------------------------------------------------------


-----------------------------Ephemeris dataset test-----------------------------

Relative position tests: Count of tested pairs, by planet indices
   |    1   2   3   4   5   6   7   8   9  10  11  12  13
---+-----------------------------------------------------
 1 |    -   3   5   8   4   7   8   4   7   2   1   9   6
 2 |    6   -   4   4   2   1   9   4   9   5   1   4   7
 3 |    4   5   -   2  10   2   4   5   7   6   3   5   8
 4 |    5   3   4   -   2   5   6   1   5  10   6   6   4
 5 |    5   6   5   4   -   6   3   5   5   8   5   8   5
 6 |    4   3   5   3   3   -  11   5   8   7   4   5   6
 7 |    2   5   9   5   7   0   -  10   4   3   6   7   8
 8 |    7   5   6   4   3   5   7   -   1   4   8   3   3
 9 |    2  10   4   2   1   4   5   5   -   8   6   1   4
10 |    3   2   4   4   4   4  10   3   2   -   3   2   4
11 |    5   3   3   0   3   5   5   2   5   3   -   7   4
12 |    6   6   6   2   8   2   5   4   3   3   6   -   4
13 |    2   2   4   5   7   7   2   6   5   5   4   3   -

Relative velocity tests: Count of tested pairs, by planet indices
   |    1   2   3   4   5   6   7   8   9  10  11  12  13
---+-----------------------------------------------------
 1 |    -   4   5   5   3   2   8   3   5   3   3   7   5
 2 |    4   -   6   5  10   4   7   8   3   3   6   8   5
 3 |    6   8   -   3   0   4   5   8   5   6   6   5   4
 4 |    3   5   2   -   4   4   4   7   4   4   7   2   3
 5 |    5   3   2   4   -   3   5   4   5   3   4   5   4
 6 |    6   5   8   7   2   -   8   4   3   4   4   0   4
 7 |    8   4   4   9   9   8   -   7   4   3   2   4   2
 8 |    3   5   9   4   2   6   5   -   4   3   5   4   5
 9 |    6   6   8   3   3   3   1   3   -   5   8   0   6
10 |    3   3   7   3   8   6   8   4   6   -   6   4   1
11 |    6   8   5   4   2   7   4   8   3   3   -   6   5
12 |    2   4   5   6   7   4   5   5   6   4   4   -   3
13 |    4   6   8   1   6   6   6   7   3   3   4   9   -

Planet indices:
   1 = Mercury           8 = Neptune
   2 = Venus             9 = Pluto
   3 = Earth            10 = Moon
   4 = Mars             11 = Sun
   5 = Jupiter          12 = Solar system barycenter
   6 = Saturn           13 = Earth-Moon barycenter
   7 = Uranus

Nutation, libration tests: Count of tested items
Item      | Angle  Rate
----------|------------
Nutation  |     0     0
Libration |    60    57


Ephemeris test dataset test results:
  # Passed: 1597
  # Failed: 0
  Status  : Passed

--------------------------------------------------------------------------------

JPL-DE405.little_endian.bin test summary: Passed

================================================================================
\end{verbatim}

\end{description}

\newpage
