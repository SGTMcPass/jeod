See the \href{file:refman.pdf}{Reference Manual}\cite{thermalbib:ReferenceManual} for a summary of member data and member methods for all classes.  This section describes the functional operation of the methods in each class.

Objects are presented in alphabetical order.
Methods are presented in alphabetical order within the object in
which they are located.  

\begin{enumerate}
\classitem{ThermalFacetRider} Base-class

This is the rider that attaches to a particular instance of the
generic and virtual InteractionFacet object.

{\bf Methods:}

\begin{enumerate}

\funcitem{accumulate\_thermal\_sources}\label{ref:facetaccumulatethermalsources}

It is assumed that any external, model-specific, heating sources have
already been accumulated (e.g., as a result of radiative absorption
(\textit{radiation\_pressure}), or aerodynamic drag heating
(\textit{aerodynamics})).  This method is intended to be used to
accumulate vehicular sources of energy, such as a power sink/source
from within the vehicle, or conduction between facets.

\funcitem{initialize}\label{ref:facetinitialize}
When the facet is initialized, we set both the
\textit{dynamic\_temperature}
(that is the operating temperature) and the
\textit{next\_temperature} (the operating temperature predicted at
the end of the current integration step) to the initialization value.

The radiative emission, known from equation \ref{eqn:thermalemission} to be 
$P_{emit} = \epsilon \sigma A T^4$ must be
calculated repeatedly.  Three of those values are constants, so those
are stored together in a new variable, \textit{rad\_constant
=}$\epsilon \sigma A$.


\funcitem{integrate}\label{ref:facetintegrate}
The integration routine is described in detail in the
\textref{integrator}{ref:thermalRK4integrator} description found in
the Mathematical Formulation section of this document.
\end{enumerate}


\classitem{ThermalIntegrableObject}er7\_utils::IntegrableObject 

This is a derived class of a generic object specialized for thermal
integration which can be integrated by JEOD.

{\bf Methods:}

\begin{enumerate}

\funcitem{compute\_temp\_dot}\label{ref:integratecomputetempdot}
Computes the time derivative of temperature and the 
instantaneous rate of emission.

\funcitem{create\_integrators}\label{integratecreateintegrators}
Required by \textit{er7\_utils::IntegrableObject}


\funcitem{destroy\_integrators}\label{integratedestroyintegrators}
Required by \textit{er7\_utils::IntegrableObject}


\funcitem{initialize}\label{ref:integrateinitialize}
Sets the initial temperature and caches a pointer to the
\textit{ThermalFacetRider} associated with the facet to be integrated.


\funcitem{integrate}\label{ref:integrateintegrate}
Implements the \textit{integrate} method of 
\textit{er7\_utils::IntegrableObject}


\funcitem{get\_temp}\label{integrategettemp}
Accessor method for the temperature

\funcitem{get\_temp\_dot}\label{integrategettempdot}
Accessor method for time derivative of temperature.


\funcitem{reset\_integrators}\label{integrateresetintegrators}
Required by \textit{er7\_utils::IntegrableObject}
\end{enumerate}


\classitem{ThermalMessages}
  Base-class

This is the message handler for the \ThermalRiderDesc.  It has
no methods.



\classitem{ThermalModelRider}
  Base-class

This is the object that rides on the interaction model directly.


{\bf Methods:}

\begin{enumerate}
\funcitem{update}
If the model is to include thermal sources coming from within the
vehicle, the \textit{model} method \textit{accumulate\_thermal\_sources} is called,
which ultimately calls the \textit{facet} method
\textref{accumulate\_thermal\_sources}{ref:facetaccumulatethermalsources}
for each of the facets.

If the temperature of the surface is intended to be dynamic, the
routine \textit{thermal\_integrator} is called
which ultimately calls 
\textref{integrate}{ref:facetintegrate} for each of the facets.
\end{enumerate}

\classitem{ThermalParams}
  Base-class

This class contains the collection of data for a thermal facet; it
contains no methods.


\end{enumerate}

