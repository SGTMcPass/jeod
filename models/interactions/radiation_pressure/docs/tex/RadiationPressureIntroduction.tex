%%%%%%%%%%%%%%%%%%%%%%%%%%%%%%%%%%%%%%%%%%%%%%%%%%%%%%%%%%%%%%%%%%%%%%%%%%%%%%%%%
%
% Purpose:  Introduction for the RadiationPressure model.
%
%
%%%%%%%%%%%%%%%%%%%%%%%%%%%%%%%%%%%%%%%%%%%%%%%%%%%%%%%%%%%%%%%%%%%%%%%%%%%%%%%%


%\section{Purpose and Objectives of \RadiationPressureDesc}
% Incorporate the intro paragraph that used to begin this Chapter here.
% This is location of the true introduction where you explain what this model
% does.

The \RadiationPressureDesc\ allows for the incorporation of radiative forces
into the \JEODid\ package for state propagation.

The magnitude of the radiation pressure force on a spacecraft is generally
small, and has negligible effect on short-duration operations.  Nevertheless,
the cumulative effect over long-duration operations can produce significant
perturbations to the orbital characteristics.  The significance of those
perturbations, compared against the perturbations generated by other environmental factors, depends on the specific orbital characteristics.
While the effects of atmospheric drag and
geo-potential terms generally decrease with altitude, the radiation pressure resulting from incident solar
radiation is largely altitude independent, and so becomes increasingly
important at higher altitudes.
For Earth orbits with
semimajor axes around 7000 km (altitude 600 km), radiation pressure effects
become comparable to atmospheric drag effects~\cite{radbib:Milani}.  They dominate
at higher orbits, becoming comparable to the higher order
(\textit{$ l {\geq} 2 $}) geopotential effects at orbit altitudes comparable
to geosynchronous orbits.

Smaller than the direct solar radiation effects, there is also indirect
radiation (or albedo pressure) to consider -- that is, solar radiation reflected
off a third body, such as Earth.
There is also the radiation emitted by the vehicle itself (thermal emission).
For highly reflective surfaces in direct sunlight, this effect is typically two to three orders of magnitude smaller than the effect of
direct solar radiation.  Nevertheless, it becomes significant in several situations:

\begin{itemize}
\item Surfaces that absorb almost all of the incident
radiation have emission forces comparable to the forces resulting from direct solar radiation pressure.
\item Unlit portions of the vehicle (such as on the dark side of the vehicle,
or in the shadow of Earth) obviously have no direct solar pressure, but will continue to experience forces resulting from thermal emission.
\item Deep-space simulations that involve a significant, localized, and
well-insulated heat source, such as a power unit have a particularly strong, localized thermal emission profile, and a solar radiation pressure that is highly dependent upon the distance from Sun.
\end{itemize}

The \RadiationPressureDesc\ in \JEODid\ calculates the effect of the direct
solar radiation pressure, and is structured to allow the easy inclusion of albedo
pressure as a user-specified extension, or in a future release.
With the incorporation of the
\href{file:\JEODHOME/models/interactions/thermal_rider/docs/thermal_rider.pdf}{Thermal
Rider}~\cite{dynenv:THERMALRIDER} model,
the temperature of any surface can be accurately simulated; with this temperature
evolution profile, the \RadiationPressureDesc\ can further model the dynamic effects of
radiation emitted from the vehicle.

The surface of the vehicle can be modeled in two ways.  The default setting is that of a
perfectly symmetric, isothermal sphere.  Alternatively, the
\href{file:\JEODHOME/models/utils/surface_model/docs/surface_model.pdf}{Surface
Model}~\cite{dynenv:SURFACEMODEL}
can be used; this provides the user with the capability of defining any surface
geometry / topology using any combination of materials.

The full Surface Model implementation requires the user to define -- for each material
used in the surface -- the albedo and the fraction of reflected light that is reflected in a
diffuse pattern (as opposed to a specular pattern).  The same can also be specified for the default
surface (which uses only one material over the entire surface).  The default surface also offers the alternative option of specifying the Coefficient of Reflection instead of the albedo and diffuse parameter.  This option IS NOT AVAILABLE for the full Surface Model, because the Coefficient of Reflection is geometry-specific and cannot be easily adapted to the total flexibility in geometry provided by the Surface Model.
