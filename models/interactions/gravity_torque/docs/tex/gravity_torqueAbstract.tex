%%%%%%%%%%%%%%%%%%%%%%%%%%%%%%%%%%%%%%%%%%%%%%%%%%%%%%%%%%%%%%%%%%%%%%%%
%
% Purpose: abstract for gravity torque model
%
%  
%
%%%%%%%%%%%%%%%%%%%%%%%%%%%%%%%%%%%%%%%%%%%%%%%%%%%%%%%%%%%%%%%%%%%%%%%%%

\begin{abstract}
The JEOD \gravitytorqueDesc\ computes the gravitational torque acting
on a spacecraft due to any number of gravitational bodies, both spherical
and non-spherical. The torque is computed from the gravity gradient, which
is computed by the JEOD Gravity Model.  The spacecraft mass properties,
in the form of an inertia tensor (matrix), are assumed to be known.
The total gravitational torque acting on the spacecraft is computed
in the spacecraft body-fixed frame and then transformed into the body structural frame
for output.

This document describes the implementation of the \gravitytorqueDesc.  It is part
of a series of inter-related documents that describe the model requirements,
specifications, mathematical theory, and architecture of the model.  A user
guide is also provided to assist with implementing the model in Trick
simulations.
\end{abstract}
