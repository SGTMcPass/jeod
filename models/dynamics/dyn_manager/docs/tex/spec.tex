%%%%%%%%%%%%%%%%%%%%%%%%%%%%%%%%%%%%%%%%%%%%%%%%%%%%%%%%%%%%%%%%%%%%%%%%%%%%%%%%
% spec.tex
% Specification of the Dynamics Manager Model
%
%
%%%%%%%%%%%%%%%%%%%%%%%%%%%%%%%%%%%%%%%%%%%%%%%%%%%%%%%%%%%%%%%%%%%%%%%%%%%%%%%%

%----------------------------------
\chapter{Product Specification}\hyperdef{part}{spec}{}
\label{ch:spec}
%----------------------------------

\section{Conceptual Design}

\subsection{Overview}

The \ModelDesc manages the dynamic elements of a simulation.
The dynamic elements in a simulation can be broadly categorized into two groups.
Vehicular bodies are the primary focus of JEOD-based simulations.
Forces and torques act to change the translational and rotational states
of these vehicular bodies. These forces and torques lead to equations of
motion that, when numerically integrated over time, yield the state history
of these vehicular bodies.

The other broad category of dynamic elements are planetary bodies and
related items. These too could be propagated via numerical integration.
That is not the approach used in JEOD. Instead, JEOD uses ephemeris models
to calculate the translational states of the planetary bodies and
(somewhat) \emph{ad hoc} rotation models to calculate the rotational states.

The model comprises three classes.\begin{itemize}
\item The DynManager class manages the dynamic elements of a simulation.
This is the main focus of this document.
\item The DynManagerInit class contains data used to initialize a DynManager
object. The DynManagerInit class was made distinct from the DynManager class
to indicate to the user (and to the developers of the model) that the
data in a DynManagerInit object are used at initialization time only.
Changing the contents of the DynManagerInit object after using it to
initialize the simulation's DynManager object has no effect on the
the simulation's DynManager object.
\item The DynManagerMessages class classifies the types of errors, warnings,
and other messages generated by the model.
\end{itemize}

\subsection{Interactions With Other Models}
The \ModelDesc interacts with several JEOD models. Chief among these
are interactions with\begin{itemize}
\item The \hypermodelref{BODYACTION}. \\
The body actions registered with this model initialize the dynamic bodies
in a simulation.
\item The \hypermodelref{DYNBODY}. \\
The dynamic bodies are the primary reason a simulation exists.
\item The \hypermodelref{EPHEMERIDES}. \\
The ephemerides models describe the locations of the planets
that are active in a simulation. Additionally, the DynManager
class directly inherits from the class EphemeridesManager.
\item The \hypermodelref{GRAVITY}. \\
The gravity models compute the gravitational accelerations on
vehicles.
\item The \hypermodelref{INTEGRATION}. \\
The integration model provides the ability to propagate the states
of the simulation's dynamic bodies over time.
\item The \hypermodelref{REFFRAMES}. \\
The states of the vehicular and planetary bodies in a simulation
are described in the form of reference frames. Additionally,
the DynManager inherits the functionality of the
RefFrameManager by virtue of its inheritance from EphemeridesManager.
\item The \hypermodelref{TIME}. \\
Time is the independent variable of the simulation.
\end{itemize}

\subsection{Modes of Operation}
The \ModelDesc operates in one of three different modes. The typical mode of
operation is the ephemeris mode. In this mode, multiple planetary bodies
exert gravitational influences on the vehicles in a simulation.
The states of these planetary bodies are determined by the simulation's
ephemerides models. The ephemerides models, using methods provided by this
model, build the base of the reference frame tree.

Simple problems should be simple to specify. The model offers two simple
modes of operation that do not require an ephemerides model to be present.
These are the single planet and empty space modes. In the single planet mode,
the simulation has but one planetary body. A non-rotating frame with origin
at the planet's center of mass is the simulation's inertial frame.
Empty space mode is even simpler. There are no planetary bodies. The
inertial frame has origin at some arbitrary point in space.

\subsection{Object Registries and Reference Frame Services}
The \ModelDesc maintains searchable registries of objects of several types:
planets, ephemeris items, reference frames, mass bodies, and dynamic bodies.
Each registry is searchable by name. Note that much of this functionality
is acquired through the inheritence of the 
EphemeridesManager \cite{dynenv:EPHEMERIDES}, which in turn inherits
from the RefFramemanager \cite{dynenv:REFFRAMES}.

One of the key concepts in JEOD 2.0 is that of a reference frame.
JEOD organizes reference frames in a tree structure.
The \ModelDesc manages the construction of the reference frame
tree. 
The root frame of the tree is the simulation's inertial
reference frame. This functionality is inherited from the
RefFrameManager \cite{dynenv:REFFRAMES}.

In addition to the basic registry described above,
the model provides the tools needed to build the base of the reference
frame tree and to identify integration frames---frames that can be used as
the basis for a propagating the state of a dynamic body. This functionality
is another example of that inherited from the 
RefFrameManager \cite{dynenv:REFFRAMES}.

\subsection{Dynamic Bodies}
Forces and torques, along with gravity, act on a dynamic body.
These forces and torques lead to equations of motion that,
when numerically integrated over time,
yield the updated state of the vehicle.
This model serves as the central hub for initializing and propagating
the states of the simulation's active dynamic bodies.

\subsubsection{Body Actions}
The model maintains a list of body actions. External entities,
typically a simulation user, adds elements to this list via an
interface provided by this model. Each body action in the list has
a subject mass body or dynamic body. The body action, when applied,
will modify some aspect of this subject body.

This model uses the body action list
at initialization time to initialize the states of the simulation's
dynamic bodies. The same list is used during the course of the simulation
run to asynchronously operate on these bodies.

\subsubsection{State Propagation}
The other key responsibility of this model with regard to dynamic bodies
is to propagate their states over time. This model contains an integrator
constructor that the dynamic bodies can use to create state integrators.
This model computes the gravitational accelerations for each dynamic
body in the simulation, invokes the dynamic bodies' force and torque accumulation methods to determine the equations of motion for those bodies,
and invokes the bodies' state integrators to propagate state over time.

\section{Mathematical Formulations}
N/A
\section{Detailed Design}
See the \href{file:refman.pdf}{Reference Manual}\cite{DynManager:refman}
for a description of classes that comprise the model and for descriptions
of the member data and member functions defined by these classes.

\section{Inventory}
All \ModelDesc files are located in
{\tt \$\{JEOD\_HOME\}/models/dynamics/dyn\_manager}.
Relative to this directory,
\begin{itemize}
\vspace{-0.2\baselineskip}
\item Header and source files are located
in the model {\tt include} and {\tt src} subdirectories.
Table~\ref{tab:source_files} lists the
configuration-managed files in these directories.
\vspace{-0.1\baselineskip}
\item Documentation files are located in the model {\tt docs} subdirectory.
See table~\ref{tab:documentation_files}
for a listing of the
configuration-managed files in this directory.
%\vspace{-0.1\baselineskip}
%\item Verification files are located in the model {\tt verif} subdirectory.
%See table~\ref{tab:verification_files}
%for a listing of the
%configuration-managed files in this directory.
\end{itemize}

\input{inventory}
