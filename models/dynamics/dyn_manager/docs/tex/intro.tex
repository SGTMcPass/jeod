%%%%%%%%%%%%%%%%%%%%%%%%%%%%%%%%%%%%%%%%%%%%%%%%%%%%%%%%%%%%%%%%%%%%%%%%%%%%%%%%
% overview_intro.tex
% Intro chapter of the Dynamics Manager Model
%
%
%%%%%%%%%%%%%%%%%%%%%%%%%%%%%%%%%%%%%%%%%%%%%%%%%%%%%%%%%%%%%%%%%%%%%%%%%%%%%%%%
\chapter{Introduction}\hyperdef{part}{intro}{}
\label{ch:intro}

\section{Purpose and Objectives of the \ModelDesc}
The \ModelDesc manages the dynamic elements (vehicles and ephemerides)
of a simulation and serves as a central repository for several kinds of
named items such as vehicles, planets, and reference frames.

\section{Context within JEOD}
The following document is parent to this document:
\begin{itemize}
\item \hyperJEOD
\end{itemize}

The \ModelDesc forms a component of the dynamics suite of
models within \JEODid. It is located at
models/dynamics/dyn\_manager.


%%%%%%%%%%%%%%%%%%%%%%%%%%%%%%%%%%%%%%%%%%%%%%%%%%%%%%%%%%%%%%%%%%%%%%%%%%%%%%%%
% change_history.tex
% History of orientation_manager.pdf
% Add a new line to the table for each update. 
%%%%%%%%%%%%%%%%%%%%%%%%%%%%%%%%%%%%%%%%%%%%%%%%%%%%%%%%%%%%%%%%%%%%%%%%%%%%%%%%

\section{Documentation History}
\begin{tabular}{||l|l|l|l|} \hline
{\bf Author } & {\bf Date} & {\bf Revision} & {\bf Description} \\ \hline \hline
 Blair Thompson & November, 2009 & 1.0 & Initial Version \\ \hline
 David Hammen & October, 2010 & 2.0 & JEOD 2.1 \\ \hline
\end{tabular}



\section{Documentation Organization}
This document is formatted in accordance with the 
NASA Software Engineering Requirements Standard~\cite{NASA:SWE}.

The document comprises chapters organized as follows:

\begin{description}
\item[Introduction] - 
This introduction describes the objective and purpose of the \ModelDesc.

\item[Product Requirements] -
The requirements chapter describes the requirements on the \ModelDesc.

\item[Product Specification] -
The specification chapter describes
the architecture and design of the \ModelDesc.

\item[User Guide] -
The user guide chapter describes
how to use the \ModelDesc.
 
\item[Inspection, Verification, and Validation] -
The inspection, verification, and validation (IV\&V) chapter describes
the verification and validation procedures and results for the \ModelDesc.
\end{description}
