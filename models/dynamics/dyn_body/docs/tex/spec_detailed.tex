\section{Detailed Design}

The classes and methods of the \ModelDesc are described in detail in the
\hyperapiref{DYNBODY}.

The outline of this section is organized along the lines of the
key concepts described in section~\ref{sec:key_concepts}.
Only those concepts that impact the detailed design are described in
this section.


\subsection{Mass}

A DynBody \hasa public MassBody attribute with friendship access 
declared in MassBody.

In dyn\_body.hh
\begin{verbatim}
class DynBody : virtual public RefFrameOwner,
    virtual public er7_utils::IntegrableObject { 
...
public:
...
    MassBody mass;
...
};
\end{verbatim}

In mass\_body.hh
\begin{verbatim}
class MassBody {
...
    friend class DynBody;
...
};
\end{verbatim}

Note that the DynBody class uses multiple inheritance. The RefFrameOwner class 
is merely an abstract class for defining interface taxonomy. The MassBody 
class is a full-fledged class.
The public MassBody friendship 
functionality is passed to other classes and functions that use a
DynBody object. The protected MassBody functionality is fully visible
within a DynBody class but not within classes that derive from the DynBody 
class. Readers who wish to learn more about the MassBody
functionality should refer to the \hypermodelref{MASS}.

In previous versions of JEOD, a Simple6DofDynBody class inherited from the 
DynBody class to provide IntegrableObject functionality to DynBody. All of 
this functionality was collapsed into DynBody itself as of JEOD v4.0, because 
nearly all JEOD use cases implemented Simple6DofDynBody, making intermediate 
inheritance unnecessary.

\subsection{State}
The DynBody class contains three BodyRefFrame objects that correspond to the
three MassProperties objects defined in the MassBody class. These are
\begin{inparaenum}[(1)]
\item \verb+core_body+, which represents the state of the
  MassBody's \verb+core_properties+ object,
\item \verb+composite_body+, which represents the state of the
  MassBody's \verb+composite_properties+ object, and
\item \verb+structure+, which represents the state of the
  MassBody's \verb+structure_point+ object.
\end{inparaenum}
The connection between these three three BodyRefFrame objects and the
corresponding MassProperties object is made by the DynBody constructor. 
Further BodyRefFrame's corresponding to the MassBody's MassPoints may be 
constructed through attachment.

\subsection{Inertial Frames}
The \DYNMANAGER, with the help of other models, determines which frames
are 'inertial' frames and which are not. The \ModelDesc uses the
\DYNMANAGER public interfaces that pertain to this concept.

\subsection{Integration Frame}
The user sets the integration frame \emph{by name} at initialization time.
The initialization-time method \verb+DynBody::initialize_model()+ sets the
integration frame based on the provided name.

All attached DynBody objects share a common integration frame. The attach
mechanisms described below ensure that this is the case.

The model provides public and protected methods to switch the integration frame.
The public methods are overloaded versions of
\verb+switch_integration_frames()+. This is the interface that should
be used to switch the integration frames. These methods ensure that the
state is preserved when switching frames and propagate the change throughout
the DynBody object.
The protected methods are overloaded versions of \verb+set_integ_frame()+.
These methods do not preserve state and are for internal use only.

\subsection{Attach/Detach}\label{sec:detailed_attach_detach}
The DynBody attach/detach mechanism separately performs the actions encompassed 
by MassBody attach/mechanisms, with further extensions to verify validity and 
perform dynamics updates (e.g. update the composite BodyRefFrame). The mathematics
of this is described in section~\ref{sec:math_attach_detach}. None of the 
attach methods are intended to be overridden by derived classes in order to 
preserve core functionality. Extensibility can be achieved by manipulating 
configuration before or after invoking these interface methods.

The top level methods used to attach and detach DynBody instances are: \begin{itemize}
\item \verb+DynBody::attach()+ to specify a DynBody to attach as a child, 
\item \verb+DynBody::add_mass_body()+ to specify a MassBody or linkage of 
    MassBodys to attach as child(ren), 
\item \verb+DynBody::detach()+ to order this body to detach from its parent 
    or detach another DynBody from this body,
\item \verb+DynBody::remove_mass_body()+ to order this body to detach child(ren) 
    MassBodys. 
\end{itemize}

Attachment orientation is specified in one of two ways: **explicitly** by 
relative position/orientation between the body structures or **implicitly** by 
names of body points to mate (with orientation defined as opposing in X and 
aligned in Y w.r.t. point frames). 
Momentum conservation is performed only if both the child and parent
have fully initialized states. Two other cases are possible:
\begin{itemize}
\item Neither the parent nor child state has been initialized. This occurs
during initialization-time attachments, which intentionally occur prior
to state initialization. Only the MassBody properties are updated in this case.
\item The parent has a fully initialized state but the child has no state
information. This occurs when a simple MassBody is attached to a DynBody
after initialization. The MassBody magically appears alongside the
DynBody. The attachment is assumed to occur with no change to the parent
body's core and structural states. State is propagated from the parent
body's structural frame.
\end{itemize}

The attach mechanism invokes protected intermediate functions to organize the 
attachment process. Several of these functions belong to MassBody and are
exploited by DynBody through friendship.
These methods are\begin{itemize}
\item \verb+DynBody::attach_update_properties()+ to perform dynamic attachment, 
\item \verb+MassBody::attach_establish_links()+ to update searchable tree linkage, 
\item \verb+MassBody::detach_establish_links()+ to remove tree linkage, 
\item \verb+MassBody::attach_update_properties()+ to combine the MassBody's internal
    properties, 
\item \verb+MassBody::attach_update_properties()+ to separate the MassBody's internal
    properties, and
\item \verb+MassPoint::compute_state_wrt_pred()+ to perform relative geometry 
    calculations. 
\end{itemize}
These functions ensure that the attribute MassBody have correct composite 
properties and linkages. The DynBody attach/detach methods also invoke 
validation checks during the process to verify the commanded attachment is 
valid.
These methods are\begin{itemize}
\item \verb+attach_validate_parent()+,
  which is sent to the child body with the parent body supplied as an argument;
\item \verb+attach_validate_child()+,
  which is sent to the parent body with the child body supplied as an argument;
\item \verb+add_mass_body_validate()+,
  which is called by the DynBody with the child MassBody supplied as an argument;
\end{itemize}
The validation methods ensure that the bodies are registered with JEOD, will 
not create a nodal attach tree such that any node is multiply or cyclically 
attached, and that the supplied arguments do not refer to invalid types (e.g., 
a DynBody may not attach as a child to a MassBody parent).

\subsection{Initialization}
Initialization of a DynBody object is a complex process that is spread amongst
constructors, \Sdefine-level initialization methods, and the body action-based
initialization mechanism. All elements of a DynBody object are initialized
by the constructors. Pointers (including essential ones) are set to NULL and
states are set to a default value. A few known, non-null initializations occur
during this zeroth phase of the initialization process.

The next step in the process is to perform model-level initializations.
This is performed by the DynBody method {\tt initialize\_model()}.
This method registers the DynBody object and the basic reference frames with the dynamics manager and sets the object's integration frame.
This step leaves the mass properties and states uninitialized.

The final step of the initialization process is best performed using the
body-action based initialization mechanism. This operates in three stages,
with mass properties and mass points initialized first, initialization-time
attachments performed next, and body states initialized last.

\subsection{State Initialization}
The model participates in the recommended body action-based state initialization
process in two ways, readiness checking and action application.
The body action class DynBodyInit and its derived classes are responsible
for state initialization, and both the readiness checking and action application
involve interactions with the \ModelDesc.
Readiness checking and action application are integral parts of the body
action-based scheme. The simulation's dynamic manager invokes a body action's
\verb+apply()+ method only if the action's \verb+is_ready()+ method assents to
the forthcoming application.
In the case of an action that derives from the DynBodyInit class, the
\verb+is_ready()+ mechanism includes checks of whether required state elements
in various BodyRefFrame objects have been initialized. The
action is ready
to be applied when all prerequisites have been satisfied.

The \ModelDesc involvement in the readiness checking centers around the
BodyRefFrame class. This class augments the RefFrame class with an
\verb+initialized_items+ data member of type RefFrameItems.
A RefFrameItems object contains an enumerated value that indicates which of the
position, velocity, orientation, and angular velocity elements of a state have
been set and provides methods to operate on that enumerated value.
The \ModelDesc methods described in section~\ref{sec:detailed_state_prop} ensure
 that a BodyRefFrame object's \verb+initialized_items+ are updated to reflect
which of the object's state elements have been set.
This mechanism makes readiness checking quite easy; all the DynBodyInit object
needs to do is query the BodyRefFrame object's \verb+initialized_items+ data
member as to whether all required state elements for that object have been set.

The \ModelDesc involvement in the application of a DynBodyInit object
centers around the DynBody class.
The \verb+DynBodyInit::apply()+ method class invokes the subject
DynBody object's \verb+set_state()+ method to set the user-specified state
elements and then invokes the subject DynBody object's \verb+propagate_state()+
method to propagate that newly computed state throughout the DynBody tree. The former
sets the user-specified state elements in a user-specified body reference frame; the latter propagates known state elements throughout the tree.
For details on the \verb+set_state()+ and \verb+propagate_state()+ methods see
see section~\ref{sec:detailed_state_prop} below.

\subsection{Forces and Torques}\label{sec:detailed_force_torque}
Forces and torques in JEOD have always been, and continue to be, represented in the
structural frame of the DynBody object with which the forces and torques are
directly associated. The association between forces/torques and a DynBody
object is achieved via the Trick \emph{vcollect} mechanism. The responsibility
to properly represent forces and torques in an appropriate structural frame
falls on the developer of a particular force/torque model while the
responsibility to properly associate the forces and torques with the correct
DynBody object falls on the simulation integrator.
From the perspective of the \ModelDesc, the collected forces and torques are
simply the elements of the six STL collect vectors data members contained in
the DynBody object's \verb+collect+ data member.

The \verb+collect+ data member is a BodyForceCollect object.
The BodyForceCollect member data are:\begin{itemize}
\item Six STL vectors, three each for forces and torques, representing the
  effector, environmental, and non-transmitted forces and torques
  acting on a body;
\item Six correspondingly named 3-vectors\footnote{The overloading of the term
  `vector' is a bit unfortunate here. The objects into which the forces
  and torques are collected are STL vectors; that name was chosen by the C++
  community. JEOD uses 3-vectors to represent forces and torques (and positions,
  velocities, ...); that use of the word `vector' predates the computer science
  use of the word by a considerable amount of time.} that represent the
  category-specific sums of of the active collected forces and torques;
\item Two more 3-vectors that contain the total force and torque as represented
  in the structural frame, and
\item Another two 3-vectors that contain the total force and torque
  transformed to the frame needed by the equations of motion.
  This equations of motion frame is the inertial integration frame for force
  and the body frame for torque.
\end{itemize}

The three STL vectors that represent forces contain CollectForce pointers.
A CollectForce object contains two pointers, active and force. The latter is a
3-vector that contains the components of the force in structural coordinates.
The former is a pointer to a boolean that indicates whether the force 3-vector
should be included in the totals.
The three STL vectors that represent torques contain CollectTorque vectors.
A CollectTorque is analogous to a CollectForce, with a torque data member
replacing the force data member.
See section~\ref{sec:detailed_collect} below for details on the
population of the six STL vectors in a BodyForceCollect object and
construction of the CollectForce and CollectTorque objects.

The method \verb+DynBody::collect_forces_and_torques()+ is responsible for
interpreting and manipulating the contents of the DynBody \verb+collect+ data
member. This method first sums the members of the six collect vectors, storing
the sums in the correspondingly named sum. For example, the effector forces
are collected at the \Sdefine-level into the \verb+collect_effector_forc+
collect vector; the sum of the active elements of this STL vector is computed
and stored in the \verb+effector_forc+ data member. The method is then
recursively invoked on the DynBody objects attached as children to the DynBody
object in question. The next stage of operation depends on whether the DynBody
object is an attached or root body.

In the case of attached bodies, the transmittable forces and torques are
converted to the parent body's structural frame and added to the parent's total.
For example, a child body's total effector torque is converted to the parent
body's structural frame and added to the parent body's
\verb+collect.effector_torq+ data member.
Forces are treated similarly, but with the added complexity that a child body
force becomes a force and a torque when transmitted to the parent body.

In the case of a root body, the categorized totals for forces and torques are
summed to form the grand total external force and external torque. These
initial grand totals are expressed in structural components.
These structurally-referenced grand totals are then transformed to the
frame in which the equations of motion are represented---the inertial
integration frame in the case of force but the body frame in the case of torque.
The final step is to formulate the equations of motion for the composite body. The total inertial-referenced external force is combined with gravitational
acceleration to form the translational equations of motion.
The total body-referenced external torque is combined with the
inertial torque to form the rotational equations of motion.
The mathematical details of these operations are described in section
~\ref{sec:math_collect} and ~\ref{sec:math_eom}.

\subsection{Wrenches}\label{sec:detailed_wrench}
Like forces and torques, wrenches in JEOD are represented in the structural
frame of the DynBody object with which the wrenches are directly associated.
Also like forces and torques, the association between wrenches and a
DynBody object is achieved  via the Trick \emph{vcollect} mechanism.
Unlike forces and torques, the collection mechanism for wrenches accounts for
how forces induce torques if the line of action of the force does not pass
through the vehicle's center of mass. This is what makes wrenches so appealing:
They simplify the development of vehicle-specific models. A modeler no longer
needs to calculate the torque induced by a force; the wrench mechanism does
that for them.

From the perspective of the \ModelDesc, the collected wrenches add to the
collected forces and torques (this older concept will not be abandoned),
but with all of the individual wrenches displaced to the vehicle
center of mass during the collection process. To keep the concepts somewhat
distinct, JEOD 3.4 uses a separate object for collecting wrenches. This is
the \verb+effector_wrench_collection+ data member of StructureIntegratedDynBody.
As of JEOD 3.4, wrenches can only be used in the context of this class.
The DynBody concepts of non-transmitted and environmental forces and torques
have not yet been propagated to wrenches. This may change in future releases.

The class Wrench provides \verb|operator+=| to simplify the wrench collection
process. This function transforms the summand wrench to be added to the point of
the application of the sum and then adds the force and transformed torque
to the force and torque of the sum.

\subsection{Collect Mechanism}\label{sec:detailed_collect}
The elements of the six collect vectors in a BodyForceCollect object are
populated by the Trick \emph{vcollect} mechanism.
This mechanism uses a \emph{converter} to convert the item list elements to a
form suitable for pushing on the STL \emph{container};
see section~\ref{sec:spec_interactions_trick} for details on the Trick side
of this mechanism.
This mechanism is used in a JEOD-based simulation to add forces and torques to
one of the collect vectors contained in a DynBody object's collect data member.
The \emph{container} is one of the six BodyForceCollect collect vectors;
the \emph{converter} is \verb+CollectForce::create+ for the force collect
vectors but \verb+CollectTorque::create+ for the torque collect vectors.

The CollectForce and CollectTorque classes are Factory Method classes,
with the \verb+create()+ methods forming overloaded named constructors
that create CollectForce and CollectTorque objects.
Each class provides five overloaded versions of \verb+create()+ distinguished
by argument type. The \verb+CollectForce::create()+ methods are
\begin{itemize}
\item \verb+static CollectForce * create (double * vec)+\\
  Creates a CInterfaceForce, a CollectForce derived class. The
  force pointer in the created CInterfaceForce object is the input \verb+vec+
  pointer. The \verb+active+ pointer is a newly allocated boolean pointer
  with the contents set to true. (A CInterfaceForce is always active.)
  The CInterfaceForce destructor releases this allocated memory.
\item \verb+static CollectForce * create (Force & force)+\\
  Creates a CollectForce with the force and active pointers in the created
  CollectForce object pointing to the input Force object's force 3-vector and
  active flag.
\item \verb+static CollectForce * create (Force * force)+\\
  This behaves similarly to the previous method; the only distinction is that
  the argument is a pointer to rather than a reference to a Force object.
\item \verb+static CollectForce * create (CollectForce & force)+\\
  Creates a CollectForce via the CollectForce copy constructor.
\item \verb+static CollectForce * create (CollectForce * force)+\\
  This behaves similarly to the previous method; the only distinction is that
  the argument is a pointer to rather than a reference to a CollectForce object.
\end{itemize}
The \verb+CollectTorque::create()+ methods are analogous to those of CollectForce class.

The class StructureIntegratedDynBody augments the collected forces and torques
with the forces and torques from the collected wrenches.

\subsection{Gravity}\label{sec:detailed_gravity}
The \ModelDesc does not compute the gravitation acceleration that acts
on a DynBody object. The \DYNMANAGER does this on the behalf
of this model, and does so by invoking the \GRAVITY.
The \Sdefine-level \verb+DynManager::gravitation()+ method  invokes the
\verb+GravityManager::gravitation()+ method for each root DynBody object.
This architecture, while a bit duplicitous, eliminates a source of user errors,
eliminates some rather ugly \Sdefine code, and offers a significant speedup
when bodies are attached (gravitation is computed for root bodies only).

The DynBody class contains a \verb+GravityInteraction+
\verb+grav_interaction+ data member. This data member is passed, along
with the DynBody object's position vector, as arguments to the GravityManager
\verb+gravitation()+ method. The GravityInteraction data members include
an array of GravityControls objects that direct the gravitational computations,
a specification of the DynBody object's integration frame, and
placeholders for the outputs from the gravitational computations.
These output products are
the gravitational potential,
the gravitational acceleration vector,
and the gravity gradient tensor.

Each element of the GravityControls array in a GravityInteraction object
specifies a gravitational body such as the Sun or Earth and specifies how
the gravitational computations are to be performed for that gravitational body.
Since each DynBody object has its own GravityInteraction object, different
DynBody objects can specify different computation techniques for a given
gravitational body.

For example, consider a multi-vehicle simulation with one vehicle in low Earth
orbit and another orbiting the Moon. The vehicle in low Earth orbit can
treat the Moon as a point mass but use a high order model for the Earth.
The vehicle in lunar orbit can use the reverse setup, with a high order model
used for lunar gravity but a low order model for Earth gravity.

The gravity computations can also be changed over the course of a simulation.
For example, consider a simulation that starts with a vehicle in Earth orbit,
performs a trans-Mars burn, and eventually reaches Mars. The simulation
might start with Mars disabled, a point mass model for the Sun, and a high order
model for the Earth. The Earth's gravitational influence diminishes as the
vehicle starts moving away from the Earth, and eventually becomes negligible
when the vehicle is well outside the Earth's Hill sphere. This suggests toning
down the fidelity of the Earth's gravity model and eventually disabling the
Earth completely. The reverse situation holds as the vehicle approaches Mars.

\subsection{Equations of Motion}\label{sec:detailed_eom}

Formulating the equations of motion is a part (a very small part) of the
operations performed by \verb+DynBody::collect_forces_and_torques()+.
See section~\ref{sec:detailed_force_torque} for a description of this method.

\subsection{State Integration}\label{sec:detailed_state_integ}
The DynBody class declares three virtual methods that pertain to state
integration, which may be overriden by classes such as StructureIntegratedDynBody.
 Users are free to write classes that derive from DynBody or
 that provide alternate implementations.

The three methods are\begin{description}
\item[{\tt create\_integrators}] This initialization-time method creates the
  state integrators that will be used during run time to integrate state.
  The method receives pointer to an instance of a class that derives from the
  IntegratorConstructor class. The \verb+create_integrators()+ method uses
  this to create state integrators that implement the user-selected
  integration algorithm.
  The DynBody implementation creates two state integrators,
  one for the translational state and one for the rotational state.
  The created translational state integrator uses 3-vectors for generalized
  position and velocity. The created rotational state integrator uses
  a 4-vector for generalized position and a 3-vector for generalized velocity.
\item[{\tt reset\_integrators}] Integrators that use historic data need to
  discard that historic data on occasion. This method simply forwards the
  \verb+reset+ on to the state integrators.
\item[{\tt integrate}] This method performs the state integration. The
  DynBody implementation accomplishes this by calling two internal
  methods. One of these internal methods integrates the translational state of
  the DynBody object's integrated frame using the object's
  translational state integrator (which was created by the call to
  \verb+create_integrators()+).
  The other integrates the integrated frame's rotational state using the
  rotational state integrator.
\end{description}
Most of the work is performed by the state integrators. See the
\hypermodelref{INTEGRATION} for details.

\subsection{Center of Mass State Integration}\label{sec:detailed_state_integ_com}
The translational and rotational equations of motion decouple at a rigid
body's center of mass. The translational state equations of motion are given by
Newton's second law, $\boldsymbol F=m \boldsymbol a$ . JEOD separates
gravitational interactions from external forces, resulting in
$\boldsymbol a_\text{net} =
\boldsymbol a_\text{grav} + \boldsymbol F_\text{ext}/m$ .
The inertial state equations of motion are given by the rotational equivalent
of Newton's second law, with an additional term that results from working in
what inherently is a rotating frame,
$\boldsymbol \tau = I \dot{\boldsymbol \omega} +
(\boldsymbol{\mathrm I}\boldsymbol{\omega}) \times \boldsymbol{\omega}$ .

\subsection{Structural State Integration}\label{sec:detailed_state_integ_struct}
The translational and rotational equations of motion are coupled elsewhere
on a rigid body. The translational state equations of motion at a point removed
from the center of mass must account for centrifugal and Euler accelerations:
$\boldsymbol a_\text{net} =
\boldsymbol a_\text{grav} + \boldsymbol F_\text{ext}/m
+ \boldsymbol{\omega} \times (\boldsymbol{\omega} \times \boldsymbol r)
+ \dot{\boldsymbol \omega} \times \boldsymbol r$ .
Rotational state is even messier. However, the equations of motion for rotational
state simplify to exactly the same equations used for rotation about the center
of mass. This shouldn't be surprising; angular velocity and angular acceleration
are "free vectors": They're the same for any point on a rigid body.
The class StructureIntegratedDynBody takes advantage of this. It uses the
same equations of motion for rotational state as does DynBody.
It does however account for coupling in the translational equations of motion.

\subsection{State Propagation}\label{sec:detailed_state_prop}
The term `state propagation' as used in this document and in the model code
means propagating the state from a reference frame associated with a DynBody
object to other reference frames associated with that DynBody object, either
directly or via attachment, using the geometry information embedded in the
MassBody objects that underlie the DynBody objects. State propagation is a
topic distinct from state integration. State needs to be propagated from frame
to frame for several reasons:
\begin{itemize}
\item During initialization. The user can set state elements in a staged manner,
and can set different state elements in different frames. For example, a user
might set the position and velocity of a subject vehicle in terms of the
relative position and velocity between a VehiclePoint on the subject vehicle and
another VehiclePoint on some reference vehicle, but set the orientation and
angular velocity in terms of the relative orientation and angular velocity
of the subject vehicle's body frame with respect to the reference vehicle's
LVLH frame.
\item After integration. The state integration mechanism integrates the state
of one reference frame, the root DynBody object's integrated frame. The states
of all of the other reference frames associated with a DynBody tree is propagated from this integrated frame.
\item During attachment. The attach mechanism calculates and sets the state of
one reference frame. The states of all of the other reference frames associated
with a DynBody tree is propagated from this updated frame.
\item During detachment. The states of all reference frames associated with a
DynBody tree is propagated from this parent body's core body reference frame.
\item During frame switch. The state of the root body's integrated frame is
updated to reflect the switch to a new integration frame. The states
of all of the other reference frames associated with a DynBody tree is propagated from this integrated frame.
\item Per some user-defined model. For example, consider a JEOD-based federate
in a distributed multi-vehicle simulation. The federate integrates vehicle
states over time, even the states of vehicles that are not owned by the
federate. When the owner of a state broadcasts new state elements, the federate
must override the integrated state with the new official values. The state
propagation mechanism can be employed to accomplish this goal.
\end{itemize}

The DynBody class provides several public methods related to
initializing/overriding states and propagating the resulting settings
throughout the DynBody tree. These methods are listed below.
All but the last three methods in the list below take two arguments,
the value of that element and the reference frame for which the appropriate
state element is to be set. The last listed method is particularly
important. Please read the entire list. These methods are
\begin{itemize}
\item \verb+set_position()+
  sets the position element of the state of the input reference frame
  to the input position and denotes that the input frame is
  the definitive source for the DynBody object's position.
\item \verb+set_velocity()+
  sets the velocity element of the input frame's state and denotes that frame as
  the definitive source for the DynBody object's velocity.
\item \verb+set_attitude_left_quaternion()+
  sets the attitude element of the input frame's state and denotes that frame as
  the definitive source for the DynBody object's velocity.
\item \verb+set_attitude_rate()+
  sets the angular velocity element of the input frame's state and denotes that
  frame as the definitive source for the DynBody object's angular velocity.
\item \verb+set_attitude_right_quaternion()+
  provides an alternate means for setting attitude.
  JEOD uses left quaternions; others use right quaternions.
  This method converts the input right quaternion to a left quaternion
  before setting the attitude element.
\item \verb+set_attitude_matrix()+
  provides yet another alternate means for setting attitude.
  This method converts the input transformation matrix to a left quaternion
  before setting the attitude element.
\item \verb+set_state()+
  sets multiple state elements at once. This method takes three arguments,
  a bit flag that specifies which state elements are to be set,
  a RefFrameState that contains the desired state values,
  and the subject reference frame.
  It sets selected elements of the input frame's state based on the input state.
  The bit flag specifies which elements are to be set.
  The input frame is denoted as the definitive source for all set elements.
\item \verb+set_state_source()+
  denotes the given reference frame to be the definitive source for a
  set of specified state elements. This method takes two arguments,
  a bit flag that specifies the state elements and the subject reference frame.
  This method does not set state. It is used when some particular frame's state
  is known to be correct but the states of other frames is incorrect.
  This occurs, for example, with detachment. Detachment makes the state of the
  composite body frame of the parent body incorrect. The state of the core
  body frame is correct, so the detach mechanism sets the core body to be the
  (temporary) source of state information.
\item \verb+propagate_state()+
  propagates state settings throughout the DynBody tree.
  This method takes no arguments.
  Calling any of the above \verb+set_+ methods creates a problematic
  situation. The state of one frame has been set, but this setting has not
  been propagated throughout the DynBody tree. States are inconsistent with
  geometry. This final method rectifies that situation. If possible,
  it propagates state from the denoted definitive sources to the integrated
  frame, and from there to all frames associated with the DynBody tree.
  Upon completion, the integrated frame becomes the definitive source for all
  state elements.
\end{itemize}

The method \verb+propagate_state()+ can always propagate state as described above once all four state elements (position, velocity, attitude, and angular
velocity) have been set. This propagation is not always immediately possible
during initialization due to the staged initialization scheme employed in JEOD.
Individual state elements can be propagated, provided the propagation is
performed in the order listed in table~\ref{tab:init_prop}.
The first column of the table lists the element to be propagated;
the second column specifies which state elements must have already been
initialized in order to perform the propagation.

\begin{table}[htp]
\centering
\caption{Initialization Time State Propagation}
\label{tab:init_prop}
\vspace{1ex}
\begin{tabular}{||l|l|} \hline
{\bf Element to be} & {\bf Required} \\
{\bf Propagated} & {\bf Elements} \\
\hline \hline
Orientation & No constraints \\
Angular velocity & Orientation \\
Position & Orientation \\
Velocity & Orientation and angular velocity \\
\hline
\end{tabular}
\end{table}


\subsection{Point Acceleration}
The DynBody method \verb+compute_vehicle_point_derivatives()+
computes the derivatives of the velocity and angular velocity vector for
a specified BodyRefFrame. The method uses the mathematics described in
section~\ref{sec:math_pt_accel} to calculate these derivatives.

The envisioned use of this method is to provide the truth data needed by an
accelerometer or inertial measurement unit (IMU). The specified BodyRefFrame
will then be the case frame associated with that accelerometer or IMU.

One caveat: This is a generic DynBody method. As such it makes the simplifying
assumption that changes in mass properties are small and do not measurably
impact dynamics.
