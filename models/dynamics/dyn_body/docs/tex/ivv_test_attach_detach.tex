\test{Attach/Detach}\label{test:attach_detach}
\begin{description}
\item[Background] The primary purpose of this test is to test that
the \ModelDesc attach and detach mechanisms operate as required.
This includes satisfying the basic attach and detach requirements from
the \MASS and satisfying the new DynBody requirement to obey the conservation laws
while attaching and detaching objects.

\item[Test description]
This test places dynamic bodies in rather contrived circumstances to
ensure the bodies come into alignment. The bodies attach at just the point
in time when they do come into alignment. Bodies also detach in this test.
The expectation is that the attachments should result in properly calculated
mass properties and that attachments and detachments should conserve linear
and angular momenta.

\item[Test directory]
{\tt models/dynamics/dyn\_body/verif/SIM\_verif\_attach\_detach} \\
This simulation defines three dynamic bodies and various \BODYACTION
attach/detach objects that make the simulation's dynamic bodies
attach to and detach from one another.

The simulation contains two run directories.
The \verb+RUN_simple_attach_detach+ directory uses only two of the
dynamic bodies. The two bodies are configured to come into alignment
ten seconds into the simulation, at which time they attach to one another.
After another ten seconds the bodies detach. Then, a rotational spin is applied to body1 and the body2 is
kinematically attached to a mass point on the body1. After some time, the spinning body is attached to the central
point making all bodies attached to the integration frame with no motion at an offset. Finally, body1 is detached and
reattached to the integration frame with no offset.

The \verb+RUN_complex_attach_detach+ directory uses all three vehicles.
The simulation bodies combine in two steps to form a three-body composite body.
The latter attachment specifies an attachment that, if performed as requested,
would create a non-tree structure.
One body detaches from this composite and reattaches later. Finally,
the specific detachment capability is tested.

\item[Success criteria]
Table~\ref{tab:attach_detach_criteria} lists the items to be tested at specific
events. Two types of criteria exist:
Logical tests of the connectivity status, and numerical tests of mass properties
and states. The connectivity must represent the attach/detach event that just
occurred. The numerical results from the simulation should match analytic
results to within numerical precision.

\item[Test results]
All tests pass.

\item[Applicable requirements]
This test demonstrates the partial or complete satisfaction of the
following requirements:
\begin{itemize}
\item \ref{reqt:mass_requirements}. Mass properties are properly initialized and
are properly updated due to attachments and detachments.
\item \ref{reqt:eom}.
Angular acceleration, including that due to inertial torque, is properly
calculated.
\item \ref{reqt:state_integ_prop}. The equations of motion are properly
integrated.
\item \ref{reqt:vehicle_points}. Vehicle points are properly constructed
and used during attachments.
\item \ref{reqt:attach_detach}. Demonstrating this capability
is the primary purpose of this test.
\end{itemize}

\end{description}

\begin{table}[htp]
\centering
\caption{Attach/Detach Test Criteria}
\label{tab:attach_detach_criteria}
\ \\[2ex]Simple Attach/Detach \\[6pt]
\begin{tabular}{||r@{}l|p{2.4in}|p{2.7in}|} \hline
\multicolumn{2}{||l|}{\bf Time}
  & {\bf Event} & {\bf Test Items} \\ \hline \hline
0 && Mass properties initialized & Body mass properties \\
 && States initialized & Body states \\[6pt]
10 && Bodies 1 and 2 come into alignment & Relative state \\
 && Body 1 attaches to body 2 & Connectivity (Body 1 attached to 2) \\
 &&& Composite mass properties \\
 &&& Composite state \\[6pt]
20 && Body 1 detaches & Connectivity (Body 1 detached from 2) \\
 &&& Body states \\[6pt]
30 && Body 1 begins rotational spin & Body states \\[6pt]
35 && Body 2 kinematic attach to Body 1 & Connectivity (Body 2 attached to 1) \\
 &&& No change to composite mass props for either body \\
 &&& Body 2 relative to Body 1 is static \\
 &&& Body 1 state continues undisturbed \\[6pt]
40 && Body 1 kinematic attach to Central Point & Connectivity (Body 1 attached to "Central Point") \\
   &&& Body states are static \\[6pt]
50 && Body 1 detaches from the Central Point & Connectivity (Body 1 attached to "Central Point") \\
   && Body 1 kinematic attaches to the Central Point with no offset & Body 1 has a zero state \\
   &&& Body states are static \\
\hline
\end{tabular}
 \ \\[4ex]Complex Attach/Detach \\[6pt]
\begin{tabular}{||r@{}l|p{2.4in}|p{2.7in}|} \hline
\multicolumn{2}{||l|}{\bf Time}
  & {\bf Event} & {\bf Test Items} \\ \hline \hline
0 && Mass properties initialized & Body mass properties \\
 && States initialized & Body states \\[6pt]
10 && Bodies 1 and 2 come into alignment & Relative state \\
 && Body 1 attaches to body 2 & Connectivity (Body 1 attached to 2) \\
 &&& Composite mass properties \\
 &&& Composite state \\[6pt]
32&.78 & Bodies 1 and 3 come into alignment & Relative state \\
 && Body 1 attaches to body 3 & Connectivity (Body 1 attached to 2 to 3) \\
 &&& Composite mass properties \\
 &&& Composite state \\[6pt]
50 && Body 1 detaches & Connectivity (Body 1 detached from 2) \\
 &&& Body states \\[6pt]
55 && Bodies 1 and 2 come into alignment & Relative state \\
 && Body 1 attaches to body 2 & Connectivity (Body 1 attached to 2 to 3) \\
 &&& Composite mass properties \\
 &&& Composite state \\[6pt]
60 && Body 1 detaches from body 3 & Connectivity (Body 2 detached from 3) \\
 &&& Body states \\
\hline
\end{tabular}
\end{table}
