\section{Mathematical Formulation}
\label{sec:mathematics}
This section summarizes key equations used in the implementation of the \ModelDesc. The outline of this section is organized along the lines of the
key concepts described in section~\ref{sec:key_concepts}.
Only those concepts that involve mathematical details are described in
this section.

\subsection{Attach/Detach}\label{sec:math_attach_detach}
The underlying MassBody class addresses the geometry of attachment and
detachment, with BodyRefFrames constructed for MassPoints as necessary. The DynBody class needs to address state as well, and must do this
in a manner consistent with the laws of physics (if possible).
This is not always possible with attachment as the attach model allows
physically impossible attachments to take place:
The attaching bodies instantaneously snap into place upon attachment.
For example, the instantaneous attachment of a MassBody or MassBody derived 
state (which does not have a dynamic state) cannot conserve momentum.
The rationale for instantaneous snap attachment in general is that modeling 
the detailed contact forces and
torques that take place during the docking of two vehicles is beyond the scope
of many orbital simulations. The simulation instead brings the vehicles into
very close proximity and alignment and then magic happens: The vehicles attach.
Linear momentum can always be conserved, even in non-physical attachments.
Angular momentum is more problematic. The model conserves angular momentum
in the frame that is centered at and moving with the post-docking combined
center of mass. This choice is consistent with the laws of physics when the
attachment does not involve a step change in position and orientation.

Conservation of linear momentum determines the velocity of the combined
vehicle. Equating the linear momentum before and after the attachment yields

\begin{equation}
(m_p + m_c) \framevect I v_+ =
  m_p \framevect I v_{p_-} + m_c \framevect I v_{c-}
\end{equation}

where $m_p$ and $m_c$ are the pre-attachment masses of the parent and child
bodies, $\framevect I v_{p_-}$ and $\framevect I v_{c_-}$ are the
pre-attachment velocities of the parent and child bodies in the parent body's
integration frame, and $\framevect I v_+$ is the post-attachment velocity of
the combined center of mass.
Solving for the post-attachment velocity yields

\begin{equation}
\framevect I v_+ =
  \frac{m_p \framevect I v_{p_-} + m_c \framevect I v_{c-}}{m_p + m_c}
\end{equation}

The pre-attachment angular momentum in the non-rotating frame with origin at
the combined center of of mass $\framevect I r_+$ and moving at velocity $\framevect I v+$ results from translation with respect to this origin and
body rotations:

\begin{equation}
\begin{aligned}
\framevect I L_- =\,
  &
  m_p (\framevect I r_{p_-}- \framevect I r_+) \times
      (\framevect I v_{p_-}- \framevect I v_+) +
  {\tmatT I B_p} (\MxV {\inertia_p} {\framevect B \omega_{p_-}})\;+ \\
  &
  m_c (\framevect I r_{c_-}- \framevect I r_+) \times
      (\framevect I v_{c_-}- \framevect I v_+) +
  {\tmatT I B_c} (\MxV {\inertia_c} {\framevect B \omega_{c_-}})
\end{aligned}
\end{equation}

Transforming to the parent body frame,

\begin{equation}
\begin{aligned}
\framevect {B_p} L_- =\,
  & \MxV {\tmat I B_p}
         {\left(
            m_p (\framevect I r_{p_-}- \framevect I r_+) \times
                (\framevect I v_{p_-}- \framevect I v_+)
         \right)}\;+ \\
  & \MxV {\inertia_p} {\framevect B \omega_{p_-}}\;+ \\
  & \MxV {\tmat I B_p}
         {\left(
            m_c (\framevect I r_{c_-}- \framevect I r_+) \times
                (\framevect I v_{c_-}- \framevect I v_+)
         \right)}\;+ \\
  & \MxV {\tmat I B_p}
         {{\tmatT I B_c} (\MxV {\inertia_c} {\framevect B \omega_{c_-}})}
\end{aligned}
\end{equation}

The post-attachment orientation post-attachment angular momentum in this frame results from rotation only:

\begin{equation}
\framevect {B_p} L_+ =
  \MxV {\inertia_+} {\framevect B \omega_{p_+}}
\end{equation}

Equating the pre- and post-angular momenta to conserve angular momentum
and solving for the post-attachment angular velocity yields

\begin{equation}
\framevect B \omega_{p_+} = \MxV{{\inertia_+}^{-1}}{\framevect {B_p} L_-}
\end{equation}

Detachment is much simpler. The only state that needs to change is that
of the parent composite body. The child body's state does not change with
detachment; it simply flies away from the parent with no instantaneous
change in any state element.
The parent body's state is almost correct. The parent similarly flies away
from the child with no instantaneous changes in any of the primitive
states. The composite state is incorrect and needs to be updated.
Any primitive state can be used as the basis for this update.
The mathematics that underlies state update is described in
section~\ref{sec:math_state_prop}.

\subsection{State Initialization}\label{sec:math_state_initialization}
The model participates in the state initialization process by setting state
elements and propagating the set states throughout the DynBody tree.
The mathematics regarding this state propagation is described in
section~\ref{sec:math_state_prop}.

\subsection{Force and Torque Collection}\label{sec:math_collect}

The total effector force acting on a parent body is calculated per the
superposition principle:
\begin{equation}
 \framevect {P_S} F_{\text{eff,tot}} =
 \sum_{\text{vector\ }i} \framevect {P_S} F_{\text{eff,i}} +
 \sum_{\text{child\ }i} \MxV {\tmatT {P_S}{C_S}}
                             {\framevect {C_S} F_{\text{eff,tot}}}
\end{equation}
where $P_S$ is the parent body's structural frame,
the first sum is over the collected effector forces for the parent body,
and the second sum is over the child bodies attached to the parent body.

The total environmental force is calculated similarly while the total
non-transmitted force omits the sum over the child bodies. The total force on a
root body is the sum of the effector, environmental, and non-transmitted forces
on that body.
\begin{equation}
 \framevect {S} F_{\text{tot}} =
 \framevect {S} F_{\text{eff,tot}} +
 \framevect {S} F_{\text{env,tot}} +
 \framevect {S} F_{\text{non-xmit,tot}}
\end{equation}

The equations of motion need the force in the inertial frame:
\begin{equation}
  \framevect I F_{\text{tot}} =
  \MxV{\tmatT I S}
      {\framevect S F_{\text{tot}}} \label{eqn:inertial_force}
\end{equation}

Torque calculations proceed similarly, but with the additional twist that
non-centerline forces produce a torque. The total effector torque acting on
a parent body is
\begin{equation}
 \framevect {P_S} \tau_{\text{eff,tot}} =
 \sum_{\text{vector\ }i} \framevect {P_S} \tau_{\text{eff,i}} +
 \sum_{\text{child\ }i}
   \MxV {\tmatT {P_S}{C_S}}
        {\framevect {C_S} \tau_{\text{eff,tot}}} +
   \framerelvect {P_S} r {P_{cm}} {C_{cm}} \times
     \left(
       \MxV {\tmatT {P_S}{C_S}}
            {\framevect {C_S} F_{\text{eff,tot}}}
     \right)
\end{equation}

The total environmental torque is calculated similarly while the total
non-transmitted torque omits the sum over the child bodies.
The total torque on a root body is the sum of the effector, environmental, and
non-transmitted torques on that body.
\begin{equation}
 \framevect {S} \tau_{\text{tot}} =
 \framevect {S} \tau_{\text{eff,tot}} +
 \framevect {S} \tau_{\text{env,tot}} +
 \framevect {S} \tau_{\text{non-xmit,tot}}
\end{equation}

The equations of motion need the torque in the body frame:
\begin{equation}
  \framevect B \tau_{\text{tot}} =
  \MxV{\tmat S B}
      {\framevect S \tau_{\text{tot}}} \label{eqn:body_torque}
\end{equation}

\subsection{Equations of Motion}\label{sec:math_eom}

Equation~\eqref{eqn:inertial_force} combined with Newton's second law yields
the acceleration due to non-gravitational forces:
\begin{equation}
  \framevect I a_\text{{non-grav,tot}} = \frac 1 m \framevect I F_{\text{tot}}
\end{equation}

The \GRAVITY computes the apparent gravitational acceleration,
the total gravitational acceleration acting on the body
less the gravitational acceleration of the integration frame.
Combining this with the non-gravitational acceleration yields the total
acceleration of the body with respect to the integration frame.

\begin{equation}
  \framerelvect I a I B =
    \framevect I a_\text{{non-grav,tot}} + \framevect I a_\text{{grav,tot}}
  \label{eqn:atot}
\end{equation}

The rotational equations of motion differ conceptually from the above
translational equations of motion in two regards.
That angular momentum is expressed in the body frame adds the complicating
factor of computing derivatives in a rotating frame.
That is offset by lack of gravitation as a specific concern,
which simplifies the the rotational equations of motion a bit.

The rotational analog of Newton's second law
states that the time derivative of the angular momentum vector as expressed in
a non-rotating frame is simply the external torque:

\begin{equation}
  \framevdot I L = \framevect I \tau_{\text{tot}}
\end{equation}

The angular momentum is readily computed in the body frame as the product
of the inertia tensor and the body-referenced angular velocity:

\begin{equation}
  \framevect B L = \MxV {\inertia_B} {\framevect B \omega}
\end{equation}

The time derivative of the body-referenced angular momentum and the
inertial-referenced angular momentum are related by the inertial torque:

\begin{align}
  \framevdot B L &=
    \MxV {\tmat I B}{\framevdot I L}
    - \framerelvect B \omega I B \times\framevect B L \nonumber \\
  &= \framevect B \tau_{\text{tot}}
    - \framerelvect B \omega I B \times\framevect B L
\end{align}

Assuming the inertia tensor is constant (or nearly so) in the body frame,
\begin{align}
  \framevdot B L &=
  \frac{d}{dt}\left(\MxV {\inertia_B} {\framevect B \omega}\right)
  \nonumber \\
  &\approx \MxV {\inertia_B} {\framevdot B \omega}
\end{align}

Combining the above yields the time derivative of the
body-referenced angular momentum:

\begin{equation}
  \framevdot B \omega =
    {\inertia_B}^{-1}
    \left(
      \framevect B \tau_{\text{tot}} -
      \framerelvect B \omega I B \times\framevect B L
    \right)
  \label{eqn:wdotbody}
\end{equation}

\subsection{State Integration}\label{sec:math_state_integ}
This model uses the \hypermodelref{INTEGRATION} to perform the state integration.
See the documentation for that model regarding the mathematics that
underlies numerical integration.

\subsection{State Propagation}\label{sec:math_state_prop}
The initialized state and the integrated state need to be propagated
throughout a DynBody object, including to other DynBody objects attached
to the object in question. The propagation from one frame to another must
be consistent with the local reference frames embodied in the MassBasicPoints
that are associated with the DynBodyFrames.

The model performs state propagation in terms of pairs of reference frames.
The mathematics of the propagation of the state from frame B to frame C
depends on whether the underlying MassBasicPoint is expressed as B to C
or C to B. Different methods address these as forward and reverse propagations.

Forward propagation involves computing the state of frame C with respect to
frame A (the integration frame) given the state of frame B with respect to
frame A and the local transformation from frame B to frame C. Note that this
local transformation is a part of the MassBasicPoint object and hence does
not contain derivative data. This local transformation is assumed to be
constant with respect to time.

\begin{align}
\tmat A C &=
  \MxV {\tmat B C} {\tmat A B}
  && \text{Orientation} \\
\framerelvect C {\omega} A C &=
  \MxV {\tmat B C} {\framerelvect B {\omega} A B}
  && \text{Angular velocity} \\
\framerelvect A r A C &=
  \framerelvect A r A B + \MxV {\tmatT A B} {\framerelvect B r B C}
  && \text{Position} \\
\framerelvect A v A C &=
  \framerelvect A v A B + \MxV {\tmatT A B}
                               {(\framerelvect B {\omega} A B \times
                                 \framerelvect B r B C)}
  && \text{Velocity}
\end{align}

Reverse propagation involves computing the state of frame C with respect to
frame A (the integration frame) given the state of frame B with respect to
frame A and the local transformation from frame C to frame B. In other words,
the local transformation is in the reverse direction of the desired
propagation.

\begin{align}
\tmat A C &=
  \MxV {\tmatT C B} {\tmat A B}
  && \text{Orientation} \\
\framerelvect C {\omega} A C &=
  \MxV {\tmatT C B} {\framerelvect B {\omega} A B}
  && \text{Angular velocity} \\
\framerelvect A r A C &=
  \framerelvect A r A B - \MxV {\tmatT A C} {\framerelvect C r C B}
  && \text{Position} \\
\framerelvect A v A C &=
  \framerelvect A v A B - \MxV {\tmatT A C}
                               {(\framerelvect C {\omega} A C \times
                                 \framerelvect C r C B)}
  && \text{Velocity}
\end{align}

\subsection{Point Acceleration}\label{sec:math_pt_accel}
The following assumes a rigid body connection between the composite center of
mass and a vehicle point. Different mathematics are needed if this assumption is
not valid.

With this assumption, the rotational acceleration at the vehicle point is equal
to the rotational acceleration at the center of mass. The point however has its
own reference frame, and the rotational acceleration needs to be expressed
in this point frame.

\begin{equation}
\framerelvdot {\text{pt}} {\omega} I {\text{pt}} =
\MxV {\tmat B {\text{pt}}}
     {\framerelvdot B {\omega} I B}
\end{equation}

With the rigid body assumption, the total translation acceleration at the point
is the total translation at the center of mass plus the translational
acceleration due to rotational motion. In the rotating composite body frame,
this acceleration is

\begin{equation}
\Delta \framevect B {a_{\text{rot}}} =
  \framerelvect B {\omega} I B \times
  (\framerelvect B {\omega} I B \times \framerelvect B r B {\text{pt}}) +
  \framerelvect B {\omega} I B \times \framerelvect B r B {\text{pt}}
\end{equation}

Transforming to inertial,
\begin{equation}
\Delta \framevect I {a_{\text{rot}}} =
  \MxV {\tmatT I B} {\Delta \framevect B {a_{\text{rot}}}}
\end{equation}

The total acceleration at the point is
\begin{equation}
\framevect I {a_{\text{pt}}} =
  \framevect I a + \Delta \framevect I {a_{\text{rot}}}
  \label{eqn:tot_accel_pt}
\end{equation}

Calculating the non-gravitational acceleration is a bit more complex. The total
acceleration at the point comprises gravitational and non-gravitational
components, as does the total acceleration at the composite center of mass.
\begin{align}
\framevect I {a_{\text{pt}}} &=
  \framevect I {a_{\text{pt,non-grav}}} +
  \framevect I {a_{\text{pt,grav}}}
  \label{eqn:tot_accel_pt_split} \\
\framevect I {a_B} &=
  \framevect I {a_{B,\text{non-grav}}} +
  \framevect I {a_{B,\text{grav}}}
  \label{eqn:tot_accel_B_split}
\end{align}

The gravitational acceleration at the point is that at the center of mass plus a
delta due to the separation between the center of mass and the point. Assuming a
small separation between the center of mass and the point, the gravity gradient
yields an excellent estimate of this delta gravitational acceleration.

\begin{align}
\framevect I {a_{\text{pt,grav}}} &=
  \framevect I {a_{B,\text{grav}}} +
  \Delta \framevect I {a_{pt,\text{grav}}} \\
\Delta \framevect I {a_{\text{pt,grav}}} &\approx
  \MxV {\nabla \framevect I {a_{B,\text{grav}}}}
       {(\MxV {\tmatT I B} {\framerelvect B r B {\text{pt}}})}
  \label{eqn:delta_grav_accel_pt}
\end{align}

Combining equations~\ref{eqn:tot_accel_pt} to~\ref{eqn:delta_grav_accel_pt}
and solving for the non-gravitational acceleration at the point,

\begin{equation}
\framevect I {a_{\text{pt,non-grav}}} =
  \framevect I {a_{B,\text{non-grav}}} +
  \Delta \framevect I {a_{\text{rot}}} -
  \Delta \framevect I {a_{\text{pt,grav}}}
\end{equation}
