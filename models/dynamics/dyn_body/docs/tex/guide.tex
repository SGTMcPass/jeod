\chapter{User Guide}\hyperdef{part}{user}{}\label{ch:user}
This chapter describes how to use the \ModelDesc from the
perspective of a simulation user, a simulation developer,
and a model developer.

\section{Analysis}

\subsection{Initialization}
Some (a rather small portion) of a DynBody object's data members need to
be initialized directly in the simulation input file. At a minimum,
the object itself must be given a name,
the object's integration frame must be specified by name, and
flags need to be set to enable vehicle dynamics.
Assuming a sim object
named \verb+veh_object+ contains a DynBody element named \verb+body+,
The following will prepare the body for a 6 DOF Earth orbit simulation.
One caveat: Please do not copy the following verbatim into your input file.
A better name for the vehicle than "vehicle" most likely exists,
and using Earth-centered inertial for a vehicle that is not orbiting the
Earth is not a good choice for integration frames.

\begin{verbatim}
  veh_object.body.set_name( "vehicle" );
  veh_object.integ_frame_name = "Earth.inertial";
  veh_object.translational_dynamics = true;
  veh_object.rotational_dynamics = true;
\end{verbatim}

Gravity is active in most simulations. The gravity controls for the
vehicle need to be specified in the input file. For example, the following
will make the body subject to gravitational influences from the Earth, Sun,
and Moon. Earth gravity uses an $8\times8$ gravity field while the Sun
and Moon are point sources.

\begin{verbatim}
  veh_object.body.grav_interaction.n_grav_controls = 3;
  veh_object.body.grav_interaction.grav_controls = alloc(3);
  \end{verbatim}
\clearpage
\begin{verbatim}
  // Earth gravity
  veh_object.body.grav_interaction.grav_controls[0].source_name = "Earth";
  veh_object.body.grav_interaction.grav_controls[0].active = true;
  veh_object.body.grav_interaction.grav_controls[0].spherical = false;
  veh_object.body.grav_interaction.grav_controls[0].degree = 8;
  veh_object.body.grav_interaction.grav_controls[0].order = 8;
  veh_object.body.grav_interaction.grav_controls[0].gradient = true;

  // Sun gravity
  veh_object.body.grav_interaction.grav_controls[1].source_name = "Sun";
  veh_object.body.grav_interaction.grav_controls[1].active = true;
  veh_object.body.grav_interaction.grav_controls[1].spherical = true;
  veh_object.body.grav_interaction.grav_controls[1].gradient = false;

  // Moon gravity
  veh_object.body.grav_interaction.grav_controls[2].source_name = "Moon";
  veh_object.body.grav_interaction.grav_controls[2].active = true;
  veh_object.body.grav_interaction.grav_controls[2].spherical = true;
  veh_object.body.grav_interaction.grav_controls[2].gradient = false;
\end{verbatim}

See the \hypermodelref{GRAVITY} for a detailed description of the
GravityInteraction class.

\subsection{Logging}
Several of the data members in a DynBody object, either directly or
indirectly through the attribute MassBody, are amenable to logging. For example,
adding the following to a recording group's reference specification
will log a vehicle's mass, the position of its composite center of mass
with respect to structure, and the composite state:

\begin{verbatim}
  veh_object.body.composite_properties.mass
  veh_object.body.composite_properties.position[0-2]
  veh_object.body.composite_body.state.trans.position[0-2]
  veh_object.body.composite_body.state.trans.velocity[0-2]
  veh_object.body.composite_body.state.rot.Q_parent_this.scalar
  veh_object.body.composite_body.state.rot.Q_parent_this.vector[0-2]
  veh_object.body.composite_body.state.rot.ang_vel_this[0-2]
\end{verbatim}

That said, logging the quaternion often is not particularly insightful.
Tools outside of the \ModelDesc such as the EulerDerivedState class provide
insight into a DynBody object's state. That class is a part of the
\hypermodelref{DERIVEDSTATE}. Users are strongly encouraged to make use of that
model as an analysis tool.

\section{Integration}
\subsection{\Sdefine-Level Integration}
Given the complexity of the \ModelDesc, integrating it into an \Sdefine file
is a relatively simple task.

The first step is to declare a DynBody derivative as an element of some
simulation object. At the very minimum, the object needs to be prepared for 
initialization in order to be used during simulation:
\begin{verbatim}
##include "dynamics/dyn_body/include/dyn_body.hh"
##include "dynamics/dyn_manager/include/dyn_manager.hh"
...
class VehicleSimObject : public Trick::SimObject
{
  public:
    jeod::DynBody dyn_body;
    ...
    VehicleSimObject( jeod::DynManager& dm )
      : dyn_manager(dm)
    {
      P_ENV ("initialization") dyn_body.initialize_model( dyn_manager );
    };
    ...
};
\end{verbatim}
Note well: The above \verb+DynBody::initialize_model()+ call must take place 
(in time) between the call to the dyn\_manager's \verb+DynManager::initialize_model()+ 
method and \verb+DynManager::initialize_simulation()+ method.
This constraint is best satisfied using Trick's job prioritization scheme, as 
shown, using the "P\_ENV" phase defined in 
"lib/jeod/JEOD\_S\_modules/default\_priority\_settings.sm."

Real vehicles change in mass as they burn fuel. To make the state integration
reflect reality the vehicle's mass mass properties need to be updated,
typically at the simulation's dynamic rate:
\begin{verbatim}
...
#define DYNAMICS 0.005
...
class VehicleSimObject : public Trick::SimObject
{
  public:
    jeod::DynBody dyn_body;
    ...
    VehicleSimObject( jeod::DynManager& dm )
      : dyn_manager(dm)
    {
      P_ENV ("initialization") dyn_body.initialize_model( dyn_manager );
      (DYNAMICS, scheduled) dyn_body.mass.update_mass_properties();
    };
    ...
};
\end{verbatim}

Surprisingly, the above two calls will be the only \Sdefine-level calls that
directly invoke model functionality even in a complex simulation.
Not so surprisingly, a lot of the functionality is invoked by other models.
The above do not initialize state, compute state derivatives, integrate state,
or analyze state. Initialization is performed by the \hypermodelref{DYNMANAGER}
and the \hypermodelref{BODYACTION}.
The \DYNMANAGER guides derivative calculation, including calls to the
\hypermodelref{GRAVITY} to compute gravitational acceleration and to
this model's force/torque collection method.
The \DYNMANAGER similarly drives the state integration process.
Finally, the \hypermodelref{DERIVEDSTATE} provides a number of analysis tools.

The vehicle as of yet does not have any forces or torques acting on it.
The modeling of forces and torques is beyond the scope of the \ModelDesc.
Using those modeled forces and torques is the bread and butter of this model.
The object merely needs to be told that those forces and torques exist.
This is accomplished by using the Trick \emph{vcollect} mechanism.
\begin{verbatim}
  vcollect veh_object.dyn_body.collect.collect_environ_forc jeod::CollectForce::create {
    veh_object.force_model.force,
    veh_object.another_force_model.force
  };
\end{verbatim}

Note that the above collects the listed forces into the
\verb+collect_environ_forc+ data member. Two other options exist for forces:
The \verb+collect_effector_forc+ and \verb+collect_no_xmit_forc+.
Torques are collected similarly into one of three groups; just use `\_torq'
instead of `\_forc' for the data member name and CollectTorque::create instead
of CollectForce::create.

There is little difference between the environmental and effector forces, but
there is a huge distinction between these first two and the non-transmitted forces.
When a DynBody object is attached as a child to a parent DynBody
object, the environmental and effector forces acting on a child body are
transmitted to the parent body. The non-transmitted forces are not transmitted
to the parent (hence the name).

So what are these non-transmitted forces and torques? The concept does not
make a lick of sense from the perspective of physics. The reason for classifying
a force or torque as `non-transmitted' is that the force or torque does not
make a lick of sense when a body is attached to another body.
A couple of examples are aerodynamic drag and gravity gradient torque.

A similar collect mechanism exists for wrenches:
\begin{verbatim}
  vcollect veh_object.dyn_body.effector_wrench_collection.collect_wrench {
    &veh_object.wrench_model.wrench,
    &veh_object.another_wrench_model.wrench
  };
\end{verbatim}
As of this release, there are no environmental or non-transmitted wrenches,
and wrenches can only be used with structure-integrated dynamic bodies.

When two bodies combine to form a composite, the nearby presence of one vehicle
can significantly alter the aerodynamic drag on the other vehicle. The best
thing to do is to use a different drag model that truly represents the combined
vehicle. Lacking this best solution, a fairly good approximation can sometimes
be attained by using the drag on the larger vehicle only. A simple way to
accomplish this is to make the smaller vehicle attach to the larger as a child
and denote the aerodynamic drag on the smaller vehicle as a non-transmitted
force. Denoting aerodynamic drag and torque as non-transmitted forces and
torques is a widespread practice. Be aware that this is a partial solution.
The best thing to do is to mark these as non-transmitted \emph{and} to switch the
parent body to a different drag model (\eg a composite plate model).

Gravity gradient torque is calculated using the composite mass properties of
a DynBody object. This means that the gravity gradient torque calculated for the
root body of a composite body automatically includes the contributions of all
attached bodies; the result is correct \emph{as-is}. Adding the gravity gradient
torque exerted on child bodies would invalidate this already correct result.
Gravity gradient torque must be categorized as a non-transmitted torque.

\subsection{Using the Model in a User Defined Model}
All of the public member data and member functions of the DynBody
 and MassBody classes are accessible to user models that operate
on a DynBody object. Fully explaining that functionality is beyond
the scope of this section of the document. Users who wish to do so should
consult section~\ref{ch:spec} of this document, the doxygen documentation
supplied with JEOD (the HTML pages are vastly superior to the PDF documents),
the header files that define the classes, and if worse comes to worse,
the code itself.

A few methods are worthy of note and are described in
table~\ref{tab:important_user_methods}.

\begin{table}[htp]
\centering
\caption{Methods of Note}
\label{tab:important_user_methods}
\vspace{1ex}
\begin{tabular}{||l|p{0.5\linewidth}|}\hline
{\bf Method} & {\bf Description} \\ \hline \hline
    \begin{tabular}[t]{@{}l@{}}
        \verb+MassBody::attach_to()+ \\
        \verb+MassBody::attach_child()+ \\
        \verb+DynBody::attach()+ \\
        \verb+DynBody::add_mass_body()+ 
    \end{tabular} &
  If you as a model developer know that two bodies are to be attached,
  and know they are to be attached immediately, invoke the attach
  method directly in your model code. There is no reason to create and enqueue
  a body action to perform the attachment. \\[6 pt]
    \begin{tabular}[t]{@{}l@{}}
        \verb+MassBody::detach()+ \\
        \verb+DynBody::detach()+ \\
        \verb+DynBody::remove_mass_body()+ 
    \end{tabular} &
  Detach is similar to attach in the sense that if you as a model developer
  know that one body is to detach from another, immediately, simply invoke
  the detach method in your model code. \\[6 pt]
\verb+DynBody::set_state()+ &
  You need to override state in a distributed multi-vehicle simulation?
  This method, or one of its kin (e.g., \verb+set_attitude_rate()+), is the 
  method to call. \\[6 pt]
\verb+DynBody::propagate_state()+ &
  Always make sure to end a sequence of state overrides with a call to
  \verb+propagate_state()+ to pass the updated state along to child bodies. 
  Failure to do so will result in states inconsistent
  with geometry. \\
\hline
\end{tabular}
\end{table}

\section{Extension}
The model was designed with extensibility in mind. The DynBody
class makes simplifying assumptions that are not appropriate in all
circumstances, particularly launch. One solution is to write a user-defined
extension to the DynBody class that bypasses these assumptions by augmenting
functionality.
