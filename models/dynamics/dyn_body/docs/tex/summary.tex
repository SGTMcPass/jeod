%%%%%%%%%%%%%%%%%%%%%%%%%%%%%%%%%%%%%%%%%%%%%%%%%%%%%%%%%%%%%%%%%%%%%%%%%%%%%%%%
%
% Purpose: Integration model executive summary
%
%
%%%%%%%%%%%%%%%%%%%%%%%%%%%%%%%%%%%%%%%%%%%%%%%%%%%%%%%%%%%%%%%%%%%%%%%%%%%%%%%%

\chapter*{Executive Summary}

The \ModelDesc, a component of the dynamics suite of models within \JEODid,
is located at models/dynamics/dyn\_body.

The primary purpose of the model is to represent dynamic aspects of
spacecraft in a simulation.
The model's DynBody class is the base class for such representations.
A DynBody has properties such as mass that are intrinsic to the body itself
as well as extrinsic properties such as the state of the body with
respect to the outside world. External forces and torques act to change
the body's state over time.

\section*{Key Concepts}

\subsection*{Mass}
Many core DynBody functions require interactions between mass properties.
Each DynBody contains a MassBody member instance. However not all MassBody's
belong to a DynBody. For this reason, every MassBody acting as a container for
a DynBody's properties has an immutable pointer to the DynBody owner.
MassBody objects have mass properties and can be attached to/detached from
other MassBody objects. A basic MassBody object is stateless and lacks a
connection to the dynamic world. The DynBody class adds that connection to a
MassBody object.

\subsection*{State}
The DynBody class makes the connection to the outside world in the form of
`state'.  Multiple states, in fact. The \ModelDesc uses a derivative of the
RefFrame class, the BodyRefFrame class, to represent state. All active
reference frames in a simulation, including those associated with a DynBody
object, are connected to one another to form the simulation's reference frame
tree.

\subsection*{Integration Frame}
The connection between a DynBody object's DynBodyFrame objects and the outside
world is through the DynBody object's integration frame. The integration frame
must be one of the simulation's inertial frames. All of the DynBodyFrame objects
associated with a given DynBody object share the DynBody object's integration
frame as a common parent.

Two driving requirements for JEOD were to provide
\begin{inparaenum}[(1)]
\item the ability for different vehicles in a multi-vehicle simulation to have
their states represented in and integrated in different frames of reference and
\item the ability to switch
integration frames for a given vehicle during the course of a simulation run.
\end{inparaenum}
The \ModelDesc accomplishes both of these objectives by enabling each root
DynBody object to
have its own integration frame and providing tools to change this frame
dynamically in a manner that is consistent with the laws of physics.

\subsection*{Attach/Detach}
MassBody objects can be attached to/detached from one another. DynBody objects
may also attach/detach to one another. MassBody subassemblies may also attach
to a DynBody object, granted that they do not attach to another DynBody or its
subassemblies. A DynBody may not attach as a subcomponent of a MassBody,
because a DynBody has a state, while the parent MassBody does not. This model
of capability is essential for modeling the docking and undocking of space
vehicles and spacecraft component. The composite mass properties change when
objects attach or detach. The composite mass properties aggregate recursively.
For example, if A is attached to B is attached to C, then A's composite
properties reflect only itself, B's composite properties reflect the
combination of A+B,and C's composite properties reflect the combined A+B+C.
Much more also happens when DynBody objects attach and detach, including
conservation of momenta. Vehicle states change, and these changes must be
performed in a manner consistent with the laws of physics.

Additionally, DynBody objects may be kinematically attached to any reference frame as a means of holding position 
relative to that frame. In this case, the mass tree is unaffected. One use case for this capability is when a vehicle 
is at a launchpad and should remain fixed positionally and rotationally w.r.t. the planet-fixed frame.

\subsection*{Initialization}
Initializing a DynBody object is a rather complex process. The zeroth stage
of the initialization process occurs when the object is constructed.
The next stage of initialization involves preparing the object for the truly
complex stages that follow. The body is registered with the dynamics manager,
as are the standard body reference frames associated with the body. The
body's integration frame is set and the standard body reference frames are
linked to this integration frame. Note that while the links between the
DynBody object are established in this phase, much work remains to be done.
The body is unattached and its mass properties, mass points/vehicle points, and
states remain uninitialized at this stage.

These yet-to-be initialized aspects of a DynBody must be initialized in the
proper order. \textbf{The recommended approach is to first initialize the mass
properties and mass points of all MassBody objects, then perform any
initialization-time attachments, and finally initialize state.} This is
the approach taken when the body action-based initialization scheme is used
to initialize mass bodies and dynamic bodies. Deviating may result in improper
initial states due to the aggregating nature of composite vehicle attachments.

\subsection*{State Initialization}
Properly initializing the state of all of the vehicles in a simulation
has been an ongoing and rather vexing issue. JEOD approaches this problem by
spreading the responsibility for state initialization over several models.

\subsection*{Forces and Torques}
In addition to the acceleration due to gravity,
a spacecraft can be subject to many external forces and torques.
For example a spacecraft's thrusters, atmospheric drag, and radiation pressure
exert forces and torques on the vehicle. The source and nature of these forces
and torques is more or less irrelevant to this model.
Forces and torques are each categorized as effector, environmental, and
non-transmitted. This categorization is external to this model; all this model
sees is a set of three Standard Template Library (STL) vectors of
CollectForces and a set of three STL vectors of CollectTorques. Each set refers
 to loads of alike nature: effector, environmental, and non-transmitted.
A more in-depth explanation follows.

\subsection*{Collect Mechanism}
The above discussion on forces and torques ignored how the STL collect vectors
are formed. How these vectors are populated is not an issue of concern in the
narrow view of the DynBody class taken by itself. From the perspective of the
DynBody class, those vectors are populated by some external agent.
This is a concern to the model taken as a whole, and even more importantly,
it is a concern to the user.

JEOD and Trick developers worked together to create the \emph{vcollect}
mechanism. The mechanism takes advantage of capabilities provided with the C++
language, and is easy to implement outside of the Trick environment.

\subsection*{Gravity}
The above discussion on forces intentionally skirted over the issue of
gravitational acceleration. JEOD has always treated gravitation as a topic
conceptually distinct from the other forces that act on a dynamic body.
One reason for doing so is that acceleration (rather than force) is the
natural output of any gravity model, including the JEOD \GRAVITY.
Acceleration (rather than force) is what is ultimately needed to formulate the
equations of motion.

More importantly, gravitation truly is distinct from the external forces.
Unlike any other force, the gravitational force acting on an object cannot be
detected directly by any device. An accelerometer, for example, measures the
acceleration due to the net non-gravitational force. Even though JEOD assumes
Newtonian physics, it is a good idea to look at gravitation from the
perspective of the more accurate general relativistic point of view.
Gravitation is not a force in general relativity. It is something quite
different from forces and is best treated as such.

\subsection*{Equations of Motion}
The external forces and gravitational acceleration form the basis for the
translational equations of motion for a dynamic body, while the external torques
form the basis for the body's rotational equations of motion. Those equations
of motion are second order differential equations; they must be integrated over
time.

\subsection*{State Integration}
This state integration must be performed numerically due to the varying and
complex nature of the forces and torques. This model uses the \INTEGRATION to
perform the numerical integration.

The model only integrates the root body of a tree of attached DynBody objects
and for that body, integrates only one reference frame. This is the body's
integrated frame. Exactly which frame is integrated is a decision made by a
class that derives from the DynBody class. The DynBody class integrates the composite body frame.
The StructureIntegratedDynBody class (introduced with JEOD 3.4) integrates the
root body's structural frame.
Users can define extended
classes that integrate the root body through alternative means.

\subsection*{State Propagation}
A DynBody contains multiple reference frames. All of these frames must reflect
the current state of the vehicle. Since only one frame is integrated, the other
frames must be made consistent with this integrated frame. This is performed
by propagating the results of the integration to all frames associated
with the DynBody object, including the DynBody objects attached as children
to the root DynBody object. This propagation is performed using the geometry
information embodied in the assembly of DynBody's and attached MassBody's.

\section*{Interactions}

\subsection*{JEOD Models Used by the \ModelDesc}
The \ModelDesc uses the following JEOD models:
\begin{itemize}
\item\hypermodelref{DYNMANAGER}. The \ModelDesc uses the \DYNMANAGER as
  \begin{inparaenum}[(1)]
  \item a name register of DynBody objects,
  \item a name register and subscription manager of RefFrame objects, and
  \item a name register and validator of integration frames.
  \end{inparaenum}
\item\hypermodelref{MASS}. The \ModelDesc uses the \MASS through friendship.
The DynBody class \hasa MassBody. Friendship does not extended to derivations
 inheriting from DynBody, protecting its core functionality.
\item\hypermodelref{GRAVITY}. The \ModelDesc relies on the \GRAVITY to
  compute gravitational acceleration. To accomplish this end, each DynBody
  object contains an object that indicates which gravitational bodies in the
  simulation have an influence on the DynBody object.
\item\hypermodelref{INTEGRATION}. The \ModelDesc uses the \INTEGRATION to
  perform the state integration. This model merely sets things up for the
  \INTEGRATION so that it can properly perform the integration. The \ModelDesc
    also uses er7\_utils::IntegrableObject through inheritence. The DynBody
    class \isa er7\_utils::IntegrableObject.  In addition, \ModelDesc also
    provides the capability to associate other instances of IntegrableObjects
    to be integrated along with the state of the \ModelDesc. This capability
    is demonstrated by associating IntegrableObjects corresponding to thermal
    states of surface facets in verif/SIM\_3\_ORBIT\_1st\_ORDER\_T10 of the
    radiation\_pressure model.
\item\hypermodelref{MATH}. The \ModelDesc uses the \MATH to operate on
  vectors and matrices.
\item\hypermodelref{MEMORY}. The \ModelDesc uses the \MEMORY to allocate
  and deallocate memory.
\item\hypermodelref{MESSAGE}. The \ModelDesc uses the \MESSAGE to generate
  error and debug messages.
\item\hypermodelref{NAMEDITEM}. The \ModelDesc uses the \NAMEDITEM to
  construct names linked to the associated MassBody name.
\item\hypermodelref{QUATERNION}. The \ModelDesc uses the \QUATERNION
  to operate on the quaternions embedded in the RefFrame objects and to
  compute the quaternion time derivative.
\item\hypermodelref{REFFRAMES}. The \ModelDesc makes quite extensive use of the
  \REFFRAMES. A DynBody's integration frame is a RefFrame object, and each of
  the body-based reference frames contained in a DynBody object is a
  BodyRefFrame object. The BodyRefFrame \isa RefFrame. The \ModelDesc uses
  the \REFFRAMES functionality to compute relative states.
\item\hypermodelref{SIMINTERFACE}. All classes defined by the \ModelDesc use
  the\\
  \verb+JEOD_MAKE_SIM_INTERFACES+ macro defined by the \SIMINTERFACE to provide
  the behind-the-scenes interfaces needed by a simulation engine such as Trick.
\end{itemize}


\subsection*{Use of the \ModelDesc in JEOD}
The following JEOD models use the \ModelDesc:
\begin{itemize}
\item\hypermodelref{BODYACTION}. The \BODYACTION provides several classes that
  operate on a DynBody object. The DynBodyInit class and its derivatives
  initialize a DynBody object's state. The DynBodyFrameSwitch class provides
  a convenient mechanism for switching a DynBody object's integration frame.
  The BodyAttach and Detach classes indirectly operate on a DynBody object
  through the \ModelDesc attach/detach mechanisms.
\item\hypermodelref{DERIVEDSTATE}. The primary purpose of the \DERIVEDSTATE
  is to express a DynBody state in some other form.
\item\hypermodelref{DYNMANAGER}. The \DYNMANAGER drives the integration of
  all DynBody objects registered with the dynamics manager.
\item\hypermodelref{GRAVITY}. The \GRAVITY computes the gravitational
  acceleration experienced by a DynBody object.
\item\hypermodelref{GRAVITYTORQUE}. The \GRAVITYTORQUE computes the gravity
  gradient torque exerted on a DynBody object, with the resultant torque
  expressed in the DynBody object's structural frame.
\end{itemize}
