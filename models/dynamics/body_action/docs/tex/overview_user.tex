%%%%%%%%%%%%%%%%%%%%%%%%%%%%%%%%%%%%%%%%%%%%%%%%%%%%%%%%%%%%%%%%%%%%%%%%%%%%%%%%
% overview_user.tex
% Toplevel user guide for the Body Action Model
%
%%%%%%%%%%%%%%%%%%%%%%%%%%%%%%%%%%%%%%%%%%%%%%%%%%%%%%%%%%%%%%%%%%%%%%%%%%%%%%%%

%----------------------------------
\chapter{User Guide}\hyperdef{part}{user}{}
\label{ch:overview:user}
%----------------------------------

This chapter provides a top-level description of the use of the \ModelDesc.
Details on how to use specific classes are described in the User Guides for
the various sub-models.

\section{Analysis}
How the \ModelDesc is to be used depends on which of the above options
was employed by the simulation integrator. This discussion assumes a
setup along the lines described in section~\ref{subsec:recommended_practice}.
The simulation user selects those capabilities from
in the S\_define file that best represent the needs of the scenario
to be simulated.

For example, consider the case of investigating the behavior of a chaser
vehicle in close proximity to a target vehicle. The S\_define file
provides multiple mechanisms for initializing the chaser and target.
The simulation user selects a set that enables specifying
\begin{itemize}
\item The target vehicle's mass properties and points of interest,
\item The target vehicle's translational state with some given inertial values,
\item The target vehicle's rotational state relative to the target
vehicle's LVLH frame,
\item The chaser vehicle's mass properties and points of interest, and
\item The chaser vehicle such in terms of the relative state between a pair of
points of interest on the chaser and target.
\end{itemize}

For each selected item, the user must place statements in the
simulation input that
\begin{itemize}
\item Populate the data items for the object. \\
\item Point the simulation body action pointer to this model object. \\
\item Call the {\tt add\_body\_action()} simulation job to add the
object to the Dynamic Manager's list of model objects.
\end{itemize}

Four mistakes that the user can make in this process are
\begin{itemize}
\item Failing to specify the subject body for each model object.
The JEOD methods do not know that the simulation integrator has made the
subject body and the model object parts of the same simulation object.
\item Attempting to reuse model objects. The {\tt add\_body\_action()}
method must not (and cannot) create a copy of the object passed to it.
\item Creating a circular set of dependencies. For example,
specifying the state of vehicle A with respect to vehicle B and
specifying the state of vehicle B with respect to vehicle A.
\item Failing to specify all needed states.
\end{itemize}
The JEOD detects and reports these errors.

\section{Integration}

The obvious, but wrong, approach to using the \ModelDesc in a Trick
simulation is to
\begin{itemize}
\item Define data elements of the appropriate model classes
for each vehicle simulation object in the simulation S\_define file and
\item Explicitly call the {\tt initialize()} and {\tt apply()} methods
for those model objects in the simulation S\_define file.
\end{itemize}

{\em Do not follow the above practice.} This practice circumvents the
flexibility built into the model and will almost certainly lead to
difficulties in the future.

\subsection{The BodyAction list}

The recommended practice regarding the use of the \ModelDesc in a
Trick simulation is to use the model in conjunction with the
simulation's DynManager object. The DynManager maintains a list
of \ModelDesc objects. Calling the DynManager's {\tt add\_body\_action}
adds an item to the list. The DynManager draws objects from this list
at the appropriate time and only does so when the objects indicate
that they are ready to be executed. This mechanism disconnects the
execution order of the enqueued model objects from the order in
which the simulation objects are declared and from the order
in which the objects are added to the list.

This separation between initialization and declaration order
was first addressed in JEOD 1.4/1.5, but the C-based solution
involved some rather convoluted and very inflexible code.
The C++ based JEOD 2.0 solution to this problem makes for
an very extensible and simpler set of code. Users are strongly
recommended to take advantage of this flexibility.

\subsection{A simple simulation}
A simple way to make use of the DynManager's BodyAction list
limit the use of the \ModelDesc to a pre-defined,
but presumably well-vetted, set.
A simulation S\_define file that implements this option will
\begin{itemize}
\item Declare a very specific set of model objects as simulation
object elements for each vehicle in the simulation.
\item Specifically invoke the DynManager {\tt add\_body\_action()}
method on each of these model objects as initialization class jobs.
\item Invoke the DynManager {\tt initialize\_simulation()} method
as an initialize class job that runs after the {\tt add\_body\_action()}
jobs.
\item Optionally invoke the DynManager {\tt perform\_actions()}
as a scheduled job to enable asynchronous use of the model during
the course of the simulation run.
\end{itemize}

This approach makes user input fairly simple. All
the simulation user needs to do is populate the pre-defined set of
model objects with the appropriate values.

The lack of flexibility is the key disadvantage of this option.
To illustrate this, consider a simulation involving a chaser and a target.
Different runs of the simulation are to investigate
\begin{itemize}
\item The launch the chaser vehicle,
\item The transfer of the chaser from orbit insertion to far-field rendezvous,
\item Various proximity operations between the chaser and target, and
\item Chaser de-orbit and entry.
\end{itemize}

\subsection{Multiple initialization capabilities}
The best way to initialize the chaser vehicle's state in these different
scenarios is to use the \ModelDesc state initialization classes best suited
for the scenario. For example,
\begin{itemize}
\item The DynBodyInitNed class to initialization the chaser on the launch pad,
\item The DynBodyInitTransState or DynBodyInitOrbit classes to initialize
the chaser at orbit insertion and prior to entry, and
\item The DynBodyInitLvlh class to initialize the chaser in close proximity
to the target.
\end{itemize}
Each of these different initialization capabilities needs to be declared
in the S\_define file to enable their use in the simulation.
One way to accomplish this end is to declare as simulation object
elements all of the state initializers needed by the various scenarios.

This option continues that used in the previous example.
The simulation S\_define file specifically
invokes the DynManager {\tt add\_body\_action()} method for
each initializer. This solves the issue of different scenarios
requiring different initializers, but introduces a new issue.
The set of state initializers now over-specifies the state.
The input file for a particular scenario will need to deactivate
those state initializers that are not to be used in that scenario.

\subsection{Deferring %
  \texorpdfstring{{\tt add\_body\_action()}}{add\_body\_action()} calls}
\label{subsec:recommended_practice}
An alternative approach to solving this over-specification problem is to
never let it happen. The previous option required a registering
in the S\_define file a call to {\tt add\_body\_action()}
for each state initializers declared in the S\_define file.
This option only registers one such call in the S\_define file.
The S\_define file
\begin{itemize}
\item Declares a potentially broad set of \ModelDesc objects as
simulation object elements for each vehicle in the simulation.
\item Declares a pointer to a BodyAction object in the same sim object
that contains the DynManager object.
\item Defines {\em as a zero-rate scheduled job} a call to the DynManager's
{\tt add\_body\_action()} method that adds the contents of this pointer
to the DynManager's BodyAction list.
\end{itemize}

The task of calling {\tt add\_body\_action()} to place an initializer
in the DynManager's body action list is deferred to the input file.
The input file now looks like
\begin{itemize}
\item Setting the contents of a particular initializer,
\item Pointing the BodyAction pointer declared in
the S\_define file point to this initializer,
\item Issuing a {\tt call} to the zero-rate
{\tt add\_body\_action()} job to add this initializer, and
\item Repeating the above for each initializer to be used in
the simulation run.
\end{itemize}

\subsection{Dynamic allocation of BodyAction objects}
\label{subsec:body_action_pool_practice}
The S\_define file may become bloated with multiple initializers of the
same type in a simulation that involves several vehicles.
Along with deferring the calls to {\tt add\_body\_action()},
this final option defers the allocation of the initializers
to the input file via the Trick input processor.

The key advantage of this final approach is that it maximizes flexibility.
This is also a potential disadvantage; enhanced flexibility invites errors
on the part of the user. Another disadvantage is that the objects allocated in the
input cannot be reliably checkpointed/restarted without due diligence on the part of the
simulation developer. However, Body Action's generally do not require checkpointing/restarting due to their implementation generally being a static and/or single-use.


\subsection{Functional Use of the Model}
While the \ModelDesc was designed primary for use at the S\_define level,
it most certainly can be used within some user-defined model.
This section describes two approaches to using the model within
some body of C++ code.

Before doing so, however, ask yourself why you want to do this.
The primary reason for using the \ModelDesc within a user-defined model
is to take advantage of the fact that the application of body actions can be
postponed until some later time. On the other hand, if some action is to be
performed immediately, simply invoke the desired functionality directly.
There is no need to use the \ModelDesc for such direct and immediate actions.

\subsubsection{Emulating S\_define/Input Processing}
One form of the functional use of this model will involve
implementing, in code, the same operations performed in an S\_define file
and a simulation input file. The user code will\begin{itemize}
\item Create a specific instance of a class that derives from BodyAction,
\item Populate it with data,
\item Add it to the Dynamic Manager's list of body actions
by invoking the Dynamic Manager's {\tt add\_body\_action()} method.
\end{itemize}

If the created object is to executed at initialization time, the job
that performs the above must run prior to the DynManager
{\tt initialize\_simulation()} job. If, on the other hand, the execution
of the created object is to be deferred (e.g., attaching a pair of
DynBody objects), the created action must initially be deactivated.
Activating the object at the appropriate time will cause the DynManager
{\tt perform\_actions()} job to perform the action.

\subsubsection{Calling Model Methods}
The \ModelDesc places several functional constraints on the way in
which the Dynamics Manager Model uses this model.
These constraints apply to any externally-developed model that
acts in lieu of the Dynamics Manager. These constraints are
\begin{enumerate}
\item A BodyAction object must be initialized.
The object's {\tt initialize()} method must be invoked prior to invoking
either the {\tt is\_ready()} or {\tt apply()} methods for that object.
\item A BodyAction object must be initialized once only.
Invoking a single object's {\tt initialize()} method multiple times may
result in undetected errors (e.g., memory leaks).
\item A BodyAction object must be initialized at the appropriate time.
\begin{enumerate}
\item BodyAttach actions must be initialized
after the {\tt mass\_subject} MassBody or {\tt dyn\_subject} DynBody object has had their initial mass properties set
and have had their mass points defined.
Same goes for the {\tt mass\_parent} MassBody or {\tt dyn\_parent} DynBody objects.
\item  DynBodyInit actions furthermore must be initialized after
the initialization-time attachments have been performed.
\item All other actions must be initialized after the states of the
simulation's DynBody objects have been set.
\end{enumerate}
\item A BodyAction object's {\tt apply()} method must not be invoked
before the object indicates that it is ready to be applied. In other words,
the object's {\tt is\_ready()} method must return {\tt true} before
calling the object's {\tt apply()} method.
\item A BodyAction object's {\tt apply()} method must not be invoked
multiple times. The \ModelDesc follows the fire-and-forget paradigm.
\end{enumerate}


\section{Extension}
\subsection{The \ModelDesc Was Designed For Extensibility}
The set of model classes provided with JEOD doesn't do what you want done?
Fine! Design a new class that does what you want.

The \ModelDesc places several functional constraints on the classes
that derive from the BodyAction base class.
These constraints apply to the classes that are a part of the model
and to any extensions of the class written by model extenders.
These constraints are on the trio of virtual methods as overridden
by the derived class.
\begin{enumerate}
\item The class must of course ultimately derive from the BodyAction base class.
\item The {\tt initialize()} method, if overridden,
must forward the {\tt initialize()} call to the immediate parent object
that handles the method.
The typical approach is to check for error conditions,
forward the {\tt initialize()} call upward, and finally perform
class-specific initializations.
\item The {\tt is\_ready()} method, if overridden,
must query the readiness of the parent class in addition to
peforming class-specific readiness checks. In particular, claiming to be ready
against the advice of the parent is forbidden.
\item The {\tt apply()} method, if overridden,
must forward the {\tt apply()} call to the immediate parent object.
The typical approach is to reformulate the user inputs in terms
that will be understood by the parent class,
forward the {\tt apply()} call  upward, and finally perform
any cleanup actions.
\end{enumerate}
