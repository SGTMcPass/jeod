%%%%%%%%%%%%%%%%%%%%%%%%%%%%%%%%%%%%%%%%%%%%%%%%%%%%%%%%%%%%%%%%%%%%%%%%%%%%%%%%
% BodyAttach_Detach_intro.tex
% Intro chapter of the BodyAttach_Detach part of the Body Action Model
%
%%%%%%%%%%%%%%%%%%%%%%%%%%%%%%%%%%%%%%%%%%%%%%%%%%%%%%%%%%%%%%%%%%%%%%%%%%%%%%%%
\chapter{Introduction}\label{ch:BodyAttach_Detach:intro}

\section{Purpose and Objectives of the
MassBody Attach/Detach Sub-Model}

The MassBody Attach/Detach Sub-Model
is responsible for attaching and detaching
a subject MassBody or DynBody object to and from another Body object or RefFrame object.

\section{Part Organization}
This part of the \ModelDesc document is organized along the
lines described in section \ref{sec:overview:docorg}. It
comprises the following chapters in order:

\begin{description}
\item[Introduction] -
This introduction describes the objective and purpose of the
Attach/Detach Sub-Model.

\item[Product Requirements] -
The Body Attach/Detach Sub-Model Product Requirements chapter
describes the requirements on the sub-model
and the requirements levied on user-defined classes that derive from
this sub-model.

\item[Product Specification] -
The Body Attach/Detach Sub-Model Product Specification chapter
describes the underlying theory, architecture, and design of
this sub-model.

\item[User Guide] -
Describes the use of this sub-model.
The Body Attach/Detach Sub-Model User Guide chapter
the following sections:
\begin{itemize}
 \item Analysts or users of simulations (Analysis).
 \item Integrators or developers of simulations (Integration).
 \item Model Extenders (Extension).
\end{itemize}

\item[Inspection, Verification, and Validation] -
The Body Attach/Detach Sub-Model IV\&V chapter
describes the techniques used to ascertain that
this sub-model satisfies the requirements levied upon it
and the summarizes the results of
the sub-model verification and validation tests.
\end{description}
