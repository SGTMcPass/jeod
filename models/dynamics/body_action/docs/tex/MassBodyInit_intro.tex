%%%%%%%%%%%%%%%%%%%%%%%%%%%%%%%%%%%%%%%%%%%%%%%%%%%%%%%%%%%%%%%%%%%%%%%%%%%%%%%%
% MassBodyInit_intro.tex
% Intro chapter of the MassBodyInit part of the Body Action Model
%
%%%%%%%%%%%%%%%%%%%%%%%%%%%%%%%%%%%%%%%%%%%%%%%%%%%%%%%%%%%%%%%%%%%%%%%%%%%%%%%%
\chapter{Introduction}\label{ch:MassBodyInit:intro}

\section{Purpose and Objectives of the
MassBodyInit Class}

The MassBodyInit Class
is responsible for initializing a MassBody object's
mass properties and mass points.

\section{Part Organization}
This part of the \ModelDesc document is organized along the
lines described in section \ref{sec:overview:docorg}. It
comprises the following chapters in order:

\begin{description}
\item[Introduction] -
This introduction describes the objective and purpose of the
MassBodyInit Class.

\item[Product Requirements] -
The MassBodyInit Class Product Requirements chapter
describes the requirements on the MassBodyInit class.

\item[Product Specification] -
The MassBodyInit Class Product Specification chapter
describes the underlying theory, architecture, and design of
this class.

\item[User Guide] -
Describes the use of this class.

\item[Inspection, Verification, and Validation] -
The DMassBodyInit Class IV\&V chapter
describes the techniques used to ascertain that
this class satisfies the requirements levied upon it
and the summarizes the results of
the sub-model verification and validation tests.
\end{description}
