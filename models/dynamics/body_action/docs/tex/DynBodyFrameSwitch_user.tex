%%%%%%%%%%%%%%%%%%%%%%%%%%%%%%%%%%%%%%%%%%%%%%%%%%%%%%%%%%%%%%%%%%%%%%%%%%%%%%%%
% DynBodyFrameSwitch_user.tex
% User guide for the DynBodyFrameSwitch class
%
%%%%%%%%%%%%%%%%%%%%%%%%%%%%%%%%%%%%%%%%%%%%%%%%%%%%%%%%%%%%%%%%%%%%%%%%%%%%%%%%

\chapter{User Guide}\label{ch:\modelpartid:user}
The following assumes the scheme outlined in
section~\ref{subsec:recommended_practice}.

\section{Analysis}

Assume a simulation S\_define file declares simulation objects
named `chaser' and `dynamics'. The chaser object contains
a DynBody object name `dyn\_body' and
a DynBodyFrameSwitch object named `frame\_switch'. The dynamics object
contains a DynManager object named `manager' and a BodyAction pointer
named `action\_ptr'.

The following will cause the vehicle to switch from the initial Earth-centered
inertial integration frame to a Moon-centered inertial integration frame upon
entering the Moon's sphere of influence.

\begin{verbatim}
chaser.frame_switch.set_subject_body(chaser.dyn_body);
chaser.frame_switch.action_name         = "frame_switch_enter";
chaser.frame_switch.integ_frame_name    = "Moon.inertial";
chaser.frame_switch.switch_sense        = DynBodyFrameSwitch::SwitchOnApproach;
chaser.frame_switch.switch_distance {M} = 66.2e6;
dynamics.body_action_ptr = &chaser.frame_switch;
call dynamics.dynamics.manager.add_body_action (dynamics.body_action_ptr);
\end{verbatim}

\section{Integration}
The simulation integrator must make the DynBodyFrameSwitch visible to
the simulation user. The above example assumes the integrator declared
a DynBodyFrameSwitch object named `frame\_switch' as part of the chaser
simulation object.

Because this is an asynchronous operation, the simulation integrator needs
to ensure that the Dynamics Manager {\tt perform\_actions()} method is called
as a scheduled job.

\section{Extension}
This model relies on the user to specify the radius of the sphere of influence.
An obvious extension to this model is to calculate that radius.
