%%%%%%%%%%%%%%%%%%%%%%%%%%%%%%%%%%%%%%%%%%%%%%%%%%%%%%%%%%%%%%%%%%%%%%%%%%%%%%%%
% BodyAttach_Detach_user.tex
% user for BodyAttach_Detach
%
%%%%%%%%%%%%%%%%%%%%%%%%%%%%%%%%%%%%%%%%%%%%%%%%%%%%%%%%%%%%%%%%%%%%%%%%%%%%%%%%

\chapter{User Guide}\label{ch:\modelpartid:user}
The following assumes the scheme outlined in
section~\ref{subsec:body_action_pool_practice}.

\section{Analysis}

The Trick input file code below illustrates the use of the \partxname.
The objects involved in this example are
\begin{description}
\item[Vehicle A] A DynBody object with two points of interest
  labeled as "point\_C" and "point\_D".
\item[Vehicle B] A DynBody object with one point of interest
  labeled as "point\_C".
\item[Vehicle C] A DynBody object with two points of interest
  labeled as "point\_B" and "point\_A"
\item[Vehicle D] A DynBody object with two points of interest
  labeled as "point\_A" and "point\_E".
\item[Mass E] This is a simple MassBody object with one point of interest
  labeled as "point\_D".
\end{description}
This example does not depict the mass and state initializations.
The mass initializations will endow the objects with mass properties
and create the points of interest identified above.
The state initializations will provide the independent vehicles
with state.

The events in this example are
\begin{itemize}
\item At initialization time, vehicle C is to be attached to vehicle B
  such that point B on vehicle C is connected to point C on vehicle B.
\item Also at initialization time, mass E is to be attached to vehicle D
  such that point D on mass E is connected to point E on vehicle D.
\item At t=10, vehicle C is to be attached to vehicle A
  such that point A on vehicle C is connected to point C on vehicle A.
  Note that because vehicle C is already attached to vehicle B,
  it is vehicle B that is attach to vehicle A rather than the C-to-A
  attachment that the user requested.
\item At t=20, vehicle D is to be attached to vehicle A
  such that point A on vehicle D is connected to point D on vehicle A.
\item At t=30, vehicle C is to detach from vehicle A.
 Note that the real attachment is B attached to A, and it is vehicle
 B that detaches from vehicle A rather than vehicle C.
\item At t=40, vehicle D is to detach from vehicle A.
\end{itemize}


\begin{verbatim}
BodyAttachAligned * C_to_B = new BodyAttachAligned[1];
C_to_B->action_name        = "C_to_B";
C_to_B->set_subject_body(vehicle_C.dyn_body.mass);
C_to_B->subject_point_name = "point_B";
C_to_B->set_parent_body(vehicle_B.dyn_body.mass);
C_to_B->parent_point_name  = "point_C";
dynamics.body_action_ptr   = C_to_B;
call dynamics.dynamics.manager.add_body_action (dynamics.body_action_ptr);

BodyAttachAligned * E_to_D = new BodyAttachAligned[1];
E_to_D->action_name        = "E_to_D";
E_to_D->set_subject_body(vehicle_D.mass_E);
E_to_D->subject_point_name = "point_D";
E_to_D->set_parent_body(vehicle_D.dyn_body);
E_to_D->parent_point_name  = "point_E";
dynamics.body_action_ptr   = E_to_D;
call dynamics.dynamics.manager.add_body_action (dynamics.body_action_ptr);

BodyAttachAligned * D_to_A = new BodyAttachAligned[1];
D_to_A->action_name        = "D_to_A";
D_to_A->set_subject_body(vehicle_D.dyn_body);
D_to_A->subject_point_name = "point_A";
D_to_A->set_parent_body(vehicle_A.dyn_body);
D_to_A->parent_point_name  = "point_D";
D_to_A->active             = false;
dynamics.body_action_ptr   = D_to_A;
call dynamics.dynamics.manager.add_body_action (dynamics.body_action_ptr);

BodyDetachSpecific * C_from_A = new BodyDetachSpecific[1];
C_from_A->action_name      = "D_from_A";
C_from_A->set_subject_body(vehicle_C.dyn_body.mass);
C_from_A->set_detach_from_body(vehicle_A.dyn_body.mass);
dynamics.body_action_ptr   = C_from_A;
call dynamics.dynamics.manager.add_body_action (dynamics.body_action_ptr);


// Mass and state initializations not shown.


stop = 10;
BodyAttachAligned * C_to_A = new BodyAttachAligned[1];
C_to_A->action_name        = "C_to_A";
C_to_A->set_subject_body(vehicle_C.dyn_body);
C_to_A->subject_point_name = "point_A";
// OK to use MassBody instance within DynBody (checked via MassBody::dyn_owner)
C_to_A->set_parent_body(vehicle_A.dyn_body.mass);
C_to_A->parent_point_name  = "point_C";
C_to_A->active             = true;
dynamics.body_action_ptr   = C_to_A;
call dynamics.dynamics.manager.add_body_action (dynamics.body_action_ptr);

stop = 20;
D_to_A->active           = true;

stop = 30;
C_from_A->active         = true;

stop = 40;
BodyDetach * D_from_A = new BodyDetach[1];
D_from_A->action_name      = "D_from_A";
D_from_A->set_subject_body(vehicle_D.dyn_body);
D_from_A->active           = true;
dynamics.body_action_ptr   = D_from_A;
call dynamics.dynamics.manager.add_body_action (dynamics.body_action_ptr);

stop = 50;
\end{verbatim}

\section{Integration}
The simulation integrator must make the \partxname classes needed
simulation user available for use.
Unlike other examples in this document,
the above only requires the integrator to declare a pool of objects.
\begin{verbatim}
sim_object {
   BodyAttachAligned * body_attach;
   BodyDetach * detach;
   BodyDetachSpecific * detach_specific;
} body_actions;
\end{verbatim}

Because this example involves asynchronous operations,
the simulation integrator also needs
to ensure that the Dynamics Manager {\tt perform\_actions()} method is called
as a scheduled job.


\section{Extension}
The DynBody and MassBody classes provides multiple ways to attach objects.
Each \partxname subclass provides only one of those techniques.
One possible extension to the sub-model is to create a class that
derives from BodyAttach that uses some other attachment
mechanism.
