%%%%%%%%%%%%%%%%%%%%%%%%%%%%%%%%%%%%%%%%%%%%%%%%%%%%%%%%%%%%%%%%%%%%%%%%%%%%%%%%
% BodyAttach_Detach_reqt.tex
% Requirements on/levied by the BodyAttach_Detach class
%
%%%%%%%%%%%%%%%%%%%%%%%%%%%%%%%%%%%%%%%%%%%%%%%%%%%%%%%%%%%%%%%%%%%%%%%%%%%%%%%%

\chapter{Product Requirements}\label{ch:\modelpartid:reqt}


\requirement[Attach framework]{Extensible Attachment Framework}
\label{reqt:BodyAttach_Detach:attach_support}
\begin{description}
  \item[Requirement:]\ \newline
    This sub-model shall provide an extensible framework for attaching
    the subject MassBody to some other MassBody object.
  \item[Rationale:]\ \newline
    The MassBody class defines multiple {\tt attach()} interfaces.
  \item[Verification:]\ \newline
    Inspection
\end{description}

\requirement{Attach MassBody/DynBody Objects}
\label{reqt:BodyAttach_Detach:attach}
\begin{description}
  \item[Requirement:]\ \newline
    This sub-model shall provide the ability to cause the subject MassBody
    object to be attached to another MassBody object
    \subrequirement{Point-to-point attach}
    \label{reqt:BodyAttach_Detach:attach_point}
      Given a pair of mass points on the two objects to be attached with zero
      offset between the mass point origins and a $180^{\circ}$ yaw
      between the mass point axes.
    \subrequirement{Matrix/offset attach}
    \label{reqt:BodyAttach_Detach:attach_matrix}
      Given an offset from the target body's structural origin to the
      subject body's structural origin and a transformation matrix from
      the target body's structural axes to the subject body's structural axes.
  \item[Rationale:]\ \newline
    This requirement derives directly from
    requirement~\ref{reqt:overview:mass_attach}.
  \item[Verification:]\ \newline
    Inspection, Test
\end{description}

\requirement[Detach from parent]{Detach MassBody/DynBody Object From Immediate Parent}
\label{reqt:BodyAttach_Detach:detach_immediate}
\begin{description}
  \item[Requirement:]\ \newline
    This sub-model shall provide the ability to cause the subject MassBody
    object to be detached from its mass tree parent object via the subject
    MassBody object's immediate {\tt detach()} method.
  \item[Rationale:]\ \newline
    This requirement derives directly from
    requirement~\ref{reqt:overview:mass_detach}.
  \item[Verification:]\ \newline
    Inspection, Test
\end{description}

\requirement[Specific detach]{Detach MassBody/DynBody Objects Below Specified Parent}
\label{reqt:BodyAttach_Detach:detach_specific}
\begin{description}
  \item[Requirement:]\ \newline
    This sub-model shall provide the ability to cause the mass subtree
    that contains the subject MassBody object to be detached
    from a specified parent.
  \item[Rationale:]\ \newline
    This requirement derives directly from
    requirement~\ref{reqt:overview:mass_detach}.
    The MassBody model provides two {\tt detach()} methods.
    This method uses the indirect detach version of that method.
  \item[Verification:]\ \newline
    Inspection, Test
\end{description}

\requirement{Kinematic Attach DynBody to RefFrame Objects}
\label{reqt:BodyAttach_Detach:ref_attach}
\begin{description}
  \item[Requirement:]\ \newline
    This sub-model shall provide the ability to cause the subject DynBody
    object to be attached to a RefFrame object kinematically.
    \subrequirement{Point-to-point attach}
    \label{reqt:BodyAttach_Detach:attach_point}
      Given a mass point on the DynBody and the reference frame to be attached with zero
      offset between the two point origins and a $180^{\circ}$ yaw
      between the mass point and RefFrame frames.
    \subrequirement{Matrix/offset attach}
    \label{reqt:BodyAttach_Detach:attach_matrix}
      Given an offset from the parent reference frame origin to the
      subject body's structural origin and a transformation matrix from
      the parent reference frame axes to the subject body's structural axes.
  \item[Rationale:]\ \newline
    This requirement derives directly from
    requirement~\ref{reqt:overview:mass_attach}.
  \item[Verification:]\ \newline
    Inspection, Test
\end{description}
