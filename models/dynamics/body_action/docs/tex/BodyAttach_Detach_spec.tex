%%%%%%%%%%%%%%%%%%%%%%%%%%%%%%%%%%%%%%%%%%%%%%%%%%%%%%%%%%%%%%%%%%%%%%%%%%%%%%%%
% BodyAttach_Detach_spec.tex
% spec for BodyAttach_Detach
%
%%%%%%%%%%%%%%%%%%%%%%%%%%%%%%%%%%%%%%%%%%%%%%%%%%%%%%%%%%%%%%%%%%%%%%%%%%%%%%%%

\chapter{Product Specification}\label{ch:\modelpartid:spec}

\section{Conceptual Design}
The \partxname comprises six classes.
\begin{itemize}
\item BodyAttach, which derives directly from the base BodyAction class.
This class provides the conceptual basis for attaching two MassBody objects.
The subject of the action is to be attached as a child of a parent body.
The geometric details of the attachment (and the actual attachment) is a task
left to derived classes.
\item BodyAttachAligned, which derives from BodyAttach.
A BodyAttachAligned object attaches MassBody objects using a point-to-point
attachment mechanism.
\item BodyAttachMatrix, which also derives from BodyAttach.
A BodyAttachMatrix object attaches MassBody objects using a geometric
attachment mechanism.
\item BodyReattach, which derives directly from BodyAction.
A BodyReattach object alters the geometry of an existing attachment.
\item BodyDetach, which derives directly from BodyAction.
A BodyDetach object detaches a MassBody object from its immediate
parent (which is not specified).
\item BodyDetachSpecific, which derives directly from BodyDetach.
A BodyDetachSpecific object detaches a MassBody subtree from a
specific parent.
\end{itemize}

\section{Mathematical Formulations}
All of the underlying mathematics is performed by the Mass Model
and the Dynamic Body Model, depending on what kinds of bodies are
being attached, reattached, or detached.
Refer to the Mathematical Formulations section of
the \hypermodelref{MASS}
and the \hypermodelref{DYNBODY}.

\section{Detailed Design}

\subsection{Attachment}
The \ModelDesc BodyAttach class provides the basis for performing
attachments. This class contains a member
data element that identifies the parent MassBody to which the subject
MassBody is to be attached.
However, this class itself does not perform the
attachment. That is the task of classes that derive from BodyAttach.
Two concepts drove this design.
\begin{itemize}
\item The MassBody class provides multiple ways to accomplish the basic task of
  attaching MassBody objects in the form of multiple {\tt attach\_to/attach\_child()} methods.
  Distinct classes that derive from BodyAttach use distinct
  {\tt attach\_to/attach\_child()} mechanisms.
\item The Dynamics Manager takes advantage of this inheritance to attach
  MassBody objects at initialization time. Any enqueued BodyAction objects
  that derive from BodyAttach and are ready to run at initialization
  time will be run as a part of the overall initialization processing.
  Attachments are performed after having performed all of the actions
  that initialize the mass properties and mass points of the simulation's
  MassBody objects but before performing any state initializations.
\end{itemize}

JEOD supplies two classes that derive from BodyAttach.
BodyAttachAligned uses the point-to-point form of {\tt attach\_to/attach\_child()}
while BodyAttachMatrix uses the offset+matrix form.
The BodyAttachAligned class contains data members that name the
attach points on the two vehicles.
The point-to-point {\tt apply()} method attaches the
subject MassBody as a child of the parent MassBody object at the named
attach points.
The BodyAttachMatrix contains data members that specify the location
and orientation of the subject body's structural frame with respect to
the parent body's structural frame. The offset+matrix {\tt apply()} method
attaches the subject MassBody as a child of the parent MassBody object
using the supplied offset and orientation.

\subsection{Reattachment}

\subsection{Detachment}
The MassBody class provides two versions of the {\tt detach()} method.
Each of the two sub-model classes that perform detachments use one
of these methods. The BodyDetach class uses the basic {\tt detach()}
method to detach the subject MassBody object from its immediate parent.
The class BodyDetachSpecific uses the form of {\tt detach()}
that specifies the parent object. The body that is actually detached
is the child of the specified parent that lies along the chain of mass bodies
that link the subject body to the parent body in terms of parent-child
relationships.

\subsection{MassBody versus DynBody Attachment and Detachment}
Properly attaching and detaching DynBody objects is a considerably more
involved process that the corresponding tasks for MassBody objects.
DynBody objects have momentum that must be conserved; energy must also
be conserved on detachment. This is not a concern for simple MassBody objects.
Fortunately for this model. all that detail is hidden in the guts of the
{\tt attach\_to/attach\_child()} and {\tt detach()} methods. This model does not need to concern
itself with the different behaviors; it merely needs to call the appropriate
{\tt attach\_to/attach\_child()} or {\tt detach()} method.
