%%%%%%%%%%%%%%%%%%%%%%%%%%%%%%%%%%%%%%%%%%%%%%%%%%%%%%%%%%%%%%%%%%%%%%%%%%%%%%%%
% overview_ivv.tex
% Toplevel verification and validation of the Body Action Model
%
%%%%%%%%%%%%%%%%%%%%%%%%%%%%%%%%%%%%%%%%%%%%%%%%%%%%%%%%%%%%%%%%%%%%%%%%%%%%%%%%

%----------------------------------
\chapter{Inspection, Verification, and Validation}\hyperdef{part}{ivv}{}
\label{ch:overview:ivv}
%----------------------------------

\section{Inspection}\label{ch:overview:inspect}
All of the high-level requirements levied on the \ModelDesc
are satisfied by inspection.

\inspection{Top-level Requirements}
\label{inspect:overview:top_level}
By inspection, the \ModelDesc satisfies the requirements specified in
requirement~\ref{reqt:overview:top_level}.

\inspection{Derived Requirements}
\label{inspect:overview:derived}
The requirements for the \ModelDesc are implemented as derived
requirements in the subsequent document parts. The inspections,
verifications, and validations used to satisfy the derived
requirements in turn satisfy these top-level requirements.
The requirements flow-down for the model is depicted in
table~\ref{tab:overview:reqt_flowdown}.

By inspection, the \ModelDesc satisfies the requirements specified in
requirements~\ref{reqt:overview:base_class}
to~\ref{reqt:overview:frame_switch}.

\begin{table}[htp]
\centering
\caption{Requirements Flowdown}
\label{tab:overview:reqt_flowdown}
\vspace{1ex}
\begin{tabular}{||l @{\hspace{4pt}} l|l @{\hspace{2pt}} l @{\hspace{4pt}} l|}
\hline
\multicolumn{2}{||l|}{\bf Requirement} &
\multicolumn{3}{l|}{\bf Artifact} \\ \hline\hline
\ref{reqt:overview:base_class} & Base class &
   Reqt. & \ref{reqt:BodyAction:base_class} &
   Base class\\
&& Reqt. & \ref{reqt:BodyAction:active} &
   Activation\\
&& Reqt. & \ref{reqt:BodyAction:virtual_methods} &
   Virtual methods\\
&& Reqt. & \ref{reqt:BodyAction:base_class_mandate} &
   Base class mandate\\[4pt]
\ref{reqt:overview:subject_body} & Subject specification &
   Reqt. & \ref{reqt:BodyAction:subject_body} &
   Subject body\\[4pt]
\ref{reqt:overview:mass_ppty_init} & Initialize mass properties &
   Reqt. & \ref{reqt:MassBodyInit:mass_ppty_init} &
   Initialize mass properties\\[4pt]
\ref{reqt:overview:mass_point_init} & Initialize mass points &
   Reqt. & \ref{reqt:MassBodyInit:mass_pnts_init} &
   Initialize mass points\\[4pt]
\ref{reqt:overview:mass_attach} & Attach MassBody/DynBody objects &
   Reqt. & \ref{reqt:BodyAttach_Detach:attach} &
   Attach MassBody/DynBody objects\\[4pt]
\ref{reqt:overview:mass_detach} & Detach MassBody/DynBody objects &
   Reqt. & \ref{reqt:BodyAttach_Detach:detach_immediate} &
   Detach from parent\\
&& Reqt. & \ref{reqt:BodyAttach_Detach:detach_specific} &
   Specific detach\\[4pt]
\ref{reqt:overview:dyn_body_init} & State initialization &
   Reqt. & \ref{reqt:DynBodyInit:cartesian_trans} &
   Translational state\\
&& Reqt. & \ref{reqt:DynBodyInit:cartesian_rot} &
   Rotational state\\
&& Reqt. & \ref{reqt:DynBodyInit:LVLH} &
   LVLH frame\\
&& Reqt. & \ref{reqt:DynBodyInit:NED} &
   NED frame\\
&& Reqt. & \ref{reqt:DynBodyInit:is_ready} &
   Readiness detection\\[4pt]
\ref{reqt:overview:frame_switch} & Frame switch &
   Reqt. & \ref{reqt:DynBodyFrameSwitch:frame_switch} &
   Frame switch\\
\hline
\end{tabular}
\end{table}
\clearpage


\inspection{Messages}
\label{inspect:overview:messages}
The \ModelDesc detects the following classes of errors and warnings:
\begin{itemize}
\item Fatal error, \verb+BodyActionMessages::fatal_error+. \\
Issued when a non-fatal error is detected and the user has requested that such
errors be treated as terminal errors.
\item Null pointer, \verb+BodyActionMessages::null_pointer+. \\
Issued when a pointer that should have been set is null.
Null pointer errors are always fatal.
\item Invalid name, \verb+BodyActionMessages::invalid_name+.\\
Issued when some name is invalid.
The error is not fatal if the action can proceed,
but the results should always be treated as suspect when this happens.
Invalid names that preclude performing the action are fatal errors.
\item Invalid object, \verb+BodyActionMessages::invalid_object+. \\
Issued when some object is not of the expected type.
Invalid object errors are treated similarly to invalid name errors.
\item Illegal value, \verb+BodyActionMessages::illegal_value+. \\
Issued when some value is not valid given the context.
A warning is issued if the value is correctable.
Uncorrectable illegal values are fatal errors.
\item Unperformed action, \verb+BodyActionMessages::not_performed+. \\
These are serious but non-fatal errors, issued under two circumstances:
\begin{itemize}
\item An object never indicates that it is ready to be performed
when it should have indicated readiness at some point in time.
This will occur, for example, when the user has created circular
dependencies amongst the queued DynBodyInit objects.
\item An object indicates that it is ready to be performed
but fails to perform the action as requested.
This will occur, for example, when the user has attempts to create
an illegal attachment.
\end{itemize}
\end{itemize}
The above are reported to the user via the \hypermodelref{MESSAGE}.

The \ModelDesc also uses the \MESSAGE~to indicate successful
application of a body action. These success messages use the
\verb+MessageHandler::debug()+ method with message class
\verb+BodyActionMessages::trace+.
These debug messages are not generated by default; users will not see these
messages unless they explicitly enable their generation.
The tests~\ref{test:BodyAttach_Detach:attach_detach},
\ref{test:DynBodyInit:state_init},
and~\ref{test:DynBodyFrameSwitch:frame_switch} do exactly that.

By inspection, the \ModelDesc satisfies
requirements~\ref{reqt:overview:errors}
and~\ref{reqt:overview:debug}.


\section{Metrics}

Table~\ref{tab:coarse_metrics} presents coarse metrics on the source
files that comprise the model.

\input{coarse_metrics}

Table~\ref{tab:metrix_metrics} presents the extended cyclomatic complexity
(ECC) of the methods defined in the model.

\input{metrix_metrics}


\section{Requirements Traceability}

The IV\&V artifacts that demonstrate the satisfaction of the requirements
in chapter~\ref{ch:overview:reqt} are depicted in
table~\ref{tab:overview:reqt_traceability}.

\begin{table}[htp]
\centering
\caption{Requirements Traceability}
\label{tab:overview:reqt_traceability}
\vspace{1ex}
\begin{tabular}{||l @{\hspace{4pt}} l|l @{\hspace{2pt}} l @{\hspace{4pt}} l|}
\hline
\multicolumn{2}{||l|}{\bf Requirement} &
\multicolumn{3}{l|}{\bf Artifact} \\ \hline\hline
\ref{reqt:overview:top_level} & Top-level requirement &
   Insp. & \ref{inspect:overview:top_level} &
   Top-level requirements\\[4pt]
\ref{reqt:overview:base_class} & Base class &
   Insp. & \ref{inspect:overview:derived} &
   Derived requirements\\[4pt]
\ref{reqt:overview:subject_body} & Subject specification &
   Insp. & \ref{inspect:overview:derived} &
   Derived requirements\\[4pt]
\ref{reqt:overview:mass_ppty_init} & Initialize mass properties &
   Insp. & \ref{inspect:overview:derived} &
   Derived requirements\\[4pt]
\ref{reqt:overview:mass_point_init} & Initialize mass points &
   Insp. & \ref{inspect:overview:derived} &
   Derived requirements\\[4pt]
\ref{reqt:overview:mass_attach} & Attach MassBody objects &
   Insp. & \ref{inspect:overview:derived} &
   Derived requirements\\[4pt]
\ref{reqt:overview:mass_detach} & Detach MassBody/DynBody objects &
   Insp. & \ref{inspect:overview:derived} &
   Derived requirements\\[4pt]
\ref{reqt:overview:dyn_body_init} & State initialization &
   Insp. & \ref{inspect:overview:derived} &
   Derived requirements\\[4pt]
\ref{reqt:overview:frame_switch} & Frame switch &
   Insp. & \ref{inspect:overview:derived} &
   Derived requirements\\[4pt]
\ref{reqt:overview:errors} & Errors &
   Insp. & \ref{inspect:overview:messages} &
   Messages\\[4pt]
\ref{reqt:overview:debug} & Trace &
   Insp. & \ref{inspect:overview:messages} &
   Messages\\
\hline
\end{tabular}
\end{table}
