%%%%%%%%%%%%%%%%%%%%%%%%%%%%%%%%%%%%%%%%%%%%%%%%%%%%%%%%%%%%%%%%%%%%%%%%%%%%%%%%
% BodyAction_reqt.tex
% Requirements on/levied by the BodyAction class
%
%%%%%%%%%%%%%%%%%%%%%%%%%%%%%%%%%%%%%%%%%%%%%%%%%%%%%%%%%%%%%%%%%%%%%%%%%%%%%%%%

\chapter{Product Requirements}\label{ch:\modelpartid:reqt}


\requirement[Base class]{Body Action Base Class Definition}
\label{reqt:BodyAction:base_class}
\begin{description}
\item[Requirement:]\ \newline
  The \ModelDesc shall define a class capable of forming the
  basis for all model classes that operate on a linked MassBody or DynBody object.
\item[Rationale:]\ \newline
  This requirement derives from requirement~\ref{reqt:overview:base_class}.
\item[Verification:]\ \newline
  Inspection
\end{description}

\requirement[Object identification]{Object Identification}
\label{reqt:BodyAction:self_identification}
\begin{description}
\item[Requirement:]\ \newline
  Each instance of the \ModelDesc base class shall
  provide the ability to identify itself.
\item[Rationale:]\ \newline
  This capability is needed for error handling and message generation.
\item[Verification:]\ \newline
  Inspection
\end{description}

\requirement[Subject body]{Subject Specification}
\label{reqt:BodyAction:subject_body}
\begin{description}
\item[Requirement:]\ \newline
  Each instance of the \ModelDesc base class shall
  identify the subject body object
  on which the model instance is to operate.
\item[Rationale:]\ \newline
  This requirement derives from requirement~\ref{reqt:overview:subject_body}.
\item[Verification:]\ \newline
  Inspection
\end{description}

\requirement{Activation}
\label{reqt:BodyAction:active}
\begin{description}
\item[Requirement:]\ \newline
  The \ModelDesc base class shall
  provide an activation capability.
\item[Rationale:]\ \newline
  This derived requirement requirement derives
  from requirement~\ref{reqt:overview:base_class}.
\item[Verification:]\ \newline
  Inspection
\end{description}

\requirement{Virtual Methods}
\label{reqt:BodyAction:virtual_methods}
\begin{description}
\item[Requirement:]\ \newline
  The BodyAction class shall define the functional interfaces
  that generically describe the operations on model objects.
  These functional interfaces are
  \subrequirement{Initialization.}
  \label{reqt:BodyAction:initialize}
  The BodyAction class shall define a generic interface
  for initializing BodyAction objects.
  \subrequirement{Query readiness.}
  \label{reqt:BodyAction:is_ready}
  The BodyAction class shall define a generic interface
  for querying the execution readiness of a BodyAction object.
  \subrequirement{Execute.}
  \label{reqt:BodyAction:apply}
  The BodyAction class shall define a generic interface
  for querying the execution readiness of a BodyAction object.
  A BodyAction instance shall provide the ability to
  perform some action on a MassBody/DynBody object.
\item[Rationale:]\ \newline
  This requirement derives from requirement~\ref{reqt:overview:base_class}.
\item[Verification:]\ \newline
  Inspection
\end{description}

\requirement[Base class mandate]{Body Action Base Class Mandate}
\label{reqt:BodyAction:base_class_mandate}
\begin{description}
\item[Requirement:]\ \newline
  All \ModelDesc classes that operate on a MassBody/DynBody object
  shall derive from the BodyAction base class.
\item[Rationale:]\ \newline
  This requirement derives from requirement~\ref{reqt:overview:base_class}.

  Note: This is a requirement levied by the \partxname on the
  subsequent sub-models.
\item[Verification:]\ \newline
  Inspection
\end{description}

\requirement{Virtual Method Overrides}
\label{reqt:BodyAction:overrides}
\begin{description}
\item[Requirement:]\ \newline
  All \ModelDesc classes that override the virtual methods
  described in requirement~\ref{reqt:BodyAction:virtual_methods}
  shall forward the method invocation to the
  most immediate parent class that handles this method.
  \subrequirement{Initialization.}
  \label{reqt:BodyAction:initialize_override}
  Derived classes that override the {\tt initialize()} shall not
  return from {\tt initialize()} without having invoked the
  overridden implementation of the method.
  \subrequirement{Query readiness.}
  \label{reqt:BodyAction:is_ready_override}
  Derived classes that override the {\tt is\_ready()} must
  not indicate that the instance is ready to execute if the
  overridden {\tt is\_ready()} method would indicated otherwise.
  \subrequirement{Execute.}
  \label{reqt:BodyAction:apply_override}
  Derived classes that override the {\tt apply()} shall not
  return from {\tt apply()} without having invoked the
  overridden implementation of the method.
\item[Rationale:]\ \newline
  This is an internal requirement levied on derived classes.
  The intent is to reduce replication of code and to ensure
  that the parent classes are safe to operate.

  Note:
  This requirement does not pertain in the exceptional cases
  where the overriding method does not return.
  In particular, the method MessageHandler::Fail() does not return.
\item[Verification:]\ \newline
  Inspection
\end{description}
