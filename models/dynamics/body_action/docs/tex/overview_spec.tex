%%%%%%%%%%%%%%%%%%%%%%%%%%%%%%%%%%%%%%%%%%%%%%%%%%%%%%%%%%%%%%%%%%%%%%%%%%%%%%%%
% overview_spec.tex
% Toplevel specification of the Body Action Model
%
%%%%%%%%%%%%%%%%%%%%%%%%%%%%%%%%%%%%%%%%%%%%%%%%%%%%%%%%%%%%%%%%%%%%%%%%%%%%%%%%

%----------------------------------
\chapter{Product Specification}\hyperdef{part}{spec}{}
\label{ch:overview:spec}
%----------------------------------

\section{Conceptual Design}
\subsection{Overview}
The \ModelDesc comprises several classes that provide various mechanisms
for setting aspects of instances of the MassBody class, DynBody class, and
their derived classes. The capabilities provided by the model include:
\begin{itemize}
\item Setting the mass properties of a MassBody object,
\item Attaching and detaching MassBody and DynBody objects to/from one another or a RefFrame,
\item Setting the state of a DynBody object, and
\item Changing the frame in which a DynBody object's state is propagated.
\end{itemize}

Note that while the first function refers to a MassBody, the latter two
capabilities pertain to DynBody. Meanwhile, attachment can involve
interactions between a DynBody and MassBody object.
The subject can be specified by MassBody or DynBody, and the correct behavior
implied (if possible).

The first two capabilities pertain to body objects in general.
The model does not specifically provide the ability to set the mass properties of a DynBody object.
Because a DynBody contains a MassBody and because the mass properties are orthogonal
to the additional capabilities provided by the DynBody class,
the same capability used to set the mass properties
of a MassBody object functions correct when applied to a DynBody object.

Attaching and detaching DynBody objects to/from one another  involves
considerably more computational effort than does attaching/detaching
simple MassBody objects. The
DynBody class adds state information to the
MassBody class. How attachment and detachment affect state is not a part
of the basic MassBody attachment and detachment calculations due to
conservation of momentum calculations.
Thanks to the construction of the attachment and detachment methods,
there is no need for a special-purpose DynBody attachment or detachment
body action. The basic method works correctly when applied to a DynBody object.

\subsection{Base Classes}

All of the instantiable \ModelDesc classes ultimately derive from the
BodyAction base class. This base class provides a common framework for
describing actions to be performed on a MassBody (or derived class) object.
The common basis for all \ModelDesc objects enables forming a
collection of body actions in the form of base class pointers. The DynManager
class does exactly this. The \ModelDesc was explicitly architected to enable
this treatment.

The \ModelDesc defines one other base class, the BodyActionMessages class.
This static class is not instantiable and has no derived classes. It exists to
conform with the JEOD scheme for generating messages.

See part \ref{part:BodyAction} for a complete description of these
base classes.

\subsection{Derived Classes}

The BodyAction base class by itself does not change a single
aspect of a subject MassBody or DynBody type object. Making changes to a
subject is the responsibility of the various classes that derive from the
BodyAction base class. The derived classes include
\begin{itemize}
\item MassBodyInit. \\
MassBody objects represent masses with structural and body reference frames.
This class initializes a subject MassBody object's mass properties,
specifies the object's structure-to-body transformation, and
any defines user-defined points of interest (``mass points'').

See part \ref{part:MassBodyInit} for a complete description of the
MassBodyInit class.

\item BodyAttach and BodyDetach. \\
Connected vehicle objects form a mass tree topology.
Classes that derive from BodyAttach attach a subject Body object to
some other Body object or RefFrame. JEOD defines the BodyAttachAligned class to connect
two bodies or a body and reference frame by aligning specified vehicle points or reference frames at coincident
position with a 180 degree yaw relative orientataion. The BodyAttachMatrix
class defines attachment by defining the relative position of a child body
structure with respect to a parent body structure point or reference frame.
The classes BodyDetach and BodyDetachSpecific undo
these attachments.

See part \ref{part:BodyAttach_Detach} for a complete description of the
attach/detach sub-model.

\item DynBodyInit. \\
The DynBody class includes several attributes absent from the MassBody class that are
needed to model a vehicle in space. This includes the object's state---its
position, velocity, attitude, and angular velocity with respect to some
inertial (non-rotating) reference frame. Classes that derive from
DynBodyInit initialize part or all of a subject DynBody object's state.
JEOD defines several classes that derive from the DynBodyInit class.

See part \ref{part:DynBodyInit} for a complete description of the
DynBodyInit sub-model.

\item DynBodyFrameSwitch. \\
The ability to switch the frame in which a DynBody object's state is
represented and propagated was a driving requirement for JEOD.
The DynBodyFrameSwitch accomplishes that goal.

See part \ref{part:DynBodyFrameSwitch} for a complete description of the
DynBodyFrameSwitch class.
\end{itemize}

\subsection{Interactions With Other Models}
The \ModelDesc interacts with several JEOD models. Chief among these
are interactions with
the \hypermodelref{MASS},
which defines the MassBody class;
the \hypermodelref{DYNBODY},
which defines the DynBody class; and
the \hypermodelref{DYNMANAGER},
which defines the DynManager class.

A JEOD simulation typically contains one or more instances of a
class that derives from the DynBody class.
Each DynBody object represents some vehicle in space.
The DynBody class contains an instance of the MassBody class,
and it can have other MassBody objects or DynBody object attached to it.

The purpose of this model is to alter some aspect of an instance of the
MassBody, DynBody, or derived class.
Some of the model classes operate directly on MassBody objects, some on DynBody
objects. In general, the user may specify either as the subject to the BodyAction,
and the correct body type will be inferred.
The \ModelDesc classes that operate on MassBody and DynBody objects
by utilizing their public interfaces methods.
Other model classes specialize to altering some aspect of a DynBody object.
These classes use the public interfaces of the DynBody class to achieve
their desired end.

Along to the DynBody objects that represent vehicles,
a JEOD simulation also contains a DynManager object that manages the
dynamic objects in a simulation.
The DynManager object uses the \ModelDesc via a Standard Template Library
list of pointers to \ModelDesc objects. Simulation users can add
elements to this list. As the list comprises pointers to objects, the
pointed-to objects can be instances of different classes so long as
they derive from the same base class. (All \ModelDesc objects derive
from a common base class.)
The simulation DynManager object uses this list during simulation
initialization time to initialize the simulation's MassBody/DynBody objects
and uses this list during simulation run time to perform
asynchronous actions.

In addition to the above three models, the \ModelDesc interacts with
several other JEOD models. A driving concern for the \ModelDesc is to
do as little as possible in the model.
The model source files instead leverage capabilities that exist in other
models. In addition to the models listed above, this model uses:
\begin{itemize}
\item The \hypermodelref{DERIVEDSTATE}
to construct Local Vertical, Local Horizontal frames,
\item The \hypermodelref{MATH}
for matrix and vector manipulations,
\item The \hypermodelref{MEMORY}
to allocate memory,
\item The \hypermodelref{MESSAGE}
to report errors and generate messages,
\item The \hypermodelref{NAMEDITEM}
to construct names,
\item The \hypermodelref{ORBITALELEMENTS}
to compute orbits,
\item The \hypermodelref{PLANET}
to determine planetary coefficients,
\item The \hypermodelref{PLANETFIXED}
to construct North-East-Down frames, and
\item The \hypermodelref{REFFRAMES}
to compute relative state.
\end{itemize}


\section{Mathematical Formulations}
N/A
\section{Detailed Design}
See the \hyperapiref{BODYACTION}
for a description of classes that comprise the model and for descriptions
of the member data and member functions defined by these classes.

\section{Inventory}
All \ModelDesc files are located in
{\tt \$\{JEOD\_HOME\}/models/dynamics/body\_action}.
Relative to this directory,
\begin{itemize}
\vspace{-0.2\baselineskip}
\item Header and source files are located
in the model {\tt include} and {\tt src} subdirectories.
Table~\ref{tab:source_files} lists the
configuration-managed files in these directories.
\vspace{-0.1\baselineskip}
\item Documentation files are located in the model {\tt docs} subdirectory.
See table~\ref{tab:documentation_files}
for a listing of the
configuration-managed files in this directory.
\vspace{-0.1\baselineskip}
\item Verification files are located in the model {\tt verif} subdirectory.
See table~\ref{tab:verification_files}
for a listing of the
configuration-managed files in this directory.
\end{itemize}

\input{inventory}
