%%%%%%%%%%%%%%%%%%%%%%%%%%%%%%%%%%%%%%%%%%%%%%%%%%%%%%%%%%%%%%%%%%%%%%%%%%%%%%%%%
%
% Purpose:  Analysis part of User's Guide for the LVLH relative derived state model
%
% 
%
%%%%%%%%%%%%%%%%%%%%%%%%%%%%%%%%%%%%%%%%%%%%%%%%%%%%%%%%%%%%%%%%%%%%%%%%%%%%%%%%
The layout of this chapter will be slightly different from
what is seen in the User Guide chapters
of most JEOD models. The Analyst and Integrator sections will be combined, and
since we do not expect this model to be extended, the Extender section is not
included.

The \LRDSDesc\ provides the capability to maintain a relative derived state
of a subject vehicle with respect to an LVLH target frame. The target frame
is generated independently by the LVLH frame model found in the utilities
section. This LVLH frame need not be associated with a
vehicle. Its definition requires only a reference planet and a translational
state in planetary inertial coordinates. For more information, see the
documentation for the LVLH frame model.

The \LRDSDesc\ is a specialization of the \textit{RelativeDerivedState} whose
target frame is a rectilinear LVLH frame, possibly, but not necessarily, of a
target vehicle. The \textit{rel\_state} field of the \LRDSDesc\ indicates the
state of the subject vehicle in rectilinear LVLH or optionally in
a circular curvilinear system based on the target frame.
\section{Using the \LRDSDesc\ in a Simulation}
If your goal is to define and maintain an LVLH relative derived state, then
you will need an instance of \textit{LvlhRelativeDerivedState} in
your S\_define file. You normally want the initializations and updates of
the \LRDSDesc\ to occur subsequent to those of the \textit{LvlhFrame}, so
the \textit{LvlhRelativeDerivedState} instance is often
in a \textit{SimObject} near the end of the S\_define file.

You will also need to have an instance of \textit{LvlhFrame} in your simulation.
See the User Guide chapter of the documentation for the LVLH frame for details.
The initialization process for LVLH frame registers the frame with the
\textit{DynManager} so that the \LRDSDesc\ can access it by name.
Just as for \textit{RelativeDerivedState}, you supply the names of the
subject and target frames in the inputfile.

Example:
\begin{verbatim}
sim_object.lvlh_RelDerState.subject_frame_name =
"vehicleB.composite_body"
...
...
sim_object.lvlh_RelDerState.target_frame_name =
"vehicleA.composite_body.ref_planet.lvlh"
\end{verbatim}

\section{Other Options}
There are two other optional values which you can set. The field
\textit{lvlh\_type} should be one of two enumerated values, either
\textit{LvlhType.Rectilinear} for standard rectangular LVLH or
\textit{LvlhType.CircularCurvilinear} for output in circular curvilinear
coordinates. The default is \textit{LvlhType.Rectilinear}.

There is also a boolian \textit{use\_theta\_dot\_correction}. If set to true,
the circular curvilinear angular velocity will account for rate of change of
the phase angle $\theta$. The default value is false.
\section{Output Data}
The \textit{rel\_state} field of \textit{LvlhRelativeDerivedState} contains
the state of the subject with respect to the target frame. The following code
example shows how to log the entire state.
\begin{verbatim}
  for ii in range(0,3) :
    dr_group.add_variable("rel_state.vehicleA_wrt_vehicleB_in_B.
                           rel_state.trans.position[" + str(ii) + "]" )
  for ii in range(0,3) :
    dr_group.add_variable("rel_state.vehicleA_wrt_vehicleB_in_B.
                           rel_state.trans.velocity[" + str(ii) + "]" )
  for ii in range(0,3) :
    dr_group.add_variable("rel_state.vehicleA_wrt_vehicleB_in_B.
                           rel_state.rot.ang_vel_this[" + str(ii) + "]" )
  for ii in range(0,3) :
    for jj in range (0,3) :
      dr_group.add_variable("rel_state.vehicleA_wrt_vehicleB_in_B.
                             rel_state.rot.T_parent_this[" +
                             str(ii) + "][" + str(jj) + "]")
\end{verbatim}
