%%%%%%%%%%%%%%%%%%%%%%%%%%%%%%%%%%%%%%%%%%%%%%%%%%%%%%%%%%%%%%%%%%%%%%%%%%%%%%%%%
%
% Purpose:  Introduction for the Relative model.
%
% 
%
%%%%%%%%%%%%%%%%%%%%%%%%%%%%%%%%%%%%%%%%%%%%%%%%%%%%%%%%%%%%%%%%%%%%%%%%%%%%%%%%


%\section{Purpose and Objectives of \RelativeDesc}
% Incorporate the intro paragraph that used to begin this Chapter here. 
% This is location of the true introduction where you explain what this model 
% does.
The \RelativeDesc\ allows the state of one reference frame (e.g. the body reference frame of a vehicle) to be expressed relative to any other reference frame (e.g., the LVLH reference frame of another vehicle). As of JEOD version 3.4, it is no longer required that one of the reference frames be associated with a DynBody; the relative state can now be calculated between any two arbitrary frames in the simulation.













