%%%%%%%%%%%%%%%%%%%%%%%%%%%%%%%%%%%%%%%%%%%%%%%%%%%%%%%%%%%%%%%%%%%%%%%%%%%%%%%%%
%
% Purpose:  Conceptual part of Product Spec for the LVLH relative derived state model
%
% 
%
%%%%%%%%%%%%%%%%%%%%%%%%%%%%%%%%%%%%%%%%%%%%%%%%%%%%%%%%%%%%%%%%%%%%%%%%%%%%%%%%


%\section{Conceptual Design}
The \LRDSDesc\ is used to express the state of a subject frame
relative to a system that is aligned with the local vertical and horizontal
of a given point with respect to a planetary object. The relative state
includes the full 6-DoF state of a given reference frame as viewed from
the LVLH frame. Unlike a true reference frame, this relative state is not
part of the reference frame tree structure. Although it has no parent, the
relative state can be considered to be a child of the LVLH frame, especially
in the rectilinear case. Introduction of curvilinear coordinates changes the 
way distances, velocities and directions are computed, but the relative state
still represents the way the subject looks from the point of view of the
LVLH frame.


