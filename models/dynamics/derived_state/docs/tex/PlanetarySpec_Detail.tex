
%%%%%%%%%%%%%%%%%%%%%%%%%%%%%%%%%%%%%%%%%%%%%%%%%%%%%%%%%%%%%%%%%%%%%%%%%%%%%%%%%
%
% Purpose:  Detailed part of Product Spec for the Planetary model
%
% 
%
%%%%%%%%%%%%%%%%%%%%%%%%%%%%%%%%%%%%%%%%%%%%%%%%%%%%%%%%%%%%%%%%%%%%%%%%%%%%%%%%

\section{Detailed Design}
See the \href{file:refman.pdf}{Reference Manual}\cite{derivedstatebib:ReferenceManual} for a summary of member data and member methods for all classes.  

\subsection{Process Architecture}
The architecture for the \PlanetaryDesc\ is trivial, comprising methods that are operationally independent.

\subsection{Functional Design}
This section describes the functional operation of the methods in each class.

The \PlanetaryDesc\ contains only one class:
\begin{itemize}
\classitem{PlanetaryDerivedState}
\textref{DerivedState}{ref:DerivedState}

This contains only the methods \textit{set\_use\_alt\_pfix}, \textit{initialize}, and \textit{update}:
\begin{enumerate}
\funcitem{set\_use\_alt\_pfix}
This method sets the flag indicating whether to use the default planet-centered, planet-fixed reference frame associated with the planet or the alternate one, if one is available for the planet.

\funcitem{initialize}
The initialization process comprises the following steps:
\begin{enumerate}
\item{} The generic DerivedState initialization routine is called to establish the naming convention of the reference object (i.e., the planet about which the vehicle is orbiting) and state identifier.
\item{} The DerivedState method \textit{find\_planet} is called (which subsequently calls the  \href{file:\JEODHOME/models/dynamics/dyn_manager/docs/dyn_manager.pdf}{\em Dynamics Manager}~\cite{dynenv:DYNMANAGER}) method of the same name to find a planet by name; that name is stored as \textit{reference\_name}, and is usually assigned in an input file.
\item{} Depending on the value of \textit{use\_alt\_pfix}, the element \textit{pfix\_ptr} is set to either the default planet-fixed reference frame or the alternate planet-fixed reference frame of the planet identified by \textit{reference\_name} and the frame is subscribed.
\item{} The identified planet is then passed into the initialize method for the PlanetFixedPosition object, \textit{state}, which simply records a pointer to the planet just found in the PlanetFixedPosition class (see \href{file:\JEODHOME/models/utils/planet\_fixed/docs/planet\_fixed.pdf}{\em Planet Fixed} documentation~\cite{dynenv:PLANETFIXED}).
\end{enumerate}

\funcitem{update}
This method uses the RefFrame method \textit{compute\_position\_from} (see \href{file:\JEODHOME/models/utils/ref\_frames/docs/ref\_frames.pdf}{\em Reference Frames} documentation~\cite{dynenv:REFFRAMES}) to calculate the Cartesian position of the \textit{subject} in the planet fixed reference frame.  Then, it uses the PlanetFixedPosition method \textit{update\_from\_cart} 
(see \href{file:\JEODHOME/models/utils/planet\_fixed/docs/planet\_fixed.pdf}{\em Planet Fixed} documentation~\cite{dynenv:PLANETFIXED}) 
to convert the Cartesian representation into spherical and elliptical representations.

\end{enumerate}
\end{itemize}
