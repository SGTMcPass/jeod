%%%%%%%%%%%%%%%%%%%%%%%%%%%%%%%%%%%%%%%%%%%%%%%%%%%%%%%%%%%%%%%%%%%%%%%%%%%%%%%%%
%
% Purpose:  Analysis part of User's Guide for the NED model
%
% 
%
%%%%%%%%%%%%%%%%%%%%%%%%%%%%%%%%%%%%%%%%%%%%%%%%%%%%%%%%%%%%%%%%%%%%%%%%%%%%%%%%

% \section{Analysis}
\label{sec:neduseranalysis}
It must be reiterated that the \NEDDesc\ does not provide a state for the vehicle with which it is associated, it provides a \textit{reference frame}.  The NedDerivedState does contain a member called \textit{ned\_state}, which contains the state of the reference frame with respect to the parent planet-centered planet-fixed reference frame.  This is identical to the planet-fixed state (PlanetaryDerivedState) of the vehicle with respect to the same planet; it is \textbf{not} the North-East-Down state of the vehicle with respect to the same planet. 

\subsection{Identifying the \NEDDesc}
If the North-East-Down reference frame has been included in the simulation, there will be an instance of \textit{NedDerivedState} located in the S\_define file.  This would typically be found in either the vehicle object (usually if the vehicle is to be described in a North-East-Down sense), or a separate relative-state object (usually if a relative state is to be expressed as such).  There should be an accompanying call to an initialization routine, which takes a reference to the \textit{subject\_body} as one if its inputs, and an accompanying call to an update function.  The essential variables \textit{reference\_name} and \textit{ned\_state.altlatlong\_type} are defined elsewhere, often in the input file.

The North-East-Down reference frame, generated by the \NEDDesc, may also be used to represent the state of another vehicle.  In this case, there will be further inclusion of a RelativeDerivedState instance; see the \textref{RelativeDerivedState User's Guide}{sec:relativeuseranalysis} for details on the implementation of RelativeDerivedState.

Example:
\begin{verbatim}
sim_object{
dynamics/derived_state:    NedDerivedState example_of_ned_state;

(initialization) dynamics/derived_state:
example_of_rel_state_object.example_of_ned_state.initialize (
    Inout DynBody &      subject_body = vehicle_1.dyn_body,
    Inout DynManager &   dyn_manager  = manager_object.dyn_manager);
    
{environment} dynamics/derived_state:
example_of_rel_state_object.example_of_ned_state.update ( )

} example_of_rel_state_object;
\end{verbatim}

Then the input file may have entries comparable to:
\begin{verbatim}
example_of_rel_state_object.example_of_ned_state.reference_name = "Earth";
example_of_rel_state_object.example_of_ned_state.ned_state.altlatlong_type = 
                                                      NorthEastDown::spherical;
\end{verbatim}


\subsection{Editing the \NEDDesc}
The planetary identification, and the planetary surface interpretation (spherical / elliptical) are open to edit by the analyst.

\subsection{Output Data}
The following outputs are available from the \NEDDesc, but once again, these are simply the planetary derived states of the origin of the North-East-Down reference frame.  The strength of the \NEDDesc\ comes in its applicability to a RelativeDerivedState; see the \textref{RelativeDerivedState User's Guide}{sec:relativeuseranalysis} for details on the implementation of RelativeDerivedState.
\begin{verbatim}
example_of_rel_state_object.example_of_ned_state.ned_state.cart_coords[0-2]
example_of_rel_state_object.example_of_ned_state.ned_state.sphere_coords.altitude
example_of_rel_state_object.example_of_ned_state.ned_state.sphere_coords.latitude
example_of_rel_state_object.example_of_ned_state.ned_state.sphere_coords.longitude
example_of_rel_state_object.example_of_ned_state.ned_state.ellip_coords.altitude
example_of_rel_state_object.example_of_ned_state.ned_state.ellip_coords.latitude
example_of_rel_state_object.example_of_ned_state.ned_state.ellip_coords.longitude
\end{verbatim}

