%%%%%%%%%%%%%%%%%%%%%%%%%%%%%%%%%%%%%%%%%%%%%%%%%%%%%%%%%%%%%%%%%%%%%%%%%%%%%%%%%
%
% Purpose:  Introduction for the DerivedState model.
%
%
%
%%%%%%%%%%%%%%%%%%%%%%%%%%%%%%%%%%%%%%%%%%%%%%%%%%%%%%%%%%%%%%%%%%%%%%%%%%%%%%%%


%\section{Purpose and Objectives of \DerivedStateDesc}
% Incorporate the intro paragraph that used to begin this Chapter here.
% This is location of the true introduction where you explain what this model
% does.

The Derived State model provides the capability to express a
vehicular state in one or more of the most frequently used reference frame concepts (such as LVLH), with respect to another object included in the simulation.  While effort has been made to identify and include those reference frame representations most frequently utilized in orbital dynamics simulations, we recognize that we have not provided complete coverage of what is a boundless number of possible representations, and so have provided the flexibility that allows for the further definition of additional reference frame representations for inclusion into the model as needed.

The following reference frame representations are included in the \DerivedStateDesc:

\begin{enumerate}
\item{\EulerDesc}
\item{\LVLHDesc}
\item{\NEDDesc}
\item{\OrbElemDesc}
\item{\PlanetaryDesc}
\item{\RelativeDesc}
\item{\SolarBetaDesc}
\end{enumerate}
