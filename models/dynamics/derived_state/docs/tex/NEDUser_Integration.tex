%%%%%%%%%%%%%%%%%%%%%%%%%%%%%%%%%%%%%%%%%%%%%%%%%%%%%%%%%%%%%%%%%%%%%%%%%%%%%%%%%
%
% Purpose:  Integration part of User's Guide for the NED model
%
% 
%
%%%%%%%%%%%%%%%%%%%%%%%%%%%%%%%%%%%%%%%%%%%%%%%%%%%%%%%%%%%%%%%%%%%%%%%%%%%%%%%%

 \section{Integration}

Simply including the \NEDDesc\ is straightforward; using it to its full capacity is more challenging.

\subsection{Generating the S\_define}

It should be reiterated that if the desire is simply to express the state of a vehicle with respect to the planet, the NedDerivedState and PlanetaryDerivedState perform the same task.  However, the PlanetaryDerivedState is preferred in this situation, because it does not perform the additional task of generating a reference frame.

The North-East-Down model should be used when the state of a vehicle is desired with respect to some point that is not at the center of the planet, e.g., a launch facility.  Under these circumstances, it becomes necessary to instantiate both a NedDerivedState, and a RelativeDerivedState, with the understanding that the NedDerivedState must be evaluated before the RelativeDerivedState (otherwise the RelativeDerivedState output will be based on an out-of-date orientation of the NED reference frame).

An important consequence of using a RelativeDerivedState to express the relative states of two objects in a North-East-Down reference frame requires the consideration of the class structure of both NedDerivedState and RelativeDerivedState.  There are two types of reference frames in use, BodyRefFrame, and RefFrame.  The BodyRefFrame is associated with a DynBody (a body with dynamic properties), while the RefFrame need not be, it can be associated with some fixed (non-dynamic) body or point.

The North-East-Down reference frame in a NedDerivedState is an instance of a RefFrame, there is no requirement that this be attached to a DynBody.  However, the RelativeDerivedState contains both a BodyRefFrame (\textit{subject\_frame}) and a RefFrame (\textit{target\_frame}); the intention is that the state of a body can be expressed in some other reference frame. Therefore, when generating the relative states, it is \textit{not} possible to use the generated North-East-Down reference frame as the subject frame, it must instead be the target frame.  Then, the vehicle whose state is being expressed must be associated with the subject frame, and the \textit{subject\_frame\_name} and \textit{target\_frame\_name} must be set appropriately.  Then, the \textit{direction\_sense} of the Relative Derived State must be \textit{ComputeSubjectStateinTarget}.

An illustration of this is found in the verification simulations for the North-East-Down model (verif/SIM\_NED).  Here, the relative state instance that expresses the state of vehicle1 (\textit{veh}) with respect to vehicle2 (\textit{veh2}), \textit{veh\_wrt\_veh2}, has:
\begin{verbatim}
subject_frame_name = "vehicle1.composite_body";
target_frame_name = "vehicle2.Earth.ned";
direction_sense = RelativeDerivedState::ComputeSubjectStateinTarget;
\end{verbatim}

Notice also in these simulations that the NedDerivedState is contained within the respective vehicle object, while the RelativeDerivedState is contained within the \textit{rel\_state} object.  This assignment is intentional, though not necessary.  What is necessary is that the state of the two vehicles be evaluated before the relative state, and that the NED reference frame be appropriately oriented, based on the state of its parent vehicle, before the relative state can be expressed in that frame.  Therefore, it was chosen to move the relative state calculations to the end of the S\_define, where they will be processed only after the vehicle and NED reference frames have been processed.  This is recommended procedure.

\subsection{Generating the Input File}
The input file (or Modified Data file) must contain the name of the planet, identified as \textit{reference\_name}, and the identification of whether the North-East-Down reference frame should be based on a spherical (geocentric) or elliptical (geodetic) surface.
\begin{verbatim}
example_of_rel_state_object.example_of_ned_state.reference_name = "Earth";
example_of_rel_state_object.example_of_ned_state.ned_state.altlatlong_type = 
                                                       NorthEastDown::spherical;
\end{verbatim}
or
\begin{verbatim}
example_of_rel_state_object.example_of_ned_state.reference_name = "Earth";
example_of_rel_state_object.example_of_ned_state.ned_state.altlatlong_type = 
                                                     NorthEastDown::elliptical;
\end{verbatim}

\subsection{Logging the Data}
See the \textref{Analysis}{sec:neduseranalysis} section for a list of available output data, although again, typically, the desired output results from the relative state implementation of the NED reference frame, rather than from the NED state.  See the \textref{Analysis}{sec:relativeuseranalysis} section for a list of available output data from a RelativeDerivedState.