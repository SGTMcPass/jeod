\subsection {Mass}
The core and composite masses are found at \textit{*\_properties.mass}.
\subsubsection {Core Property}
This value has to be set externally (e.g. by the user).
\subsubsection {Composite Property}
This value is computed trivially,
\begin{equation}
  M_{composite} = M_{core} + \sum_{children} M_{composite,i}
\end{equation}
where $M_{composite,i}$ is the composite mass of each of the children of this
Mass Body.  Clearly, generation of the composite mass is an iterative
procedure, requiring first the calculation of the corresponding value for each
of the children, which require the same for theirs, etc.

\subsection{Center of Mass}
The position of the core and composite centers of mass are found at
\textit{*\_properties.position}; the value is expressed in the structural axes.
\subsubsection {Core property}
This value has to be set externally (e.g. by the user).
\subsubsection {Composite Property}
The position of the composite center of mass is derived from the respective
positions of all of the components of the sub-tree originating with this body,
using classical mechanics:
\begin{equation}
{M_{composite}} \cdot \vec{x}_{composite} = M_{core}  \cdot \vec{x}_{core}  +
\sum_{children}{M_{composite,i} \cdot \vec{y}_{composite,i} }
\label{CoM_cal}
\end{equation}

where $\vec{y}_{composite,i} $ is the position of the respective composite
center of mass for each of the children, expressed with respect to, and in,
the structural axes of this mass body.

\begin{equation}
 \vec{y}_{composite,i} = \vec{x}_i + \relvect T i {this} \left(
 \vec{x}_{composite,i} \right)
\end{equation}

with $\vec{x}_i$ the position of the child's structure point expressed in, and
with respect to the structural axes of this body, and $\relvect T i {this}$ is
the transformation matrix from the structural axes of the child body to the
structural axes of this body.

Again, this is clearly an interative process.


\subsection{Inertia Tensor}
The inertia tensor for the core and composite bodies are found at
\textit{*\_properties.inertia}; the value is referenced to the respective body
axes. The diagonal elements are positive moments of inertia, while the
off-diagonal elements are negative products of inertia.


\subsubsection {Core Property}
This value has to be set externally (e.g. by the user).
\subsubsection {Composite Property}
Computation of the composite body inertia tensor is a multi-step process:
\begin{enumerate}
 \item Compute the inertia tensor for the core-body, referenced to the
 composite-body body-axes, rather than the core-body body-axes.  Since the two
 sets of axes are aligned, we can use the parallel axis theorem:
 \begin{equation}\label{eq:inertia_offset}
  \inertia_{core:comp} = \inertia_{core:core} + M_{core} \begin{bmatrix} y^2 +
  z^2 & -xy & -xz \\-xy & x^2 + z^2 & -yz \\ -xz & -yz & x^2+y^2 \end{bmatrix}
 \end{equation}
 where $x$, $y$, and $z$ represent the position of the core center of mass
 relative to the composite center of mass, expressed in the composite
 body-axes.
 \item For each child, transform a copy of the child's composite-body inertia
 tensor so that it is referenced to this (i.e. the parent) body's
 composite-body body-axes, rather than its own.  This is a multi-step process:
 \begin{enumerate}
  \item Compute the position of the origin of the child's composite-body-axes
  relative to that of this body.
  \item Compute the orientation of the child's composite-body-axes relative to
  that of this body.
  \item Use the orientation data to apply a rotational transformation to the
  child's composite-body inertia tensor such that it references the parent's
  composite-body-axes.  This step is necessary because the composite-body-axes
  for the child body are, in general, not aligned with those for the parent
  body. Consider
  \begin{equation*}
   \tau = \inertia \alpha
  \end{equation*}
  Hence, $\tau_{parent}$ can be expressed as
  \begin{equation*}
   \tau_{parent} = \inertia_{parent} \alpha_{parent}
  \end{equation*}
  and also expressed as a transformation of the same expression in the child
  frame:
  \begin{equation*}
   \tau_{parent} = T_{child \rightarrow parent} \left( \inertia_{child}
   \left( T_{parent \rightarrow child} \left( \alpha_{parent} \right) \right)
   \right)
  \end{equation*}
 Consequently,\begin{equation}
   \inertia_{parent} = T_{child \rightarrow parent}  \inertia_{child}
   T_{parent \rightarrow child}
  \end{equation}
  This term represents the inertia tensor in a set of axes aligned with the
  parent composite-body-axes, with an origin that still matches that of the
  child composite-body-axes.
  \item Evaluate and add the parallel-axis theorem addition term in equation
  \ref{eq:inertia_offset}, such that:
  \begin{equation}\label{eq:inertia_child}
   \inertia_{child:parent} = T_{child \rightarrow parent}
   \inertia_{child:child}   T_{parent \rightarrow child} + M_{core}
   \begin{bmatrix} y^2 + z^2 & -xy & -xz \\-xy & x^2 + z^2 & -yz \\ -xz & -yz
   & x^2+y^2 \end{bmatrix}
  \end{equation}
  where $x$, $y$, and $z$ represent the position of the child composite-body
  center of mass relative to the parent composite-body center of mass,
  expressed in the parent composite-body-axes.  $M_{child}$ is the mass of the
  child composite-body.

 \end{enumerate}
 \item Add to the adjusted parent body inertia tensor (equation
 \ref{eq:inertia_offset}) the resulting inertia tensors for each of the
 children (equation \ref{eq:inertia_child})to give the composite body inertia
 tensor for the parent.\begin{equation}
  \inertia_{comp:comp} = \inertia_{core:comp} + \sum_{children}
  \inertia_{child:parent}
 \end{equation}

 \end{enumerate}




\subsection{Inverse Inertia}
The inverse-inertia tensor for a body is found at
\textit{inverse\_inertia}; the value is referenced to the respective body
axes.The inverse inertia is only needed where torques are going to be applied
to a
MassBody, and then only if the MassBody has a dynamic state to respond to
those torques (i.e. if it is contained in a DynBody).  Furthermore, since torques (and
forces) are only applied to a mass tree in its entirety, the only inverse
inertia that is needed is that of the composite properties of the root body.
Consequently, the inverse inertia is only computed if the MassBody is contained in a
DynBody, if it is at the root of its tree (including single-entity trees), and
then only the composite inertia is inverted.

\subsection{Transformation from Structure to Body}
The transformation from the strucutral-axes to the body-axes is found at
\textit{*\_properties.T\_parent\_this} and at
\textit{*\_properties.Q\_parent\_this} (for a transformation matrix and
quaternion set, respectively).  This value is set at initialization for the
core-properties;

\subsubsection {Core Property}
This value has to be set externally (e.g. by the user) at initialization; it
is fixed for the duration of the simulation.
\subsubsection {Composite Property}
Since both sets of body axes are aligned, the value set for the
core-properties gets copied into the composite-properties at initialization.

\subsection{Attaching Bodies}
The process by which bodies are attached is outlined in the Detailed Design
section (\ref{sec:attaching_bodies})







